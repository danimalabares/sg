\input{/Users/daniel/github/config/preamble-por.sty}
\input{/Users/daniel/github/config/thms-por.sty}

\begin{document}

{\Huge Geometria simpl\'etica}

Além do material do curso, uso bastante Lee, Intro. to Smooth Manifolds, e \href{https://www.damtp.cam.ac.uk/user/tong/dynamics.html}{Tong, Lectures on Classical Mechanics}.

\section{Aula 1}
\subsection{Origem da geometria simpl\'etica}
\begin{itemize}
	\item Formula\c c\~ao da geom\'etrica da mec\^anica (s\'ec XIX).
	\item Vers\~ao moderna, 1960-70.
	\item Diferentes descrip\c c\~oes da mec\^anica cl\'asica:
		\begin{itemize}
			\item Newtoniano: $F=ma$, ecua\c c\~ao diferencial ordin\'aria de segunda ordem.
			\item Lagrangiano: princ\'ipio gravitacional (Eq. E-L). Following Tong, these equations are:

			\item Hamiltoniano.
		\end{itemize}
\end{itemize}

\subsection{Formalismo hamiltoniano (simplificado)}

This happened in the 1880's (according to Tong).


\begin{itemize}
	\item Espa\c co de base $\mathbb{R}^{2}=\{(p,q)\} $ (conjunto de estados)
	\item Fun\c c\~ao Hamiltoniana $H\in C^{\infty}(\mathbb{R}^{2m} )$.
	\item Campo Hamiltoniano: $X_{H}\in \mathfrak{X}(\mathbb{R}^{2n})$.
		\begin{align*}
			X_{H}=\begin{pmatrix}  \frac{\partial H}{\partial p_{i}}\\-\frac{\partial H}{\partial q_{i}} \end{pmatrix} = \left(\begin{tabular}{c|c}0&$\operatorname{Id}_{n}$\\ \hline
		$-\operatorname{Id}_{n}$&0
\end{tabular}\right)
\end{align*}

	Which coincides with Lee's formula
\begin{align*}
	\dot x^{i}(t)&=\frac{\partial H}{\partial y^{i}}(x(t),y(t)),\\
	\dot y^{i}(t)&=-\frac{\partial H}{\partial x^{i}}(x(t),y(t))
\end{align*}
where Lee defined the \textit{\textbf{Hamiltonian vector field}} as the  \textit{analogue of the gradient with respect to the symplectic form}, that is, satisfying $\omega(X_{H},Y)=dH(Y)$ for any vector field $Y$.

Also look at Tong's formulation:
\begin{align*}
	\dot p_{i}&=-\frac{\partial H}{\partial q_{i}}\\
	\dot q_{i}&=\frac{\partial H}{\partial p_{i}}\\
	-\frac{\partial L}{\partial t}&=\frac{\partial H}{\partial t}
\end{align*}
where $L$ is the Lagrangian and the Hamiltonian function $H$ is obtained as the Legendre transform of the Langrangian. Tong shows how the Hamiltonian formalism allows to replace the $n$ $2^{\operatorname{nd}}$ order differential equations by $2n$ $1^{\operatorname{st}}$ order differential equations for $q_{i}$ and $p_{i}$.

\begin{quotation}
	In practice, for solving problems, this isn't particularly helful. But, as we shall see, conceptually it's very useful!
\end{quotation}
At least for me, it looks like a first insight on why symplectic geometry lives on even-dimensional spaces.
\end{itemize}

\subsection{Evolu\c c\~ao temporal (equa\c c\~oes de Hamilton)}
Curvas integrais
\[c(t)=(q_{i}(t),p_{i}(t))\]
de $X_{H}$, ie.
\[c'(t)=X_{H}(c(t))\iff\begin{cases}
	\dot q_{i}&=\frac{\partial H}{\partial p_{i}} \\
	\dot p_{i} &=\frac{\partial H}{\partial q_{i}}
\end{cases}\]
que s\~ao as \textit{\textbf{Equa\c c\~oes de Hamilton}} (de novo).

\begin{example}
	Part\'icula de massa $m$ em $\mathbb{R}^{3} =\{q_{1},q_{2},q_{3}\} $ sujeita a campo de for\c ca conservativa
	\[F=-\nabla V,\quad V\in C^{\infty}(\mathbb{R}^{3}\]
	\[q(t)=(q_{1},q_{2},q_{3})\]
	Equa\c c\~ao de Newton:
	\[m\ddot q=\partial V(q) \iff m\ddot q_{i}=\frac{\partial V}{\partial q_{i}}(q),\qquad i=1,2,3. \]

Ponto de vista Hamiltoniano:
\begin{itemize}
	\item Espa\c code fase $\mathbb{R}^{5}=\{(q_{i},p_{i})\} $.
	\item Hamiltoniano: $H(p,q)=\frac{1}{2m}\sum_{i}p_{i}^{2} +V(q)$ 
	\item Equa\c c\~oes de Hamilton
	 \[\begin{cases}
	 	\dot q_{i}=p_{i}/m\iff p_{i}=m\dot q_{i}\\
	 	\dot p_{i}=-\frac{\partial V}{\partial q_{i}}\qquad &
	 \end{cases}\]
\end{itemize}

\[\begin{tikzcd}
	H\in C^{\infty}(\mathbb{R}^{2n}) \arrow[r,rightsquigarrow]&\nabla H\arrow[r,rightsquigarrow ,"-J_{0}\nabla H"]&X_{H}
\end{tikzcd}\]
where $J_{0}=\begin{pmatrix}0&-\operatorname{I}\\\operatorname{I}&0\end{pmatrix} $. So it looks like another way of obtaining (defining?) the Hamiltonian vector field is to take the gradient of $H$ and then applying $J_{0}$. So it would be nice to see eventually that this is the same as Lee's definition of "symplectic gradient" so to say.
\end{example}

Compondo $\nabla H$ e $X_{H}$ : taxa de varia\c c\~ao de $H$ ao longo dos fluxos. {\color{persimmon}Mas: o que \'e a composi\c c\~ao de dois campos vetoriais? Tal vez \'e a derivada exterior de $H$, $dH$ em lugar do gradiente de  $H$.}

\begin{itemize}
	\item \textit{\textbf{Fluxo gradiente}}
		\begin{align*}c'(t)&=\nabla H(c(t))\\
			\frac{d}{dt}H(c(t))&=\left<\nabla H(c(t)),c'(t)\right> =\|\nabla H(c(t))\|^{2}
\end{align*}
$\nabla H$ aponta na dire\c c\~ao que $H$ varia\c c\~ao.
\item \textit{\textbf{Fluxo hamiltoniano}} 
	\begin{align*}
		c'(t)& =X_{H}(c(t))\\
		\frac{d}{dt}H(c(t))&=\left<\nabla H(c(t)),c'(t)\right> \\
		&=\left<\nabla H(c(t)),-J_{0}\nabla H(c(t))\right>\\
		&=0
	\end{align*}
	{\color{persimmon}?}, $H\in C^{\infty}(\mathbb{R}^{2n} )$, $H\rightsquigarrow dH\in \Omega^{1}(\mathbb{R}^{2n} )$.
	\item \textit{\textbf{Gradiente.}} $\nabla H(x)\in T_{x}\mathbb{R}^{2n} =\mathbb{R}^{2n}$ \'e \'unico.
\begin{align*}
	g_{0}(\nabla H(x),\cdot )=\left<\nabla H(x),\cdot \right> =dH(x)
\end{align*}
onde $g_{0}$ \'e a m\'etrica Euclidiana. De outra forma,
\begin{align*}
	g_{0}^{\flat}:\mathbb{R}^{2n} &\overset{\sim}{\to  } (\mathbb{R}^{2n})^{*}\\
u&\mapsto g_0(u,\cdot )
\end{align*}
assim,
\[\nabla H(x)\overset{\sim}{\to }dH(x).\]
Analogamente, $X_{H}(x)\in \mathbb{R}^{2n}$ \'e \'unico {\color{persimmon}tal que?}
\[\Omega_0(X_{H}(x),\cdot )=dH(x),\qquad \Omega_0(u,v)=-dJ_0V,\]
ou:
\begin{align*}
	\Omega_0^{\flat}: \mathbb{R}^{2n} &\overset{\sim}{\longrightarrow}(\mathbb{R}^{2n})^{*}\\
	X_{H}(x) &\longleftrightarrow dH(x) 
\end{align*}
\end{itemize}

\begin{remark}
	Note que $\Omega_{q}$ define uma 2-forma ({\color{persimmon}c…?)} em $\mathbb{R}^{2n}=\{(q_{i},p_{i})\} $.
	\[\omega_0=\sum_{i=1}^{n} dq_{i}\wedge dp_{i}\in \Omega_2(\mathbb{R}^{2n}),\]
	$X_{H}$ \'e \'unico tal que $i_{X_{H}}\omega_{0}=dH$. So this was Lee's definition $\ddot \smile$.
\end{remark}

\begin{defn}[tempor\'aria]
	Uma \textit{\textbf{variedade simpl\'etica}} \'e $(M,\omega)$, $\omega\in \Omega^{2}(M)$ localmente isomorfa a $(\mathbb{R}^{2n},\sum_{i}dq_{i}\wedge dp_{i})$.

	[Dessenho mostrando que o pullback da carta coordenada leva $\omega$ em $\sum_{i}dq_{i}\wedge dp_{i}$.

\begin{thm}[de Darboux, em Lee]\leavevmode
	Let $(M,\omega)$ be a $2n$-dimensional symplectic manifold. For any $p \in M$ there are smooth coordinates $(x^{1},\ldots,x^{n},y^{1},\ldots,y^{n})$ centered at $p$ in which $\omega$ has the coordinate representation $\omega=\sum_{i=1}^{n} dx^{i} \wedge dy^{i}$.
\end{thm}

And Lee does a proof using the \textit{theory of time-dependant flows}.

	\[\begin{tikzcd}
		\substack{\text{mec\^anica}  \\ \text{cl\'asica} }\arrow[r,"\text{qu\^antica} "] \arrow[dr]&\substack{\text{Teoria de rep.}  \\ \text{de Lie} }\arrow[d]&\text{geo. Riemanniana} \arrow[dl]\\
		&\text{geo. simpl\'etica}\arrow[d]\arrow[u]\arrow[dl,bend right]\\
		\text{geo. K\"ahler} &\substack{\text{din\'amica}  \\ \text{{\color{persimmon}Lagrangiana?}} } 
	\end{tikzcd}\]
	
\end{defn}

\subsection{\'Algebra linear simpl\'etica}

$V$ espa\c co vetorial real, $\Omega:V\times V\to \mathbb{R}$ forma bilinea ansim\'etrica, i.e. $\Omega\in \Lambda^{2} V^{*}$.

\begin{defn}
	$\Omega$ \'e n\~ao degenerada se $\Omega(u,v)=0\forall v\iff u=0$.

	Following Lee, this can also be stated as: for each nonzero $v\in V$ there exists $w\in V$ such that $\omega(v,w)\neq 0$; and it is equivalent to the linear map $v\mapsto \omega(v,\cdot )\in V^{*}$ being invertible, and also that in terms of some (hence every) basis, the matrix $(\omega_{ij})$ representing $\omega$ is nonsingular.

Ou seja, se
\[\ker \Omega:=\{u\in V|\Omega(u,v)=0\;\forall v\} \]
ent\~ao $\Omega$ \'e n\~ao degenerada se e somente se $\ker (\Omega)=\{0\} $.

$\Omega\in \Lambda^{2} V^{*}$ \'e n\~ao degenerada \'e chamada simpl\'etica. $(V,\Omega)$ \'e um \textit{\textbf{espa\c co vectorial simpl\'etico}}.
\end{defn}

\begin{remark}\leavevmode 
	\begin{enumerate}
		\item $\{e_1,..,e_{n}\} $ base de $V$, $\Omega$ \'e representado por uma matriz antisim\'etrica
	\[A=(A_{ij}),\qquad A_{ij}=\Omega(e_{i},e_{j}),\qquad \Omega(u,v)=u^{\operatorname{t}} A,v.\]

	\item $\Omega$ \'e n\~ao degenerada se e somente se $\det (A)\neq 0$.

		Note que
		 \begin{align*}\det A=\det A^{\operatorname{t}} =\det (-A)=(-1)^{\dim V}\det (A)\\
			 \text{ implica que }\quad  \det A\neq 0\implies m=\dim V=2n
			 \end{align*}
	
	\item $\Omega\in \Lambda^{2} V^{*}$. Defina
		\begin{align*}
			\Omega^{\flat}: V &\longrightarrow V^{*} \\
			u &\longmapsto \Omega(u,\cdot )
		\end{align*}
		note que $\ker \Omega=\ker (\Omega^{\flat} )$, assim $\Omega$ \'e n\~ao degenerada se e somente se  $\Omega^{\flat}$ \'e isomorfismo.
	\end{enumerate}
	\end{remark}

\section{Aula 2}

\subsection{Subespa\c cos de evs}

Sejam $(V,\Omega)$ evs e $V\subseteq V$ subespa\c co.

\begin{defn}
\[W^{\Omega} :=\{u\in |\Omega(u,w)=0\;\forall w\in W\}\]
	Considere a restri\c c\~ao  de $\Omega$ \`a W:
	\[i:W\hookrightarrow V\qquad i^{*} \Omega(\Omega|_{W}\in \Lambda_2W^{*},\]
	ent\~ao
	\[\ker (\Omega|_{W})=\{w\in W |\Omega(w,w')=0\;\forall w'\in W\}=W\cap W^{\Omega}\]
Casos de interesse:
\begin{itemize}
	\item \textit{\textbf{Isotr\'opico}}: $W\subseteq W^{\Omega}$ ($\iff\Omega|_{W}\equiv 0$).
	\item \textit{\textbf{Coisotr\'opico}}: $W^{\Omega} \subseteq W$.
	\item  \textit{\textbf{Lagrangiano}}: $W=W^{\Omega}$.
	\item \textit{\textbf{Simpl\'etico}}: $W\cap W^{\Omega} =\{0\}$ ($\Omega|_{W}$ \'e n\~ao degenerado (=simpl\'etico)). 
\end{itemize}
\end{defn}

\begin{lemma}
	$\dim W+\dim W^{\Omega} =\dim V$.
\end{lemma}
\begin{proof}[Demostra\c c\~ao]
	\begin{align*}
		\Omega^{1}: V &\overset{\sim}{\to }V^{*} \\
	u &\longmapsto \Omega(u,\cdot )
	\end{align*}
	Note que $W^{\Omega} \mapsto \operatorname{Ann}(W)$, assim
	\[\dim W+\dim \operatorname{Ann}(W)'=\dim V\]
\end{proof}

\begin{remark}\leavevmode
	\begin{itemize}
		\item $W\subseteq V$ subespa\c co simpl\'etico  se e somente se  $V=W\oplus W^{\Omega}$.
		\item $W$ isotr\'opico $\implies \dim W\leq \frac{\dim V}{2}$.
		\item $W$ coisotr\'opico $\implies \dim W\geq \frac{\dim V}{2}$.
		\item $W$ Lagrangiano se $\dim W=\frac{\dim V}{2}$.
	\end{itemize}
\end{remark}

De fato, $W$ \'e Lagrangiano se e somente se  $W$ \'e isotr\'opico e $\dim W=\frac{\dim V}{2}$.

\begin{exercise}\leavevmode 
	\begin{itemize}
		\item $(W^{\Omega})^{\Omega}=\Omega$ ($W$ isotr\'opico se e somente se  $W^{\Omega}$).
		\item $(W_1\cap W_2)^{\Omega} =W_1^{\Omega} +W_2^{\Omega}$.
	\end{itemize}
\end{exercise}

\begin{example}\leavevmode 
	\begin{itemize}
		\item Subespa\c cos de dimens\~ao 1 s\~ao isotr\'opicos (subespa\c cos de codimens\~ao 1 s\~ao\\ coisotr\'opicos).
		\item $V=V\oplus W^{*}$, onde $V$ tem a forma $\Omega_{can?}$ e $W$ e $W^{*}$ s\~ao Lagrangianos.
		\item $\mathbb{R}^{2n}$, $\{e_1,\ldots,e_n,f_1,\ldots,f_n\} $ base simpl\'etica, ent\~ao $\operatorname{span} \{e_i,f_i\}$ \'e simpl\'etico, e \\$\operatorname{span} \{e_1,\ldots,e_k\} $ \'e isotr\'opico (se $k=n$ \'e Lagrangiano).
		\item $(V_1,\Omega_1)$ e $(V_2,\Omega_2)$ evs's, $T:V_1\to V_2$ isometr\'ia linear, $\operatorname{graf}(T):=\{(u,Tu):u\in V_1\} \subseteq V_1\times V_2$. $T$ \'e simplectomorfismo se e somente se $\operatorname{graf}(T)$ \'e um subespa\c co Lagrangiano em $V_1\times V_2$.
		\item $\dim \operatorname{graf}(T)=\dim V_1=\frac{1}{2}\dim (V_1\times V_2)$.
		\item $\Omega_{V_1\times \bar{V_2}}((u,Tu),(v,Tv))=\Omega(u,v)-\underbrace{\Omega_2(Tu,Tv)}_{=T^{*} \Omega_2(u,v)}$ ($=0\iff \Omega_1=T^{*} \Omega_2$).
	\end{itemize}
\end{example}

\begin{thm}[Exist\^encia das bases simpl\'eticas]\leavevmode
	Para cualquer $(V,\Omega)$ evs existe uma base simpl\'etica.
\end{thm}
\begin{proof}[Demostra\c c\~ao]
	Seja $e_1\in V\setminus \{0\} $. Como $\Omega$ \'e n\~ao degenerada, existe $f_1\in V$ tal que $\Omega(e_1,f_1)=1$. Considere $W_1=\operatorname{span}\{e_1,f_1\} $. Ent\~ao $\Omega|_{W_1}$ \'e n\~ao degenerado (ie. $W_1$ \'e simpl\'etico), o que acontece se e somente se $V=W_1\oplus W_1^{\Omega}$. Assim, existem $e_2\neq 0$ in $W_1^{\Omega}$ e $f_2\in W_1^{\Omega}$ tal que $\Omega(e_2,f_2)=1$, etc… ($V=W_1\oplus  W_2\oplus \ldots\oplus W_n$). O conjunto $\{e_1,\ldots,e_n,f_1,\ldots,f_n\} $ \'e uma base simpl\'etica.
\end{proof}

\begin{exercise}
	$V$ ev de dimens\~ao $2n$ e $\Omega\in \Lambda^{2} V^{*}$ \'e n\~ao degenerada se e somente se $\Omega^{n}=\Omega\wedge \ldots\wedge \Omega\in \Lambda^{2n} V^{*} \neq 0$.
\end{exercise}

\subsection{Equival\^encia entre ev's simpl\'eticos}

$(V,\Omega)$ e $(V',\Omega')$ s\~ao \textit{\textbf{equivalentes}} se existe um \textit{\textbf{simplectomorfismo}} linear $\varphi :V\overset{\sim}{\to }V'$ (isometr\'ia linear) tal que
	\[\varphi^{*}\Omega'=\Omega\in \Lambda^{2} V^{*}\] 
onde
\[\varphi^{*} \Omega'(u,v)=\Omega'(\varphi(u),\varphi(v).\]
Dado $(V,\Omega)$ evs, definimos
\[\operatorname{Sp}(V):=\{T\in \operatorname{GL}(V)|T^{*} \Omega=\Omega\} \]
\begin{example}\leavevmode 
	\begin{enumerate}
	\item $V=\mathbb{R}^{2n}$, $\Omega_0(u,v)=-u^{\mathbf{T}} J_0v$ onde $J_0=\begin{pmatrix}0&-\operatorname{I}\\\operatorname{I}&0\end{pmatrix}$, com base can\^onica $\{e_1,\ldots,e_n,$ $f_1,\ldots,f_n\} $. Temos

\begin{equation}\label{eq:1}
	\begin{cases}
				\Omega_0(e_i,e_{j})=0\\
				\Omega_0(e_i,f_j)=\delta_{ij} \\
				\Omega_0(f_i,f_j)=0
				\qquad &
			\end{cases}
\end{equation}

\begin{defn}
	Uma base de $(V,\Omega)$ satisfazendo \cref{eq:1} \'e chamada \textit{\textbf{base simpl\'etica}}.
\end{defn}

Following Lee, Example. 22.2, the condition may be that $\Omega=\sum_{i=1}^{n} \alpha^{i} \wedge \beta^{i}$ where $\alpha^{i}$ and $\beta^{i}$ are just the dual basis covectors of the base $\{A_{1},\ldots,A_{n},B_1,\ldots,B_{n}\} $ of $V$.

\begin{remark}\leavevmode 
		Escolher/Achar uma base simpl\'etica \'e equivalente \`a escolher/achar um simplectomorfismo
	\[(V,\Omega)\overset{\sim}{\to }(\mathbb{R}^{2n},\Omega_{0})\]
\end{remark}

	\item $W$ espa\c co vetorial sobre $\mathbb{R}$, sejam $V=W\oplus W^{*}$, $w,w\in W$ e $\alpha,\alpha \in W^{*}$
		\[\Omega_{\operatorname{?}}( (w,\alpha),(w',\alpha')):=\alpha'(w)-\alpha(w')\]
		\'e n\~ao degenerada e anti-sim\'etrica. Assim,
		\[(W\oplus W^{*},\Omega_{?})\]
	\'e um espa\c co vetorial simpl\'etico.

\begin{remark}
	Se $\{e_1,\ldots,e_n\} $ \'e uma base simpl\'etica de $W$ e $\{f_1,\ldots f_n\} $ \'e a base dual de $W^{*}$, ent\~ao
\[(W\oplus W^{*},\Omega_{?}\cong (\mathbb{R}^{2n},\Omega_0).\]
\end{remark}

Note que ainda que dado 
\[A:W\overset{\sim}{\to}W\]
automorfismo ?,
\[T_{A}:=\begin{pmatrix}A&0\\0&(A^{*} )^{-1}\end{pmatrix}:W\oplus W^{*} \to W\oplus W^{*}\]
\'e simplectomorfismo, ($T_{A}=A\oplus (A^{*} )^{-1}$ ).

\paragraph{Moral:} $\operatorname{GL}(W)\hookrightarrow \operatorname{Sp}(W\oplus W^{*} )$ 
\begin{align*}
	EV&\overset{\text{funtor} }{\rightsquigarrow}EVS\\
	A\circlearrowleft W&\longmapsto W\oplus W^{*} \circlearrowright T_{A}
\end{align*}

\item $V$ ev sobre $\mathbb{C}$, $\dim_{\mathbb{C}}=n$, com produto interno hermitiano
	\[h:V\times V\to \mathbb{C}\]
i.e. satisfazendo
\begin{enumerate}
	\item $h(u,\lambda v)=\lambda h(u,v)$  $\forall \lambda\in \mathbb{C}$,
	\item $h(u,v)=\overline{h(v,w)}$,
	\item $h(u,u)>0$ $\forall u\neq 0$,
\end{enumerate}
pode ser escrito como
\[h(u,v)=g(u,v)+i\Omega(u,v)\]

Agora considere $V$ como espa\c co vetorial sobre $\mathbb{R}$ (de dimens\~ao $2n$ ).

\begin{exercise}\leavevmode 
	\begin{itemize}
		\item $g$ \'e produto interno positivo definido.
		\item $\Omega$ \'e antisim\'etrica, n\~ao degenerada (simpl\'etica).
		\item Ache uma base de $V$ (dica: extens\~ao de base ortonormal de $h$…)
		\item $\operatorname{U}(n)\subset \operatorname{SP}(V,\Omega)$.
	\end{itemize}
\end{exercise}

\item Produto direto: $(V_1,\Omega_1)$, $(V_2,\Omega_2)$ espa\c cos vetoriais.
	\[\begin{tikzcd}
		&V_1\times V_2\arrow[ld,"\pi_1",swap]\arrow[rd,"\pi_2"]\\
		V_1&&V_2
	\end{tikzcd}\]
Tem a forma simpl\'etica \'e o pullback:
	\[\Omega:=\pi_1^{*} \Omega_1+\pi_2^{*} \Omega_2\]
	ou seja,
	\[\Omega((u_1,u_2),(v_1,v_2)):=\Omega_1(u_1,v_1)+\Omega_2(u_2,v_2),\]
	que \'e n\~ao degenerado e antsim\'etrico tamb\'em.
\end{enumerate}
\end{example}

\paragraph{Nota\c c\~ao:} se $(V,\Omega)$ \'e um espa\c co vetorial simpl\'etico, denotamos por $(V,-\Omega):=\bar{V}$, que tamb\'em \'e um evs.

\section{Aula 3}

\section{Aula 4}

\section{Aula 5}

Lembranza da \'ultima aula:
\begin{enumerate}
	\item Defini\c c\~ao de variedade simpl\'etica.
	\item Pelo menos dois exemplos.
	
	\item Forma de volume/orientabilidade.
	
	\item Campos simpl\'eticos/campos hamiltonianos.

	\item Obstru\c c\~ao cohomol\'ogica de para estrutura simpl\'etica.
\end{enumerate}

\paragraph{Hoje:} Fibrados cotangentes.

Seja $Q$ uma variedade e $M :=T^* Q$ o fibrado cotangente.

\paragraph{Lembrando} Se $Q$ \'e uma variedade, $x\in Q$. O \textit{\textbf{espa\c co tangente}} em $ x$ s\~ao derivaç\~oes ou clases de equivalencia de curvas… base local do espa \c co tangente $\partial_{x_i}$…base dual disso \'e base do espa\c co cotangente nesse ponto… o fibrado cotangente $\bigsqcup_{x\in Q}T^*_{x}Q$ \'e variedade suave.

O fibrado cotangente possui uma 1-forma tautol\'ogica definida assim:

\begin{defn}
	$\alpha\in\Omega^{1}(M)$, onde $M:=T^*Q$, dada por
	\[\alpha_p(X)=p(\pi_*(X))\]
	ou seja, como $X$ \'e tangente ao fibrado cotangente, ele est\'a anclado a algum covetor, assim a gente pode evaluar ele no covector. Tamb\'em pode ser pensado como o pullback de um covector em $Q$ baixo a proje\c c\~ao cotangente usual.
\end{defn}

Em coordenadas locais $(x_1,...,$ 

\begin{exercise}\leavevmode 
	\begin{enumerate}
		\item A 1-forma tautol\'ogica $\alpha\in\Omega^{1}(T^*Q)$ \'e a \'unica 1-forma satisfazendo
			\[\forall \mu\in\Omega^{1}(Q),\qquad \mu^*\alpha=\mu\]
			onde pensamos a $\mu$ do lado izquerdo como um mapa  $\mu:Q\to T^*Q$, ie. uma se\c c \~ao do fibrado cotangente, e do lado direito simplesmente como uma 1-corma em $Q$.

	\end{enumerate}
\end{exercise}

\begin{defn}
	$M=T^*Q$, $\alpha\in\Omega^{1}(M)$ ent\~ao a \textit{\textbf{forma simpl\'etica can\'onica}} de $T^*Q$ \'e
	\[\omega_{\operatorname{can}}=-d\alpha\]
\end{defn}

\begin{remark}\leavevmode
	\begin{itemize}
		\item $d\omega_{\operatorname{can}}=-d^{2}\alpha=0$.
		\item Formalmente $\omega=\sum_{i=1}^{n}dx_i\wedge d\xi_i$
	\end{itemize}
\end{remark}

Assim, temos uma variedade simpl\'etica can\'onica associada a toda variedade, $\\(T^* Q,\omega_{\operatorname{can}}$.

\begin{remark}\leavevmode 
	\begin{itemize}
		\item Dado $B\in\Omega^{2}(Q)$ com $dB=0$, a forma
			 \[\omega_{B}\omega_{\operatorname{can}}+\pi^*B\]
			 \'e simpl\'etica e o termo $\pi^*B$ se chama de \textit{\textbf{magn\'etico}}.

			\item Se $Q$ \'e Riemanniana com m\'etrica $g$ temos o mapa induzido
				\begin{align*}
					g^{\sharp}: TQ &\longrightarrow T^*Q \\
					u &\longmapsto g(u,\cdot )
				\end{align*}
				Assim, o pullback the $\omega_{\operatorname{can}}$ \'e uma forma simpl\'etica em $TQ$.

	Al \'em disso, a m\'etrica nos fornece de uma fun\c c\~ao Hamiltoniana dada por $H\in C^{\infty}(TQ)$, $H(v)=\frac{1}{2}g(v,v)=\frac{1}{2}\|v\|^{2}$.

	Veremos que o fluxo Hamiltoniano de $H$ em $(TQ,\omega)$ \'e fluxo geod\'esico em $Q$.

	Tem dois generaliza\c c\~oes naturais:
	\begin{itemize}
		\item  $\bar{H}(v)=\frac{1}{2}g(u,v)+V(x)$ com $V\in C^{\infty}(Q)$, mec\^anica cl\'asica.

		\item $H(v)=\frac{1}{2}g(v,v)$ com respeito a $\omega_{B}$.
	\end{itemize}
	\end{itemize}
\end{remark}

\begin{question}[Projeto?]
	Exist\^encia de \'orbitas peri\'odicas em n\'iveis de energia?
\end{question}

\begin{defn}
	O \textit{\textbf{levantamiento cotangente}} de um difeomorfismo (na mesma dire\c c\~ao do difeomorfismo) \'e $\varphi:Q_1\overset{\sim}{\to }Q_2$ \'e $\hat{\varphi}=((T\varphi)^{*})^{-1}$.
\end{defn}

\begin{question}
	Preserva a forma can\'onica?
\end{question}

\begin{prop}
	Sim. $\hat{\varphi}:T^*Q_1\to T^*Q_2$ satisfaz $\hat{\varphi}^*\alpha_2=\alpha_1$ onde $\alpha_i$ \'e a forma tautol\'ogica, para $i=1,2$. Isso implica que  $\hat{\varphi}^*\omega_2=\omega_1$.
\end{prop}

Isso implica que temos um funtor $Q\rightsquigarrow T^*Q$ que se chama de \textit{\textbf{funtor cotagente}} e permite levar problemas de geometria diferencial para a geometria simpl \'etica.

\begin{proof}[Demostra\c c\~ao]
	\[\begin{tikzcd}
		T^*Q_1\arrow[r,"\varphi"]\arrow[d,"\pi_1"]&T^*Q_2\arrow[d,"\pi_2"]\\
		Q_1\arrow[r,"\varphi"]&Q_2
	\end{tikzcd}\]
	A clave dessa prova \'e que o diagrama commuta, assim pode se-trocar um termo $\pi_2\circ \hat{\varphi}$ por $\varphi \circ \pi_1$.
\end{proof}

O funtor que produzimos $\operatorname{Dif}(Q)\hookrightarrow \operatorname{Simp }(T^*Q$ n\~ao e fiel (surjetivo), ie. existem simplectomorfismos no fibrado cotangente que n\~ao vem de difeomorfismos na variedade.

\begin{remark}
	Dada uma 1-forma $A\in\Omega^{1}$. Pode se-produzir um mapa no cotangente simplesmente trasladando por $A$:
	\begin{align*}
		T_A: T^*Q &\longrightarrow T^*Q \\
		(x,\xi) &\longmapsto (x,\xi+A_x)
	\end{align*}
	que n\~ao pode ser um levantamento porque se projecta na identidade!

	\begin{exercise}
		$T_A$ \'e um simplectomofrismo $\iff$ $dA=0$.
	\end{exercise}
\end{remark}

Mas, como sabemos quais simplectomorfismos no cotangente s\~ao sim levantamentos de difeomorfismos na variedade?

\begin{exercise}
	Seja $F:T^*Q\to T^*Q$ um simplectomorfismo. Quando $F=\hat{\varphi}$ \'e levantamento de algum $\varphi:Q\overset{\sim}{\to } Q $. Pois, isso acontece $\iff$ $F$ preserva a forma tautol\'ogica, ie. $F^*\alpha=\alpha$.
\end{exercise}

\begin{remark}
	Levantamento cotangente de campos de vetores. Come\c ca com um campo $X\in\mathfrak{X}(Q)$, integra para obter um fluxo $\varphi_{t}$, que \'e uma fam\'ilia de difeomorfismos na variedada, voc\^e sabe levantar isso com o funtor obtendo outro fluxo (porque levantamento de fluxo \'e fluxo) $\hat{\varphi}_{t}$, e diferenciando obt\'em $\hat{X}\in\mathfrak{X}(T^*Q)$.
\end{remark}

\begin{remark}
	Para cualquer fibrado vetorial $E \to  M$, podemos ver a se\c c\~oes $\Gamma(E)$ como um subconjunto das fun \c c\~oes suaves na variedade $C^{\infty}(E)$---s\~ao as fun\c c\~oes lineares nas fibras. A\'i tem um modo natural de definir para cualquer campo vetorial $X\in\Gamma(TQ)\subseteq C^{\infty}(T^*Q)$ uma fun\c c\~ao, $H_X(p)=p(X_{\pi(p)}=\alpha(\hat{X})$.
\end{remark}

\begin{prop}
	$\hat{X}=$ campo Hamiltoniano de $H_X$.
\end{prop}

\end{document}
