\input{/Users/daniel/github/config/preamble-por.sty}
%This preamble is in github.com/danimalabares/config

\begin{document}

{\Huge Geometria simpl\'etica}

\tableofcontents

\section{Aula 1}
Além do material do curso, uso bastante Lee, Intro. to Smooth Manifolds, e \href{https://www.damtp.cam.ac.uk/user/tong/dynamics.html}{Tong, Lectures on Classical Mechanics}.

\subsection{Origem da geometria simpl\'etica}
\begin{itemize}
	\item Formula\c c\~ao da geom\'etrica da mec\^anica (s\'ec XIX).
	\item Vers\~ao moderna, 1960-70.
	\item Diferentes descrip\c c\~oes da mec\^anica cl\'asica:
		\begin{itemize}
			\item Newtoniano: $F=ma$, ecua\c c\~ao diferencial ordin\'aria de segunda ordem.
			\item Lagrangiano: princ\'ipio gravitacional (Eq. E-L). Following Tong, these equations are:

			\item Hamiltoniano.
		\end{itemize}
\end{itemize}

\subsection{Formalismo hamiltoniano (simplificado)}

This happened in the 1880's (according to Tong).


\begin{itemize}
	\item Espa\c co de base $\mathbb{R}^{2}=\{(p,q)\} $ (conjunto de estados)
	\item Fun\c c\~ao Hamiltoniana $H\in C^{\infty}(\mathbb{R}^{2m} )$.
	\item Campo Hamiltoniano: $X_{H}\in \mathfrak{X}(\mathbb{R}^{2n})$.
		\begin{align*}
			X_{H}=\begin{pmatrix}  \frac{\partial H}{\partial p_{i}}\\-\frac{\partial H}{\partial q_{i}} \end{pmatrix} = \left(\begin{tabular}{c|c}0&$\operatorname{Id}_{n}$\\ \hline
		$-\operatorname{Id}_{n}$&0
\end{tabular}\right)
\end{align*}

	Which coincides with Lee's formula
\begin{align*}
	\dot x^{i}(t)&=\frac{\partial H}{\partial y^{i}}(x(t),y(t)),\\
	\dot y^{i}(t)&=-\frac{\partial H}{\partial x^{i}}(x(t),y(t))
\end{align*}
where Lee defined the \textit{\textbf{Hamiltonian vector field}} as the  \textit{analogue of the gradient with respect to the symplectic form}, that is, satisfying $\omega(X_{H},Y)=dH(Y)$ for any vector field $Y$.

Also look at Tong's formulation:
\begin{align*}
	\dot p_{i}&=-\frac{\partial H}{\partial q_{i}}\\
	\dot q_{i}&=\frac{\partial H}{\partial p_{i}}\\
	-\frac{\partial L}{\partial t}&=\frac{\partial H}{\partial t}
\end{align*}
where $L$ is the Lagrangian and the Hamiltonian function $H$ is obtained as the Legendre transform of the Langrangian. Tong shows how the Hamiltonian formalism allows to replace the $n$ $2^{\operatorname{nd}}$ order differential equations by $2n$ $1^{\operatorname{st}}$ order differential equations for $q_{i}$ and $p_{i}$.

\begin{quotation}
	In practice, for solving problems, this isn't particularly helful. But, as we shall see, conceptually it's very useful!
\end{quotation}
At least for me, it looks like a first insight on why symplectic geometry lives on even-dimensional spaces.
\end{itemize}

\subsection{Evolu\c c\~ao temporal (equa\c c\~oes de Hamilton)}
Curvas integrais
\[c(t)=(q_{i}(t),p_{i}(t))\]
de $X_{H}$, ie.
\[c'(t)=X_{H}(c(t))\iff\begin{cases}
	\dot q_{i}&=\frac{\partial H}{\partial p_{i}} \\
	\dot p_{i} &=\frac{\partial H}{\partial q_{i}}
\end{cases}\]
que s\~ao as \textit{\textbf{Equa\c c\~oes de Hamilton}} (de novo).

\begin{example}
	Part\'icula de massa $m$ em $\mathbb{R}^{3} =\{q_{1},q_{2},q_{3}\} $ sujeita a campo de for\c ca conservativa
	\[F=-\nabla V,\quad V\in C^{\infty}(\mathbb{R}^{3}\]
	\[q(t)=(q_{1},q_{2},q_{3})\]
	Equa\c c\~ao de Newton:
	\[m\ddot q=\partial V(q) \iff m\ddot q_{i}=\frac{\partial V}{\partial q_{i}}(q),\qquad i=1,2,3. \]

Ponto de vista Hamiltoniano:
\begin{itemize}
	\item Espa\c code fase $\mathbb{R}^{5}=\{(q_{i},p_{i})\} $.
	\item Hamiltoniano: $H(p,q)=\frac{1}{2m}\sum_{i}p_{i}^{2} +V(q)$ 
	\item Equa\c c\~oes de Hamilton
	 \[\begin{cases}
	 	\dot q_{i}=p_{i}/m\iff p_{i}=m\dot q_{i}\\
	 	\dot p_{i}=-\frac{\partial V}{\partial q_{i}}\qquad &
	 \end{cases}\]
\end{itemize}

\[\begin{tikzcd}
	H\in C^{\infty}(\mathbb{R}^{2n}) \arrow[r,rightsquigarrow]&\nabla H\arrow[r,rightsquigarrow ,"-J_{0}\nabla H"]&X_{H}
\end{tikzcd}\]
where $J_{0}=\begin{pmatrix}0&-\operatorname{I}\\\operatorname{I}&0\end{pmatrix} $. So it looks like another way of obtaining (defining?) the Hamiltonian vector field is to take the gradient of $H$ and then applying $J_{0}$. So it would be nice to see eventually that this is the same as Lee's definition of "symplectic gradient" so to say.
\end{example}

Compondo $\nabla H$ e $X_{H}$ : taxa de varia\c c\~ao de $H$ ao longo dos fluxos. {\color{persimmon}Mas: o que \'e a composi\c c\~ao de dois campos vetoriais? Tal vez \'e a derivada exterior de $H$, $dH$ em lugar do gradiente de  $H$.}

\begin{itemize}
	\item \textit{\textbf{Fluxo gradiente}}
		\begin{align*}c'(t)&=\nabla H(c(t))\\
			\frac{d}{dt}H(c(t))&=\left<\nabla H(c(t)),c'(t)\right> =\|\nabla H(c(t))\|^{2}
\end{align*}
$\nabla H$ aponta na dire\c c\~ao que $H$ varia\c c\~ao.
\item \textit{\textbf{Fluxo hamiltoniano}} 
	\begin{align*}
		c'(t)& =X_{H}(c(t))\\
		\frac{d}{dt}H(c(t))&=\left<\nabla H(c(t)),c'(t)\right> \\
		&=\left<\nabla H(c(t)),-J_{0}\nabla H(c(t))\right>\\
		&=0
	\end{align*}
	{\color{persimmon}?}, $H\in C^{\infty}(\mathbb{R}^{2n} )$, $H\rightsquigarrow dH\in \Omega^{1}(\mathbb{R}^{2n} )$.
	\item \textit{\textbf{Gradiente.}} $\nabla H(x)\in T_{x}\mathbb{R}^{2n} =\mathbb{R}^{2n}$ \'e \'unico.
\begin{align*}
	g_{0}(\nabla H(x),\cdot )=\left<\nabla H(x),\cdot \right> =dH(x)
\end{align*}
onde $g_{0}$ \'e a m\'etrica Euclidiana. De outra forma,
\begin{align*}
	g_{0}^{\flat}:\mathbb{R}^{2n} &\overset{\sim}{\to  } (\mathbb{R}^{2n})^{*}\\
u&\mapsto g_0(u,\cdot )
\end{align*}
assim,
\[\nabla H(x)\overset{\sim}{\to }dH(x).\]
Analogamente, $X_{H}(x)\in \mathbb{R}^{2n}$ \'e \'unico {\color{persimmon}tal que?}
\[\Omega_0(X_{H}(x),\cdot )=dH(x),\qquad \Omega_0(u,v)=-dJ_0V,\]
ou:
\begin{align*}
	\Omega_0^{\flat}: \mathbb{R}^{2n} &\overset{\sim}{\longrightarrow}(\mathbb{R}^{2n})^{*}\\
	X_{H}(x) &\longleftrightarrow dH(x) 
\end{align*}
\end{itemize}

\begin{remark}
	Note que $\Omega_{q}$ define uma 2-forma ({\color{persimmon}c…?)} em $\mathbb{R}^{2n}=\{(q_{i},p_{i})\} $.
	\[\omega_0=\sum_{i=1}^{n} dq_{i}\wedge dp_{i}\in \Omega_2(\mathbb{R}^{2n}),\]
	$X_{H}$ \'e \'unico tal que $i_{X_{H}}\omega_{0}=dH$. So this was Lee's definition $\ddot \smile$.
\end{remark}

\begin{defn}[tempor\'aria]
	Uma \textit{\textbf{variedade simpl\'etica}} \'e $(M,\omega)$, $\omega\in \Omega^{2}(M)$ localmente isomorfa a $(\mathbb{R}^{2n},\sum_{i}dq_{i}\wedge dp_{i})$.

	[Dessenho mostrando que o pullback da carta coordenada leva $\omega$ em $\sum_{i}dq_{i}\wedge dp_{i}$.

\begin{thm}[de Darboux, em Lee]\leavevmode
	Let $(M,\omega)$ be a $2n$-dimensional symplectic manifold. For any $p \in M$ there are smooth coordinates $(x^{1},\ldots,x^{n},y^{1},\ldots,y^{n})$ centered at $p$ in which $\omega$ has the coordinate representation $\omega=\sum_{i=1}^{n} dx^{i} \wedge dy^{i}$.
\end{thm}

And Lee does a proof using the \textit{theory of time-dependant flows}.

	\[\begin{tikzcd}
		\substack{\text{mec\^anica}  \\ \text{cl\'asica} }\arrow[r,"\text{qu\^antica} "] \arrow[dr]&\substack{\text{Teoria de rep.}  \\ \text{de Lie} }\arrow[d]&\text{geo. Riemanniana} \arrow[dl]\\
		&\text{geo. simpl\'etica}\arrow[d]\arrow[u]\arrow[dl,bend right]\\
		\text{geo. K\"ahler} &\substack{\text{din\'amica}  \\ \text{{\color{persimmon}Lagrangiana?}} } 
	\end{tikzcd}\]
	
\end{defn}

\subsection{\'Algebra linear simpl\'etica}

$V$ espa\c co vetorial real, $\Omega:V\times V\to \mathbb{R}$ forma bilinea ansim\'etrica, i.e. $\Omega\in \Lambda^{2} V^{*}$.

\begin{defn}
	$\Omega$ \'e n\~ao degenerada se $\Omega(u,v)=0\forall v\iff u=0$.

	Following Lee, this can also be stated as: for each nonzero $v\in V$ there exists $w\in V$ such that $\omega(v,w)\neq 0$; and it is equivalent to the linear map $v\mapsto \omega(v,\cdot )\in V^{*}$ being invertible, and also that in terms of some (hence every) basis, the matrix $(\omega_{ij})$ representing $\omega$ is nonsingular.

Ou seja, se
\[\ker \Omega:=\{u\in V|\Omega(u,v)=0\;\forall v\} \]
ent\~ao $\Omega$ \'e n\~ao degenerada se e somente se $\ker (\Omega)=\{0\} $.

$\Omega\in \Lambda^{2} V^{*}$ \'e n\~ao degenerada \'e chamada simpl\'etica. $(V,\Omega)$ \'e um \textit{\textbf{espa\c co vectorial simpl\'etico}}.
\end{defn}

\begin{remark}\leavevmode 
	\begin{enumerate}
		\item $\{e_1,..,e_{n}\} $ base de $V$, $\Omega$ \'e representado por uma matriz antisim\'etrica
	\[A=(A_{ij}),\qquad A_{ij}=\Omega(e_{i},e_{j}),\qquad \Omega(u,v)=u^{\operatorname{t}} A,v.\]

	\item $\Omega$ \'e n\~ao degenerada se e somente se $\det (A)\neq 0$.

		Note que
		 \begin{align*}\det A=\det A^{\operatorname{t}} =\det (-A)=(-1)^{\dim V}\det (A)\\
			 \text{ implica que }\quad  \det A\neq 0\implies m=\dim V=2n
			 \end{align*}
	
	\item $\Omega\in \Lambda^{2} V^{*}$. Defina
		\begin{align*}
			\Omega^{\flat}: V &\longrightarrow V^{*} \\
			u &\longmapsto \Omega(u,\cdot )
		\end{align*}
		note que $\ker \Omega=\ker (\Omega^{\flat} )$, assim $\Omega$ \'e n\~ao degenerada se e somente se  $\Omega^{\flat}$ \'e isomorfismo.
	\end{enumerate}
	\end{remark}

\section{Aula 2}

\subsection{Subespa\c cos de evs}

Sejam $(V,\Omega)$ evs e $V\subseteq V$ subespa\c co.

\begin{defn}
\[W^{\Omega} :=\{u\in |\Omega(u,w)=0\;\forall w\in W\}\]
	Considere a restri\c c\~ao  de $\Omega$ \`a W:
	\[i:W\hookrightarrow V\qquad i^{*} \Omega(\Omega|_{W}\in \Lambda_2W^{*},\]
	ent\~ao
	\[\ker (\Omega|_{W})=\{w\in W |\Omega(w,w')=0\;\forall w'\in W\}=W\cap W^{\Omega}\]
Casos de interesse:
\begin{itemize}
	\item \textit{\textbf{Isotr\'opico}}: $W\subseteq W^{\Omega}$ ($\iff\Omega|_{W}\equiv 0$).
	\item \textit{\textbf{Coisotr\'opico}}: $W^{\Omega} \subseteq W$.
	\item  \textit{\textbf{Lagrangiano}}: $W=W^{\Omega}$.
	\item \textit{\textbf{Simpl\'etico}}: $W\cap W^{\Omega} =\{0\}$ ($\Omega|_{W}$ \'e n\~ao degenerado (=simpl\'etico)). 
\end{itemize}
\end{defn}

\begin{lemma}
	$\dim W+\dim W^{\Omega} =\dim V$.
\end{lemma}
\begin{proof}[Demostra\c c\~ao]
	\begin{align*}
		\Omega^{1}: V &\overset{\sim}{\to }V^{*} \\
	u &\longmapsto \Omega(u,\cdot )
	\end{align*}
	Note que $W^{\Omega} \mapsto \operatorname{Ann}(W)$, assim
	\[\dim W+\dim \operatorname{Ann}(W)'=\dim V\]
\end{proof}

\begin{remark}\leavevmode
	\begin{itemize}
		\item $W\subseteq V$ subespa\c co simpl\'etico  se e somente se  $V=W\oplus W^{\Omega}$.
		\item $W$ isotr\'opico $\implies \dim W\leq \frac{\dim V}{2}$.
		\item $W$ coisotr\'opico $\implies \dim W\geq \frac{\dim V}{2}$.
		\item $W$ Lagrangiano se $\dim W=\frac{\dim V}{2}$.
	\end{itemize}
\end{remark}

De fato, $W$ \'e Lagrangiano se e somente se  $W$ \'e isotr\'opico e $\dim W=\frac{\dim V}{2}$.

\begin{exercise}\leavevmode 
	\begin{itemize}
		\item $(W^{\Omega})^{\Omega}=\Omega$ ($W$ isotr\'opico se e somente se  $W^{\Omega}$).
		\item $(W_1\cap W_2)^{\Omega} =W_1^{\Omega} +W_2^{\Omega}$.
	\end{itemize}
\end{exercise}

\begin{example}\leavevmode 
	\begin{itemize}
		\item Subespa\c cos de dimens\~ao 1 s\~ao isotr\'opicos (subespa\c cos de codimens\~ao 1 s\~ao\\ coisotr\'opicos).
		\item $V=V\oplus W^{*}$, onde $V$ tem a forma $\Omega_{can?}$ e $W$ e $W^{*}$ s\~ao Lagrangianos.
		\item $\mathbb{R}^{2n}$, $\{e_1,\ldots,e_n,f_1,\ldots,f_n\} $ base simpl\'etica, ent\~ao $\operatorname{span} \{e_i,f_i\}$ \'e simpl\'etico, e \\$\operatorname{span} \{e_1,\ldots,e_k\} $ \'e isotr\'opico (se $k=n$ \'e Lagrangiano).
		\item $(V_1,\Omega_1)$ e $(V_2,\Omega_2)$ evs's, $T:V_1\to V_2$ isometr\'ia linear, $\operatorname{graf}(T):=\{(u,Tu):u\in V_1\} \subseteq V_1\times V_2$. $T$ \'e simplectomorfismo se e somente se $\operatorname{graf}(T)$ \'e um subespa\c co Lagrangiano em $V_1\times V_2$.
		\item $\dim \operatorname{graf}(T)=\dim V_1=\frac{1}{2}\dim (V_1\times V_2)$.
		\item $\Omega_{V_1\times \bar{V_2}}((u,Tu),(v,Tv))=\Omega(u,v)-\underbrace{\Omega_2(Tu,Tv)}_{=T^{*} \Omega_2(u,v)}$ ($=0\iff \Omega_1=T^{*} \Omega_2$).
	\end{itemize}
\end{example}

\begin{thm}[Exist\^encia das bases simpl\'eticas]\leavevmode
	Para cualquer $(V,\Omega)$ evs existe uma base simpl\'etica.
\end{thm}
\begin{proof}[Demostra\c c\~ao]
	Seja $e_1\in V\setminus \{0\} $. Como $\Omega$ \'e n\~ao degenerada, existe $f_1\in V$ tal que $\Omega(e_1,f_1)=1$. Considere $W_1=\operatorname{span}\{e_1,f_1\} $. Ent\~ao $\Omega|_{W_1}$ \'e n\~ao degenerado (ie. $W_1$ \'e simpl\'etico), o que acontece se e somente se $V=W_1\oplus W_1^{\Omega}$. Assim, existem $e_2\neq 0$ in $W_1^{\Omega}$ e $f_2\in W_1^{\Omega}$ tal que $\Omega(e_2,f_2)=1$, etc… ($V=W_1\oplus  W_2\oplus \ldots\oplus W_n$). O conjunto $\{e_1,\ldots,e_n,f_1,\ldots,f_n\} $ \'e uma base simpl\'etica.
\end{proof}

\begin{exercise}
	$V$ ev de dimens\~ao $2n$ e $\Omega\in \Lambda^{2} V^{*}$ \'e n\~ao degenerada se e somente se $\Omega^{n}=\Omega\wedge \ldots\wedge \Omega\in \Lambda^{2n} V^{*} \neq 0$.
\end{exercise}

\subsection{Equival\^encia entre ev's simpl\'eticos}

$(V,\Omega)$ e $(V',\Omega')$ s\~ao \textit{\textbf{equivalentes}} se existe um \textit{\textbf{simplectomorfismo}} linear $\varphi :V\overset{\sim}{\to }V'$ (isometr\'ia linear) tal que
	\[\varphi^{*}\Omega'=\Omega\in \Lambda^{2} V^{*}\] 
onde
\[\varphi^{*} \Omega'(u,v)=\Omega'(\varphi(u),\varphi(v).\]
Dado $(V,\Omega)$ evs, definimos
\[\operatorname{Sp}(V):=\{T\in \operatorname{GL}(V)|T^{*} \Omega=\Omega\} \]
\begin{example}\leavevmode 
	\begin{enumerate}
	\item $V=\mathbb{R}^{2n}$, $\Omega_0(u,v)=-u^{\mathbf{T}} J_0v$ onde $J_0=\begin{pmatrix}0&-\operatorname{I}\\\operatorname{I}&0\end{pmatrix}$, com base can\^onica $\{e_1,\ldots,e_n,$ $f_1,\ldots,f_n\} $. Temos

\begin{equation}\label{eq:1}
	\begin{cases}
				\Omega_0(e_i,e_{j})=0\\
				\Omega_0(e_i,f_j)=\delta_{ij} \\
				\Omega_0(f_i,f_j)=0
				\qquad &
			\end{cases}
\end{equation}

\begin{defn}
	Uma base de $(V,\Omega)$ satisfazendo \cref{eq:1} \'e chamada \textit{\textbf{base simpl\'etica}}.
\end{defn}

Following Lee, Example. 22.2, the condition may be that $\Omega=\sum_{i=1}^{n} \alpha^{i} \wedge \beta^{i}$ where $\alpha^{i}$ and $\beta^{i}$ are just the dual basis covectors of the base $\{A_{1},\ldots,A_{n},B_1,\ldots,B_{n}\} $ of $V$.

\begin{remark}\leavevmode 
		Escolher/Achar uma base simpl\'etica \'e equivalente \`a escolher/achar um simplectomorfismo
	\[(V,\Omega)\overset{\sim}{\to }(\mathbb{R}^{2n},\Omega_{0})\]
\end{remark}

	\item $W$ espa\c co vetorial sobre $\mathbb{R}$, sejam $V=W\oplus W^{*}$, $w,w\in W$ e $\alpha,\alpha \in W^{*}$
		\[\Omega_{\operatorname{?}}( (w,\alpha),(w',\alpha')):=\alpha'(w)-\alpha(w')\]
		\'e n\~ao degenerada e anti-sim\'etrica. Assim,
		\[(W\oplus W^{*},\Omega_{?})\]
	\'e um espa\c co vetorial simpl\'etico.

\begin{remark}
	Se $\{e_1,\ldots,e_n\} $ \'e uma base simpl\'etica de $W$ e $\{f_1,\ldots f_n\} $ \'e a base dual de $W^{*}$, ent\~ao
\[(W\oplus W^{*},\Omega_{?}\cong (\mathbb{R}^{2n},\Omega_0).\]
\end{remark}

Note que ainda que dado 
\[A:W\overset{\sim}{\to}W\]
automorfismo ?,
\[T_{A}:=\begin{pmatrix}A&0\\0&(A^{*} )^{-1}\end{pmatrix}:W\oplus W^{*} \to W\oplus W^{*}\]
\'e simplectomorfismo, ($T_{A}=A\oplus (A^{*} )^{-1}$ ).

\paragraph{Moral:} $\operatorname{GL}(W)\hookrightarrow \operatorname{Sp}(W\oplus W^{*} )$ 
\begin{align*}
	EV&\overset{\text{funtor} }{\rightsquigarrow}EVS\\
	A\circlearrowleft W&\longmapsto W\oplus W^{*} \circlearrowright T_{A}
\end{align*}

\item $V$ ev sobre $\mathbb{C}$, $\dim_{\mathbb{C}}=n$, com produto interno hermitiano
	\[h:V\times V\to \mathbb{C}\]
i.e. satisfazendo
\begin{enumerate}
	\item $h(u,\lambda v)=\lambda h(u,v)$  $\forall \lambda\in \mathbb{C}$,
	\item $h(u,v)=\overline{h(v,w)}$,
	\item $h(u,u)>0$ $\forall u\neq 0$,
\end{enumerate}
pode ser escrito como
\[h(u,v)=g(u,v)+i\Omega(u,v)\]

Agora considere $V$ como espa\c co vetorial sobre $\mathbb{R}$ (de dimens\~ao $2n$ ).

\begin{exercise}\leavevmode 
	\begin{itemize}
		\item $g$ \'e produto interno positivo definido.
		\item $\Omega$ \'e antisim\'etrica, n\~ao degenerada (simpl\'etica).
		\item Ache uma base de $V$ (dica: extens\~ao de base ortonormal de $h$…)
		\item $\operatorname{U}(n)\subset \operatorname{SP}(V,\Omega)$.
	\end{itemize}
\end{exercise}

\item Produto direto: $(V_1,\Omega_1)$, $(V_2,\Omega_2)$ espa\c cos vetoriais.
	\[\begin{tikzcd}
		&V_1\times V_2\arrow[ld,"\pi_1",swap]\arrow[rd,"\pi_2"]\\
		V_1&&V_2
	\end{tikzcd}\]
Tem a forma simpl\'etica \'e o pullback:
	\[\Omega:=\pi_1^{*} \Omega_1+\pi_2^{*} \Omega_2\]
	ou seja,
	\[\Omega((u_1,u_2),(v_1,v_2)):=\Omega_1(u_1,v_1)+\Omega_2(u_2,v_2),\]
	que \'e n\~ao degenerado e antsim\'etrico tamb\'em.
\end{enumerate}
\end{example}

\paragraph{Nota\c c\~ao:} se $(V,\Omega)$ \'e um espa\c co vetorial simpl\'etico, denotamos por $(V,-\Omega):=\bar{V}$, que tamb\'em \'e um evs.

\section{Aula 3}

\section{Aula 4}

\section{Aula 5}

Lembranza da \'ultima aula:
\begin{enumerate}
	\item Defini\c c\~ao de variedade simpl\'etica.
	\item Pelo menos dois exemplos.
	
	\item Forma de volume/orientabilidade.
	
	\item Campos simpl\'eticos/campos hamiltonianos.

	\item Obstru\c c\~ao cohomol\'ogica de para estrutura simpl\'etica.
\end{enumerate}

\paragraph{Hoje:} Fibrados cotangentes.

\subsection{Forma tautol\'ogica no fibrado cotangente}

Seja $Q$ uma variedade e $M :=T^* Q$ o fibrado cotangente.

\paragraph{Lembrando} Se $Q$ \'e uma variedade, $x\in Q$. O \textit{\textbf{espa\c co tangente}} em $ x$ s\~ao derivaç\~oes ou clases de equivalencia de curvas… base local do espa \c co tangente $\partial_{x_i}$…base dual disso \'e base do espa\c co cotangente nesse ponto… o fibrado cotangente $\bigsqcup_{x\in Q}T^*_{x}Q$ \'e variedade suave.

O fibrado cotangente possui uma 1-forma tautol\'ogica definida assim:

\begin{defn}
	$\alpha\in\Omega^{1}(M)$, onde $M:=T^*Q$, dada por
	\[\alpha_p(X)=p(\pi_*(X))\]
	ou seja, como $X$ \'e tangente ao fibrado cotangente, ele est\'a anclado a algum covetor, assim a gente pode evaluar ele no covector. Tamb\'em pode ser pensado como o pullback de um covector em $Q$ baixo a proje\c c\~ao cotangente usual.
\end{defn}

\begin{defn}[Monitoria]
	\[\begin{tikzcd}
		T^*M=\{(p,\xi)|\xi:T_pM\to \mathbb{R}\text{ linear} \} \arrow[d,"\pi"]\\
		M
	\end{tikzcd}\]
	A \textit{\textbf{forma tautologica}}  \'e $\lambda$ dada por
	\[\lambda_{(p,\xi)}(v)\in\mathbb{R},\qquad v\in T_{(p,\xi)(T^*M)}\]
	\'e igual a
	\[\xi(d\pi_{(q,\xi)(v)})\]
	ussando o mapa
\[T_{(p,\xi)}(T^* M)\overset{d\pi_{(p,\xi)}}{\longrightarrow}T_{p}M\]
\end{defn}

Em coordenadas locais $(x_1,...,x_n,y_1,\ldots,y_n)$ do espaço cotangente, temos que
\[\lambda=\sum_{i=1}^nA_i dx_i+\sum_{i=1}^nB_i dy_i\]
Avaliando $\lambda$ nos vectores can\'onicos $\frac{\partial}{\partial x_j}\Big|_{(p,\xi)}$ e $\frac{\partial}{\partial y_j}$ notamos que $A_i=\xi\left( \frac{\partial}{\partial x_j} \right) $ pois a diferencial de $\pi$ faz as $B_j$ ser zero.

\begin{exercise}\leavevmode 
	\begin{enumerate}
		\item A 1-forma tautol\'ogica $\alpha\in\Omega^{1}(T^*Q)$ \'e a \'unica 1-forma satisfazendo
			\[\forall \mu\in\Omega^{1}(Q),\qquad \mu^*\alpha=\mu\]
			onde pensamos a $\mu$ do lado izquerdo como um mapa  $\mu:Q\to T^*Q$, ie. uma se\c c \~ao do fibrado cotangente, e do lado direito simplesmente como uma 1-corma em $Q$.

	\end{enumerate}
\end{exercise}

\begin{defn}
	$M=T^*Q$, $\alpha\in\Omega^{1}(M)$ ent\~ao a \textit{\textbf{forma simpl\'etica can\'onica}} de $T^*Q$ \'e
	\[\omega_{\operatorname{can}}=-d\alpha\]
\end{defn}

\begin{remark}\leavevmode
	\begin{itemize}
		\item $d\omega_{\operatorname{can}}=-d^{2}\alpha=0$.
		\item Formalmente $\omega=\sum_{i=1}^{n}dx_i\wedge d\xi_i$
	\end{itemize}
\end{remark}

Assim, temos uma variedade simpl\'etica can\'onica associada a toda variedade, $\\(T^* Q,\omega_{\operatorname{can}}$.

\begin{remark}\leavevmode 
	\begin{itemize}
		\item Dado $B\in\Omega^{2}(Q)$ com $dB=0$, a forma
			 \[\omega_{B}\omega_{\operatorname{can}}+\pi^*B\]
			 \'e simpl\'etica e o termo $\pi^*B$ se chama de \textit{\textbf{magn\'etico}}.

			\item Se $Q$ \'e Riemanniana com m\'etrica $g$ temos o mapa induzido
				\begin{align*}
					g^{\sharp}: TQ &\longrightarrow T^*Q \\
					u &\longmapsto g(u,\cdot )
				\end{align*}
				Assim, o pullback the $\omega_{\operatorname{can}}$ \'e uma forma simpl\'etica em $TQ$.

	Al \'em disso, a m\'etrica nos fornece de uma fun\c c\~ao Hamiltoniana dada por $H\in C^{\infty}(TQ)$, $H(v)=\frac{1}{2}g(v,v)=\frac{1}{2}\|v\|^{2}$.

	Veremos que o fluxo Hamiltoniano de $H$ em $(TQ,\omega)$ \'e fluxo geod\'esico em $Q$.

	Tem dois generaliza\c c\~oes naturais:
	\begin{itemize}
		\item  $\bar{H}(v)=\frac{1}{2}g(u,v)+V(x)$ com $V\in C^{\infty}(Q)$, mec\^anica cl\'asica.

		\item $H(v)=\frac{1}{2}g(v,v)$ com respeito a $\omega_{B}$.
	\end{itemize}
	\end{itemize}
\end{remark}

\begin{question}[Projeto?]
	Exist\^encia de \'orbitas peri\'odicas em n\'iveis de energia?
\end{question}

\begin{defn}
	O \textit{\textbf{levantamiento cotangente}} de um difeomorfismo (na mesma dire\c c\~ao do difeomorfismo) \'e $\varphi:Q_1\overset{\sim}{\to }Q_2$ \'e $\hat{\varphi}=((T\varphi)^{*})^{-1}$.
\end{defn}

\begin{question}
	Preserva a forma can\'onica?
\end{question}

\begin{prop}
	Sim. $\hat{\varphi}:T^*Q_1\to T^*Q_2$ satisfaz $\hat{\varphi}^*\alpha_2=\alpha_1$ onde $\alpha_i$ \'e a forma tautol\'ogica, para $i=1,2$. Isso implica que  $\hat{\varphi}^*\omega_2=\omega_1$.
\end{prop}

Isso implica que temos um funtor $Q\rightsquigarrow T^*Q$ que se chama de \textit{\textbf{funtor cotagente}} e permite levar problemas de geometria diferencial para a geometria simpl \'etica.

\begin{proof}[Demostra\c c\~ao]
	\[\begin{tikzcd}
		T^*Q_1\arrow[r,"\varphi"]\arrow[d,"\pi_1"]&T^*Q_2\arrow[d,"\pi_2"]\\
		Q_1\arrow[r,"\varphi"]&Q_2
	\end{tikzcd}\]
	A clave dessa prova \'e que o diagrama commuta, assim pode se-trocar um termo $\pi_2\circ \hat{\varphi}$ por $\varphi \circ \pi_1$.
\end{proof}

O funtor que produzimos $\operatorname{Dif}(Q)\hookrightarrow \operatorname{Simp }(T^*Q$ n\~ao e fiel (surjetivo), ie. existem simplectomorfismos no fibrado cotangente que n\~ao vem de difeomorfismos na variedade.

\begin{remark}
	Dada uma 1-forma $A\in\Omega^{1}$. Pode se-produzir um mapa no cotangente simplesmente trasladando por $A$:
	\begin{align*}
		T_A: T^*Q &\longrightarrow T^*Q \\
		(x,\xi) &\longmapsto (x,\xi+A_x)
	\end{align*}
	que n\~ao pode ser um levantamento porque se projecta na identidade!

	\begin{exercise}
		$T_A$ \'e um simplectomofrismo $\iff$ $dA=0$.
	\end{exercise}
\end{remark}

Mas, como sabemos quais simplectomorfismos no cotangente s\~ao sim levantamentos de difeomorfismos na variedade?

\begin{exercise}
	Seja $F:T^*Q\to T^*Q$ um simplectomorfismo. Quando $F=\hat{\varphi}$ \'e levantamento de algum $\varphi:Q\overset{\sim}{\to } Q $. Pois, isso acontece $\iff$ $F$ preserva a forma tautol\'ogica, ie. $F^*\alpha=\alpha$.
\end{exercise}

\begin{remark}
	Levantamento cotangente de campos de vetores. Come\c ca com um campo $X\in\mathfrak{X}(Q)$, integra para obter um fluxo $\varphi_{t}$, que \'e uma fam\'ilia de difeomorfismos na variedada, voc\^e sabe levantar isso com o funtor obtendo outro fluxo (porque levantamento de fluxo \'e fluxo) $\hat{\varphi}_{t}$, e diferenciando obt\'em $\hat{X}\in\mathfrak{X}(T^*Q)$.
\end{remark}

\begin{remark}
	Para cualquer fibrado vetorial $E \to  M$, podemos ver a se\c c\~oes $\Gamma(E)$ como um subconjunto das fun \c c\~oes suaves na variedade $C^{\infty}(E)$---s\~ao as fun\c c\~oes lineares nas fibras. A\'i tem um modo natural de definir para cualquer campo vetorial $X\in\Gamma(TQ)\subseteq C^{\infty}(T^*Q)$ uma fun\c c\~ao, $H_X(p)=p(X_{\pi(p)}=\alpha(\hat{X})$.
\end{remark}

\begin{prop}
	$\hat{X}=$ campo Hamiltoniano de $H_X$.
\end{prop}

\section{Aula 6}

Hoje: Colchete de Poisson, Darboux.

\subsection{Colchete de Poisson}


$M$ variedade, $\omega\in\Omega^2(M)$ n\~ao degenerada (quase-simpl\'etica). Podemos fazer

\begin{align*}
	w^\flat: TM &\longrightarrow T^*M \\
	x &\longmapsto i_X\omega
\end{align*}
So that
\[f\in C^\infty(M)\rightsquigarrow X_f\in\mathfrak{X}(M)\]
e
\[i_{X_f}\omega=df.\]

\begin{defn}
	$f,g\in C^\infty(M)$.
	\begin{align*}
		\{\cdot ,\cdot \}: C^\infty(M)\times C^\infty(M) &\longrightarrow C^\infty(M) \\
		\{f,g\} & =\omega(X_g,X_f)=dg(X_f)=\mathcal{L}_{X_f}g=-\mathcal{L}_{X_g}f 
	\end{align*}
	
\end{defn}

\begin{prop}[Exerc\'icio]
	$d\omega=0\iff \{\cdot ,\cdot \}$ satisfaz identidade de Jacobi.  $\implies (M,\omega)$ simpl\'etica, $\{\cdot ,\cdot \}$ \'e colchete de Lie em $C^\infty(M)$ e isso se chama de um \textit{\textbf{colchete de Poisson em $(M,\omega)$}}.
\end{prop}

\begin{exercise}
	$\{f,gh\} =\{f,g\} h+\{f,h\} g$.
\end{exercise}

\begin{example}
	$\mathbb{R}^{2n}$.
\end{example}

\begin{defn}
	$f,g\in C^\infty(M)$ est\~ao em \textit{\textbf{involu\c c\~ao}} se $\{f,g\} =0$. ie. $X_g$ \'e tangente aos n\'iveis $f=$const (e vice versa).
\end{defn}

\begin{remark}
	Nesse caso, a derivada de $g$ ao longo das curvas integrais de $X_f$ \'e zero.
\end{remark}

\paragraph{Motiva\c c\~ao} $(M,\omega)$ simpl\'etica, $H\in C^\infty(M)$ queremos integrar $X_H$ (ie. resolver $c'(t)=X_H(c(t))$). Suponha que existe $f\in C^\infty(M)$ com $\{f,H\} =0$, chamada \textit{\textbf{integral primeira}}. ie.  $f$ \'e constante ao longo do fluxo Hamiltoniano.

No s\'eculo XIX, quando Poisson vivia, a ideia era que se temos um n\'umero sufieiente de integrais primeiras "independentes", podemos "integrar" $X_H$. (Aqui "integrar" significa dar uma solu\c c\~ao a equa\c c\~ao diferencial do fluxo Hamiltoniano).

Em 1810, Poisson deu a f\'ormula
\[\{f,g\} =\sum_{i,j}\frac{\partial f}{\partial p_i}\frac{\partial g}{\partial q_i}-\frac{\partial f}{\partial q_i}\frac{\partial g}{\partial p_i}.\]

\begin{thm}[Poisson]
	$\{f,H\} =0=\{g,H\} \implies \{\{f,g\},H\} =0$.
\end{thm}

\begin{thm}[Jacobi]\leavevmode
	\[\{H,\{f,g\}\} +\{g,\{H,f\}\} +\{f,\{g,H\}\} =0\]
\end{thm}

\paragraph{1880} Lie usou essa identidade no seu trabalho de transforma\c c\~oes (\'algebras de  Lie).

\paragraph{Vers\~ao moderna (sec.  XX) de integrabilidade} Veremos adiante…

\begin{thm}[Arnold-Liouville]\leavevmode
	$(M,\omega)$ de dimens\~ao $2n$ e seu Hamiltoniano $H=f_1$ que \'e a primeira de uma sequencia de $n=\dim M/2$ fun\c c\~oes independentes (as derivadas s\~ao linearmente independentes) $f_2,\ldots,f_n\in C^{\infty}(M)$ tais que $\{f_i,f_j\} =0$ e que $(f_1,\ldots,f_n):M\to \mathbb{R}^{n}$ \'e uma submers\~ao. Ent\~ao
	\[N=\{(f_1,\ldots,f_n)=\text{cte} \} \cong \mathbb{T}^n\]
	se compacto e conexo. Al\'em disso, a din\^amica de $X_H$ em $\mathbb{T}^n$ \'e quase peri\'odica (=\'e um fluxo linear no toro, que pode ser racional ou irracional).
\end{thm}

\begin{remark}[Projeto?]\leavevmode
	Qu\'e acontece com essa din\^amica no toro se perturbamos o sistema? O problema de dois corpos \'e completamente integravel. Por exemplo, a din\^amica da Terra e o Sol pode se-resolver, mas o problema adicionando a Lua \'e o problema de 3 corpos, que ningu\'em sabe c\'omo resolver. Aqui a Lua \'e uma perturba\c c\~ao.

	Teorema KAM, quanto mais irracional \'e o fluxo, mais robusto \'e o toro, mais inest\'avel.
\end{remark}

Em fim, tudo isso para motivar os colchetes de Poisson.

\subsection{Teorema de Darboux}

$(M,\omega)$ variedade simpl\'etica com o colchete $\{\cdot ,\cdot \}$.

\begin{remark}\leavevmode
	\begin{enumerate}
		\item $\omega$ est\'a completamente determinada por $\{\cdot ,\cdot \}$, ie. se duas estruturas simpl\'eticas d\~ao lugar ao mesmo colchete de Poisson, elas s\~ao iguais.Por que?
			\begin{align*}
				\omega^\sharp: T^*M &\longrightarrow TM 
			\end{align*}
			est\'a dada em cada ponto por
			\[(\omega^\sharp )_{ij}=\{x_i,x_j\}\]
			por defini\c c\~ao.
		
		\item A estrutura simpl\'etica can\'onica $\omega_0=\sum_{i}dp_i\wedge dp_i$ em $\mathbb{R}^{2n}$ est\'a determinada (\'e a \'unica tal que) por
			\[\{q_i,q_j\}=0=\{p_i,p_j\},\qquad \{p_i,q_j\} =\delta_{ij}.\]
			\'E como se tivesse uma base simpl\'etica boa em todos os pontos…
	\end{enumerate}
\end{remark}

\begin{thm}[Darboux]\leavevmode
	$(M,\omega)$ simpl\'etica, ent…åo ao redor de todo ponto $x\in M$ existem coordenadas locais $(q_1,\ldots,q_n,p_1,\ldots,p_n)$ tais que $\omega=\sum_{i=1}^ndq_i\wedge dp_i$, ou, equivalentemente vale
		\[\{q_i,q_j\}=0=\{p_i,p_j\},\qquad \{p_i,q_j\} =\delta_{ij}.\]
\end{thm}

Tem um lema que va a provar essencialmente tudo.

\begin{lemma}[Primeiro paso da indu\c c\~ao ]
	Ao redor de qualquer ponto $x\in M$ existem coordenadas $(q,p,y_1,\ldots,y_{2n-2}$ tais que
	\[1=\{p,q\},\quad \{p,y_j\}=0=\{q,y_j\},\qquad \{y_i,y_j\} =\varphi_{ij}(y).\]
Ou seja, a matriz da forma \'e
\[\begin{pmatrix} \begin{pmatrix}0&-1\\1&0\end{pmatrix}& 0&0&\ldots&0\\
0&&&&\\
0&&A(y)&&\\
0&&&&\end{pmatrix} \]
ou seja, temos uma expres\~ao
\[\omega=dq\wedge dp+\omega_N\]
onde $\omega_N$ \'e dada por $A(y)$ e  \'e simpl\'etica.
\end{lemma}

\begin{proof}[Demostra\c c\~ao do Lema]
	\begin{enumerate}[label=\textbf{Paso \arabic*}]
		\item Seja $p$ uma fun\c c\~ao tal que $X_p(x)\neq 0$. Pelo teorema de fluxo tabular (retifica\c c\~ao ) existe uma fun\c c\~ao $q$ tal que $X_p=\frac{\partial}{\partial q}$, de modo que $\{p,q\}=dq(X_p)=1$ e $dp(X_q)=-1$.

		\item En\~ao $X_p$ e $X_q$ s\~ao linearmente independentes, pois $1=\{p,q\} =\omega(X_p,X_q)\neq 0$, o que aconteceria por antisimetria se s\~ao linearmente dependentes. Al\'em disso, comutam, pois
			\[ [X_p,X_q]\overset{\text{aula pasada?} }{=}X_{\{p,q\} =1}=0.\]

			Agora usamos a generaliza\c c\~ao do teorema do fluxo tabular: se $X_1,\ldots,X_k$ s\~ao campos  linearmente independentes e que comutam, ent\~ao existem coordenadas\\ $(x_1,\ldots,x_n)$ dais que $X_i=\frac{\partial}{\partial x_i}$. (Teo. fun\c c\~ao inversa.) Assim, existem coordenadas locais $y_1,\ldots,y_{2n}$ tais que
			\[X_q=\frac{\partial}{\partial y_{2n-1}},\qquad  X_p=\frac{\partial}{\partial y_{2n}}.\]
			Logo
			\[dy_j(X_q)=0=dy_j(X_p)\]
			para $j=1,\ldots,2n-2$.

		\item As diferenciais
		\[dq,dp,dy_1,\ldots,dy_{2n-2}\]
		s\~ao linearmente independentes, pois se
		\[adq+bdp+\sum_{i}c_{ij}y_i=0\]
		pois as $y_i$ j\'a s\~ao LI, e avaliando em $X_i$ obtemos $a=0$, e no $X_q$ que $b=0$.

		Temos um sistema de coordenadas  $(q,p,y_1,\ldots,y_{2n-2}$ ao redor de $x$ tal que as condi\c c\~oes  do teorema salvo a \'ultima se cumplem. Agora veamos que $\{y_i,y_j\}$ n\~ao depende de $p,q$.

		 \item S\'o lembrar que
			 \[X_q=-\frac{\partial}{\partial p},\qquad X_q=\frac{\partial}{\partial q}\]
			 assim
			 \[ \frac{\partial}{\partial p}\{y_i,y_j\} =-\{q,\{y_i,y_j\}\} =0\]
			 onde a segunda igualdade \'e jacobi. Fim.
	\end{enumerate}
\end{proof}

\begin{proof}[Demostra\c c\~ao do Teo. Darboux]
Segue do lema por indu\c c\~ao
\end{proof}

\begin{defn}
Uma \textit{\textbf{estrutura de Poisson}} em uma variedade $M$ \'e
\begin{align*}
	\{\cdot ,\cdot \}: C^\infty(M)\times C^\infty(M) &\longrightarrow C^\infty(M)
\end{align*}
$\mathbb{R}$-bilinear, antisim\'etrica, Jacobi e Leibniz, ie. $\{f,gh\} =\{f,g\} h+\{f,h\} g$.
\end{defn}

\begin{example}\leavevmode 
	\begin{itemize}
		\item $(M,\omega)$ simpl \'etica com $\{f,g\} =\omega(X_g,X_f)$.
	\end{itemize}
\end{example}

\section{Aula 7}

Na aula passada vimos:

\begin{itemize}
	\item Colchetes de Poisson.
	\item Teorema de Darboux. Prova: demostrar que tem relaç\~oes que caractetizam a forma de maneira \'unica.
	\item \'E poss\'ivel descrever estruturas cimpl\'eticas en termos de colchete de Poisson.: Variedades de Poisson. Issto \'e axiomatizar as propriedades b\'asicas do colchete de Poisson. Esses objetos podem ser entendidos como foleaç\~oes simpl\'eticas.
\end{itemize}

\subsection{Subvariedades}

Seja $(M,\omega)$ simpl\'etica e $N\overset{i}{\hookrightarrow }(M,\omega)$. Ent\~ao temos
\[\omega_N=i^* \omega\in\Omega^2(N)\]
que \'e fechada porque o pullback comuta com derivada exterior.

\begin{align*}
	 \ker(\omega_N)& =\{X\in T N:\omega(X,Y)=0\;\forall Y\in T N\} \\
	 &=T N\cap T N^\omega\subseteq T N
\end{align*}

\subsection{Pausa para distribuiç\~oes}
$P$ variedade.

\begin{defn}
	Uma \textit{\textbf{distribuiç\~ao (generalizada)}} em $P$ \'e
	\[P\ni x\longmapsto D_x\subseteq T_xP\text{ subespaço} \]
	e o posto da distribuiç\~ao em $x$ := $\dim D_x$.

	A distribuiç\~ao \'e \textit{\textbf{suave}} se para todo $x_0\in P$, $\forall v\in D_{x_0}$ existe um campo vetorial $X\in\mathfrak{X}(P)$ que extende a $v$ e est\'a contido na distribiuç\~ao no sentido de que $X_x\subseteq D_x\forall x$ e $X_{x_0}=v$.
\end{defn}

\begin{example}
	N\'uncleo de 2-formas \'e um exemplo de distribui\c c\~ao, mas n\~ao \'e suave em geral.
\end{example}

\begin{defn}
	Uma distribui\c c\~ao suave $D\subseteq TP$ \'e dita \textit{\textbf{integravel}} se $\forall x\in P$ existe uma subvariedade $S\ni x$,  $TS=D|_{S}$
\end{defn}

No caso de uma dsitribui\c c\~ao (suave) integr\'avel, por todo ponto passa uma subvariedade integral conexa maximal chamadas \textit{\textbf{folhas}}.

\begin{remark}\leavevmode 
	\begin{itemize}
		\item Distribui\c c\~ao suave, de posto constante \'e a mesma coisa que um subfibrado vetorial $D\subseteq TP$. Nesse caso,

		\begin{thm}[Frobenius]
			$D$ \'e integr\'avel se e somente se \'e \textit{\textbf{involutivo}}, ou seja  \[[\Gamma(D),\Gamma(D)]\subseteq \Gamma(D).\]
		\end{thm}

		\begin{proof}[Demostra\c c\~ao]
			Note que $\implies $ \'e trivial porque se tem uma  variedade que realiza a distribui\c c\~ao, o colchete de Lie sempre vai ser outro campo vetorial tangente.
		\end{proof}
		
	\item Suponha que $D=\ker(\omega)$ com $\omega\in\Omega^2(P)$ \'e suave  $\iff$ $D$ tem posto constante. Aqu\'i $\impliedby$ \'e f\'acil.

	\item Se $d\omega=0 \implies D=\ker \omega$ \'e involutivo.
	\end{itemize}
\end{remark}

\paragraph{Conclus\~ao}  Se $\omega$ \'e uma 2-forma fechada e $D=\ker \omega$ tem posto constante, da lugar a uma folhea\c c\~ao (regular=folhas de mesma dimens\~ao) em $P$.


\subsection{Voltando}

\begin{defn}
	$N$ \'e dita
	\begin{itemize}
		\item \textit{\textbf{isotr\'opica}} quando $T_x N\subseteq T_X N^\omega\iff \omega_N=0\iff \ker \omega_N=T N$.

		\item \textit{\textbf{coisotr\'opica}} quando $T_x N^\omega\subseteq T_x N$.

		\item \textit{\textbf{lagrangiana}} quando $T_x N=T_x N^\omega\iff i^*\omega=\omega_N=0$ e $\dim N=\dim M/2$.

		\item \textit{\textbf{simpl\'etica}} $T_x N\cap(T_x N)^\omega=\{0\}\;\forall x\in N\iff \omega_N$ \'e simpl\'etica.

		\item \textit{\textbf{posto constante}} $T_x N\cap T_x N^\omega\subseteq T_x N\;\forall N$ tem posto constante.
	\end{itemize}
\end{defn}

\begin{example}\leavevmode 
	\begin{itemize}
		\item curvas s\~ao isotr\'opicas.
		\item hipersuperficies s\~ao coisotr\'opicas.
		\item Veremos v\'arios exeplos de subespaços lagrangianos.
	\end{itemize}
\end{example}

\subsubsection{Sobre subvariedades coisotr\'opicas}

\textbf{Isto tamb\'em vale para subvariedades de posto constante.} 

Vamos ver uma vers\~ao geom\'etrica de um exer\'icio da lista 1, onde pegabamos o quociente de um espaço vetorial por el n\'ucleo de uma forma para obter um espaço vetorial simpl\'etico.

\begin{exercise}
	Suponha que as folhas da folhea\c c\~ao s\~ao fibras de uma sobmers\~ao

	\[\begin{tikzcd}
		N\arrow[r,hook]\arrow[d,two heads,swap,"q"]&(M,\omega)\\
		B=N/\sim
	\end{tikzcd}\]
	ent\~ao existe uma forma simpl\'etica $\bar{\omega} \in\Omega^2(B)$ tal que $q^*\bar{\omega} =\omega_N$.
\end{exercise}

\begin{example}
	O fluxo hamiltoniano do oscilador harm\'onico $H(p,q)=\frac{1}{2}\sum_{i}q_i^2+p_i^2$ com $c=1/2$ da  $\mathbb{C}P^{n-1}$
\end{example}

\begin{exercise}
	$\psi:M\to \mathbb{R}^{k}$, $\psi=(\psi_1,\ldots,\psi_k)$. $N=\psi^{-1}(c)$ para $c$ valor regular.
	\begin{itemize}
		\item $N$ coisotr\'opico $\iff \{\psi_i,\psi_j\}|_{N} =0$.

		\item $N$ simpl\'etico $\iff \left( \{\psi_i,\psi_j\} |_{N} \right)_{ij}$ \'e invert\'ivel.
	\end{itemize}
\end{exercise}

\section{Aula 8}

Lembre:

\begin{itemize}
	\item Subvariedades lagrangianas, (co-)isotr\'opicas, simpl\'eticas. Aprofundamos\\ nas coisotr\'opicas (posto constante), como as hipersuprficies ou conjuntos de n\'ivel, que tem uma folhea\c c\~ao, e com condi\c c\~oes  de regularidade pode passar para o espaço quociente, que \'e simpl\'etico, como $\mathbb{C}P^{n}$.
\end{itemize}

\subsection{Alguns exemplos de subvariedades lagrangianas}

\begin{example}
	Dois variedades simpl\'eticas e um difeomorfismo entre elas. Ent\~ao $\varphi$ \'e simplectomorfismo se e s\'o se seu gr\'afico \'e lagrangiano. Talvez isso pode ser ussado para pensar em simplectomorfismos em um objeto cuantico.
\end{example}

\begin{remark}
	Considere
	\begin{align*}
		\varepsilon: M_1 &\longrightarrow M_1\times M_2 \\
		x &\longmapsto (x,\varphi x)
	\end{align*}
	ent\~ao o grafo de $\varphi$ \'e lagrangiano $\iff \omega_1-\varphi^*\omega_2$.
\end{remark}

\begin{example}[no fibrado cotangente]\leavevmode 
	\begin{itemize}
		\item A se\c c\~ao zero $Q\hookrightarrow T^*Q$ \'e nos mostra que $Q$ \'e uma subvariedade lagrangiana.

		\item A fibra (cotangente) de um ponto tamb\'em \'e uma subvariedade lagrangiana de $T^*Q$.

		\item Logo, o espaço de fibras?

		\item Pegue uma subvariedade da base $S\subset Q$. Considera o \textit{\textbf{fibrado conormal}} $N^*S$, $\nu_S^*$. \'E o dual do fibrado tangente. \'E o anulador de $TS$,  $\{(x,\xi)\in T^*Q:x\in S, \xi|_{TxS}=0\}$. Note que \'e um subfibrado do fibrado cotangente.
	\end{itemize}
\end{example}

Os dois exemplos anteriores s\~ao $S=Q$ e  $S=\{x\}$ da seguinte prop:

\begin{prop}
	$N^*S\hookrightarrow T^*Q$ \'e (um subfibrado) uma subvariedade lagrangiana.
\end{prop}

\begin{proof}[Demostra\c c\~ao]
	Ussando coordenadas adaptadas e a forma tautol\'ogica do $T^*Q$, damos coordenadas $N^*Q$ da forma $(x_1,\ldots,x_k,\xi_{k+1},\ldots,\xi_n)$ e assim o pullback da forma tautol\'ogica \'e zero porque ele evalua os covectores $\xi_{\text{grande} }$ em vectores $x_{\text{pequeno} }$.
\end{proof}

\begin{example}
	Uma forma $\mu$ vista como se\c c\~ao do fibrado cotangente pode ser pensada como um mergulho de $Q$ em $T^*Q$.
\end{example}

\begin{prop}
	Essa subvariedade \'e lagrangiana $\iff d\mu=0$.
\end{prop}

\subsection{M\'etodo de Moser}

\begin{upshot}
	Moser's trick is a thing that gives you a diffeomorphism that pulls back $\omega_2$ to $\omega_1$.
\end{upshot}

Dadas dois formas simpl\'eticas numa variedade, como podemos achar um simplectomorfismo entre elas? A ideia do m\'etodo \'e assim:

\begin{enumerate}[label=\textbf{Step \arabic*}]
	\item Interpolar as dois formas mediante uma familia cont\'inua $\omega_t$ de formas simpl\'eticas.

	\item Buscar uma (isotop\'ia) fam\'ilia de difeomorfismos $\varphi_t$ com $\varphi_0=\operatorname{id}$ e tal que $\varphi^*_t\omega_t=\omega_0$. Com isso a gente procura levar o problema para uma EDO.

	\item Os fluxos s\~ao isotop\'ias com uma rela\c c\~ao de comutatividade. Eles correspondem com campos vetoriais. As isotop\'ias em geral est\~ao em correspond\^encia com \textit{\textbf{campos de vetores n\~ao aut\'onomos}}.
\end{enumerate}

\begin{defn}
	Uma fam\'ilia suave de difeomorfismos $\{\phi_t\}$ com $\phi_0=\operatorname{id}$ \'e chamada \textit{\textbf{isotop\'ia}}. Suave significa que $(t,x)\mapsto \varphi_t$ \'e suave.
\end{defn}

\begin{example}
	Fluxos (complets) s\~ao isotop\'ias tq $\phi_s\circ \phi_t=\phi_{s+t}$.
\end{example}

\begin{defn}
	Um \textit{\textbf{campo de vetor $t$-dependente}} ou  \textit{\textbf{n\~ao aut\'onomo}} \'e fam\'ilia suave de campos $X_t\in\mathfrak{X}(M)$. De novo, suave \'e que $(t,x)\mapsto X_t(x)$ \'e suave.
\end{defn}

\[\text{isotop\'ia} \leftrightarrow \text{campos $t$-dependentes}  \]
A diferencia\c c\~ao sempre \'e simples n\'e? Fixa um ponto e varia o tempo, obt\'em uma curva.
\[\phi_t\mapsto X_t(\varphi_t(x)=\frac{d}{d\tau}|_{t=\tau}\varphi_\tau(x).\]

A rec\'iproca \'e mais dif\'icil. A ideia e extender a variedade  \'a $M\times \mathbb{R}$, e considerar $\overline{X}(x,t)=(X_t(x),\frac{d}{dt})$. Esse depende do tempo, assim podemos achar um fluxo $\phi_t$ de $\overline{X}_t$. Aqui se deve extender o fluxo ussando bump functions, assim a gente tem que $\phi_t$ est\'a definido para toda $t$.

Note que $\phi_t(x,s)=(G_t,t+s)$ para alguma fun\c c\~ao $G$ na variedade. Podemos achar uma inversa dela assim:
\[(x,s)=\phi_{-t}(\phi_t(x,s))=G_{-t}(G_t(x,s),t+s),s)\]
ie. a inversa de
\[x\mapsto G_t(x,s)\]
\'e
\[y\mapsto G_{-t}(y,s+t)\]
Logo, 
\[\phi_t(x)=G_t(x,0)\]
\'e uma isotop\'ia e como a derivada do fluxo
\[\frac{d}{dt}\phi_t(x,0)=\overline{X}(G_t(x,0),t)\implies \frac{d}{dt}G_t(x,0)=X_t(x,0)).\]
E \'e isso. Temos a correspondencia.

Voltando ao m\'etodo de Moser, para achar $\varphi^*\omega_1=\omega_0$, pegamos uma isotop\'ia que puxa $\omega_t$ em $\omega_0$, e queremos diferenciar a isotop\'ia. No caso de um fluxo, trata-se da derivada de Lie por defini\c c\~ao.

\begin{lemma}
	$\{\varphi_t\}$ isotop\'ia em $M$, $\{X_t\}$ campo aut\'onomo. Sejam $\eta\in\Omega^k(M)$, $\beta_t\in\Omega^k(M)$. Ent\~ao vale:
	\[\frac{d}{dt}(\varphi^*_t\epsilon)=\varphi^*_t(\mathcal{L}_{X_t}\eta\]
	onde estamos pegando a derivada num tempo $t$ fixo. Da\'i veremos que pela regra da cadeia segue que
	\[\frac{d}{dt}(\varphi^*_t\beta_t)=\varphi^* (\mathcal{L}_{X_t}\beta_t+\frac{d}{dt}\beta_t\]
\end{lemma}

\begin{proof}[Demostra\c c\~ao]
	\begin{enumerate}[label=\alph*.]
		\item Considere os seguintes operadores em $\Omega^\bullet$ :
			\[Q_1(\eta)=\frac{d}{dt}\varphi^*_t\eta,\qquad Q_2(\eta)=\varphi^*_t\mathcal{L}_{X_t}\eta\]
			Da\'i note que esses operadores comutam com a derivada exterior, s\~ao Leibniz respeito ao producto cunha e coincidem em fun\c c\~oes . Da\'i segue que $Q_1=Q_2$.

		\item A regra da cadeia diz que para uma fun\c c\~ao $F(a,b)$,
\[\frac{d}{dt}F(t,t)=\frac{\partial}{\partial a}F(t,t)+\frac{\partial}{\partial b}F(t,t)\]
e olha para $\varphi^*_a\beta_b$ como a $F$. Sustiuindo e ussando  a., o resultado segue.
	\end{enumerate}
\end{proof}

Uma aplica\c c\~ao disso \'e

\begin{thm}[de estabilidade de Moser]\leavevmode
	$M$ compacta, $\{\omega_t\}$ formas simpl\'eticas, $t\in [0,1]$. Se as formas s\~ao todas cohomologas ent\~ao elas s\~ao simplectomorfas, i.e. $[\omega_t]=[\omega_0]\implies \exists \phi_t$ tq $\phi^*_t\omega_t=\omega_0$. Ou, de outra forma, se existe uma fam\'ilia suave de formas $\beta_t$ tais que
	 \[\omega_t=\omega_0+d\beta_t\]
	 ent\~ao existe uma isotop\'ia $\{\varphi_t\}$ tal que $\varphi^*_t\omega_t=\omega_0$.
\end{thm}

\begin{proof}[Demostra\c c\~ao]
	Note que n\~ao \'e imediato que as clases de cohomologia nos dem uma famil\'ia suave, mas \'e equivalente sim (usando decomposi\c c\~ao de Hodge? Tem algo mais simples?). O m\'etodo \'e achar um campo de vetores aut\'onomo resolvendo
	\[i_{X_t}\omega_t=-\frac{d}{dt}\beta_t\]
	pois dela segue que
	\[\mathcal{L}_{X_t}\omega_t=-d\left( \frac{d}{dt}\beta_t \right)\]
	E da\'i a segunda afirma\c c\~ao do lema.
\end{proof}

\section{Aula 9}

Lembre: M\'etodo de Moser.
 
A prova foi:
\begin{proof}[Demostra\c c\~ao]
	Calcule
	\[\frac{d}{dt}\varphi^*_t\omega_t=0\]
	isso implica que
	\[\mathcal{L}_{X_t}\omega_t=-d\left( \frac{d}{dt}\beta_t \right)\]
	e isso que
	\[i_{X_t}\omega_t=-\frac{d}{dt}\beta_t\]
\end{proof}

Com isso conseguimos associar uma isotop\'ia a um campo $t$-dependente (integra\c c\~ao).

\subsection{Aplica \c c\~ao ao teorema de Darboux}

\begin{lemma}
	$X_t$ campo de vetores $t$-dependente, $t\in\mathbb{R}$. Suponha que
	\[X_t|_{x_0}=0\quad \forall t.\]
	Ent\~ao existe uma vizinhança $U\ni x_0$ e uma familia $\varphi_t:U\to M$ de 
	\begin{itemize}
		\item (Inclus\~ao ) $\varphi_0=\operatorname{id}$.
		\item $\frac{d}{dt}\varphi_t(x)=X_t(\varphi_t(x))$
		\item $\varphi_t(x_0)=x_0$
		\item $\varphi _t:U\overset{\operatorname{difeo}}{\longrightarrow}\varphi_t(U)$.
	\end{itemize}
\end{lemma}

\begin{proof}[Demostra\c c\~ao]
	Varia\c c\~ao do caso $M$ compacto
	\[\bar{X}(x,t):=\left( X_t(x),\frac{d}{dt}\right)\quad \text{em $M\times \mathbb{R}$} \]
	\[\bar{X}(x_0,t)=\left( 0,\frac{d}{dt} \right) \]
	assim existe uma curva integral $\gamma(t)=(x_0,t)$ de $ \bar{X}$ por $(x_0,0)$ est\'a definida para toda $t\in\mathbb{R}$.

	Por EDO, existe uma vizinhança $W$ de $(x_0,0)$ em $M\times \mathbb{R}$ onde o fluxo de $\bar{X}$ existe $\forall t\in[0,1]$.

	Tome $U=\bigcap_{w\{M\times \{0\}\}} $.
\end{proof}

Valem a f\'ormula para $\frac{d}{dt}(\varphi_t^*\omega_t)$…

\begin{thm}[Darboux]\leavevmode
	$(M,\omega)$ simpl\'etica, $\dim M=2n$. Para todo $x\in M$ existe uma vizinhança $U\ni x$, aberto  $0\in V\subseteq \mathbb{R}^{2n}$ e um difeomorfismo
	\begin{align*}
		\phi: V\subseteq \mathbb{R}^{2n} &\longrightarrow U\subseteq M \\
		0 &\longmapsto x
	\end{align*}
	tal que 
	\[\phi^* \omega=\sum_{i}dq_i\wedge dp_i.\]

	[Desenho de carta coordenada]
\end{thm}

\begin{proof}[Demostra\c c\~ao]
	Podemos assumir que $M$ \'e bola aberta de $\mathbb{R}^{2n}$ com estrutura sumpl\'etica $\omega$ aribtr\'aria.

	Para usar o m\'etodo de Moser, definamos
	\begin{align*}
		\omega_1& =\omega\\
		\omega_0=\sum_{i}dq_i\wedge dp_i
	\end{align*}
	Podemos assumir que na origem
	\[\omega_1|_{x=0}=\omega_0|_{x=0}\qquad T_0\mathbb{R}^{2n}=\mathbb{R}^{2n}\]
	simplesmente porque qualquer dois formas simpl\'eticas s\~ao equivalentes num espaço vetorial simpletico usando uma mudança de coordenadas.

\begin{itemize}
	\item	Podemos assumir pelo Lema de Poincar\'e que
	\[\omega_1-\omega_0=d\beta,\quad \beta|_{0}=0\]
supondo pela mesma raz\~ao que antes que $\beta|_{0}=0$.
	\item $\omega_t=(1-t)\omega_0+t\omega_1\iff \omega_t=\omega_0+td\beta$ 
\end{itemize}
Precisamos checar que $\omega_t$ s\~ao n\~ao degeneradas numa vizinhança de 0. 

Note que em $x=0$, $\omega_t|_{x=0}=\omega_0|_{x=0}=\omega_1|_{x=0}$, assim $\omega_t |_{x=0}$ \'e n\~ao degenerada para toda $t$, mas precisamos de uma vizinhança, n\~ao s\'o um ponto.

\begin{lemma}
	Se tem uma fam\'ilia $\omega_t|_{x_0}$ \'e simpl\'etica $\forall t$, $t\in [0,1]$, ent\~ao existe uma vizinhança de $x_0$ onde $\omega_t$ \'e n\~ao degenerada $\forall t\in[0,1]$.
\end{lemma}

\begin{proof}[Demostra\c c\~ao]
	Considere
	\[(x,s)\to \det (\omega_s(x))=\text{determinante da matriz que representa a forma} \]
	essa fun\c c\~ao \'e n\~ao zero em zero, assim para cada $t$ existe uma vizinhança onde ela n\~ao \'e zero. Logo, pela compacidade de $[0,1]$, $ \exists $ uma vizinhança $B\ni x_0$ onde $\det (\omega_s(x))$ n\~ao se anula $\forall s\in [0,1]$.
\end{proof}

Ent\~ao j\'a temos essa vizinhança que precisavamos.

Defina $X_t$ como a solu\c c\~ao da equa\c c\~ao de Moser:
\[i_{X_t}\omega_t=-\beta.\]
Como $\beta|_{0}=0\implies X_t|_{x=0}=0\implies \exists \varphi_t$, $t\in [0,1]$.

Pelo lema 1, existe uma vizinhança $V\ni 0$ e 
\begin{align*}
	\varphi_t: V &\longrightarrow B \\
	\varphi_t^*\omega_t&=\omega_0 
\end{align*}
tome $t=1$, $0\in U=\varphi_1(V)$.
\end{proof}

Com esse mesmo m\'etodo a gente consegue provar uma generaliza\c c\~ao do teorema de Darboux.

\subsection{Teorema de Darboux generalizado (Weinstein)}

\begin{thm}
	$Q\overset{i}{\hookrightarrow}M$ subvariedade (mergulhada) e $\omega_0,\omega_1$ em $M$ simpl\'eticas. Suponha que
	\[\omega_0|_{x}=\omega_1|_{x}\qquad \forall x\in Q\]
	ent\~ao existem vizinhanças $U_0$ e $U_1$ de $Q$ em M e um difeomorfismo
	\[\varphi:U_0\overset{\sim}{\longrightarrow}U_1  \]
	tal que 
	\[\varphi^*\omega_1=\omega_0\]
	e que $\varphi(x)=x\;\forall x\in Q$
\end{thm}

\begin{remark}
	O teorema de Darboux \'e quando $Q$ \'e um ponto s\'o!
\end{remark}

\begin{remark}
	A condi\c c\~ao $\omega_0|_{x}=\omega_1|_{x}$ significa que $\omega_0$ e $\omega_1$ coincidem em todo o espaço tangente a $M$ nos pontos de $Q$, n\~ao \'e que o pullback em $Q$ coincide. Tem mais vetores no espaço tangente a $M$.
\end{remark}

	Vamos precisar de um Lema de Poincar\'e relativo.

	\begin{lemma}
		$Q\hookrightarrow M$ subvariedade. Seja $\eta \in\Omega^{k}(M)$, $d\eta=0$,  $i^*\eta=0$. Ent\~ao existe uma vizinhança $U$ de $Q$ em $M$, $\beta\in\Omega^{k}(U)$ tal que
		\begin{align*}
			\eta & =d\beta\\
			\beta|_{x}&=0,\quad \forall x\in Q
		\end{align*}
		($\beta|_{T_xM}=0$ $\forall x\in Q$).

		A ideia aqui \'e simplesmente que podemos achar uma vizinhança de $Q$  que se contrae a $Q$ (retrato por deforma\c c\~ao?)
	\end{lemma}

\begin{proof}[Demostra\c c\~ao]
	Em fim, pelo lema, para $\eta=\omega_1-\omega_0$, $i^*\eta=0$. Compare com a demostra\c c\~ao anterior, $\beta$ se anulava no 0, agora $\eta$ se anula em toda $Q$ (\'e uma vers\~ao param\'etrica disso).

	$Q\hookrightarrow M$ tem vizinhança $U$ onde $\exists \beta\in\Omega^{1}(U)$,
	\[\omega_1-\omega_0=d\beta,\quad \beta |_{x}=0\]
	\begin{itemize}
		\item Seja $\omega_t=(1-t)\omega_0+t\omega_1=\omega_0+t d\beta$.
		\item $\forall t\in [0,1]$, $x\in Q$, $\omega_ t |_{x}=\omega_0|_{x}=\omega_1|_{x}$.
	\end{itemize}
	Pelo lema 2, $x$ tem vizinhança em M onde $\omega_t$ \'e simpl\'etica $\forall t\in [0,1]$.

	Tomando a uni\~ao das vizinhanças, temos vizinhança de $Q$ onde $\omega_t$ simpl\'etico $\forall t\in[0,1]$.

	\textbf{M\'etodo}
	\begin{itemize}
		\item Define $X_t$ por $i_{X_t}\omega=-\beta$. Isso implica que $\frac{d}{dt}\varphi^*_t\omega_t=0$.

		\item Como $\beta|_{x}=0$, ent\~ao $\forall x\in Q$, $X_t |_{x}=0\;\forall x\in Q$.

		\item Pelo lema 1, $\exists U_0$ onde $\varphi_t$ est\'a definido $\forall t\in [0,1]$.

		\item E mais $X_t |_{Q}=0\implies \varphi_t |_{Q}=\operatorname{id}_Q$.

		\item Tome $\phi=\varphi_1$ e $U_1=\varphi_1(U_0)$.
	\end{itemize}
\end{proof}

\subsubsection{Sobre o Lema de Poincar\'e relativo}

O principal ingrediente \'e teorema da vizinhança tubular.

Lembre:

\begin{thm}[Vizinhança tubular]
	$Q\hookrightarrow M$ subvariedade mergulhada. Existe uma vizinhança $Q\subseteq U\subseteq M$ para qual existe $\pi:U\to Q$ tal que 
	\begin{align*}
		\pi\circ i&=\operatorname{id}_Q\\
		i\circ \pi&\simeq \operatorname{id}_U, \quad \text{(homotop\'ia suave)} 
	\end{align*}
\end{thm}

Da\'i, o lema de Poincar\'e segue a existencia de um \textit{\textbf{operador de homotop\'ia}}.

Em geral, quando temos uma homotop\'ia
\begin{align*}
	F: M\times [0,1] &\longrightarrow N \\
	F_0 :M&\to N\\
	F_1:M&\to N
\end{align*}
exsite um operador 
\[H:\Omega^{k}(M) \to \Omega^{k-1}(M)\]
tal que
\[F_1^*\eta-F^*_0\eta=d(H\eta)-Hd\eta\]

Note que no caso de formas fechadas, o termo da direita se anula e a gente prova a invariança homot\'opica da cohomologia. No nosso caso, o operador de homotop\'ia nos da $\eta=dH\eta$j \'a que $d\eta$ se anula em  $Q$.

\subsubsection{Vizinhança tubular}

\begin{thm}
	Existe uma vizinhança $U_0$ de $Q$ em $NQ$ e uma vizinhança $U_1$ de $Q$ em $M$ tais que
	\begin{enumerate}[label=\alph*.]
		\item $U_0\cap (NQ)_x$ \'e convexo $\forall x\in Q$.

		\item Existe um difeomorfismo  $\phi:U_0\overset{\sim}{\longrightarrow} U_1$ tal que $\phi(x)=x$, e $d\phi|_{x}:T_x(NQ)\overset{\operatorname{id}}{\longrightarrow}TM|_{x}$
	\end{enumerate}
\end{thm}

\begin{proof}[Demostra\c c\~ao]
	Idea: aplica\c c\~ao exponencial.
\end{proof}

\section{Monitoria 2}

\begin{prop}
	$\phi:M^{2n}\to \mathbb{R}^{k}$ suave, $c \in\mathbb{R}^{k}$ valor regular.
	 \[N:=\phi^{-1}(c) \text{ coisotr\'opica }\iff \{\phi_i,\phi_j\} |_{N}=0 \]
\end{prop}


\section{Aula 10}

Lembre

\begin{itemize}
	\item Darboux generalizado: duas formas numa subvariedade que coinciden nos pontos da subvariedade, existem vizinhanças da subvariedade simplectomorfas.

	\item A prova disso: usa m\'etodo de Moser.  Para usar o m\'etodo de Moser:
		\begin{lemma}[Poincar\'e relativo]
			$Q \overset{i}{\longrightarrow}M$. $\eta\in\Omega^{k}(M)$ fechada e tal que $i^*\eta=0$. Ent\~ao existe  vizinhança $U\supset Q$ e $\beta\in\Omega^{k-1}(U)$ tal que $\eta=d\beta$ e $\beta|_{Q}=0$.

			(O lema de Poincar\'e usual \'e quando $Q$ \'e um ponto)
		\end{lemma}
		\begin{proof}[Demostra\c c\~ao do lema de Poincar\'e]
			A\'i tem que mergulhar $Q$ no fibrado tangente $NQ$ que \'e um fibrado que n\~ao precisa de m\'etrica para ser definido. Por\'em, na prova a gente intruduiz uma m\'etrica em $Q$ e identifica $NQ$ com  $T^\perp Q$. Da\'i usando a aplica\c c\~ao exponencial conseguimos ver que $Q$ \'e um retrato por deforma\c c\~ao de uma vizinhança dele no $M$---a exponencial \'e a ponte de $NQ$ [a $Q$.

			Isso da uma homotop\'ia
			\begin{align*}
				F_t: U_0 &\longrightarrow U_0 \\
				(x,v) &\longmapsto (x,tv)\\
				F_0&=i\circ \pi\\
				F_1&=\operatorname{id}_{U_0}
			\end{align*}
			Da\'i \'e s\'o pegar o operador de homotop\'ia
			\[\mathcal{H}:\Omega^{k}(U_0)\to \Omega^{k-1}(U_0)\]
			que \'e tal que
			\[F_1^*\eta=F_0^* \eta=\mathcal{H}(d\eta)+d(\mathcal{H}\eta)\]
			\begin{claim}
				O operador de homotop\'ia \'e
				\[H(\eta)=\int_{0}^1I^*_ti_{ \frac{\partial }{\partial t}}(F^*\eta)dt\]
				onde
				\begin{align*}
					F: [0,1]\times U_0 &\longrightarrow U_0 \\
					(t,y) &\longmapsto F_t(y)
				\end{align*}
				e
				\begin{align*}
					I_t: U_0 &\longrightarrow [0,1]\times U_0 \\
					y &\longmapsto (t,y)
				\end{align*}
				de forma que
				\[F_t=F\circ I_t\]
				\textbf{Nota\c c\~ao} Seja
				\begin{align*}
					\tau_t: \mathbb{R}\times U_0 &\longrightarrow \mathbb{R}\times U- \\
					(x,y) &\longmapsto (s+t,y)
				\end{align*}
				de forma que
				\[I_t=\tau_t\circ I_0,\quad F_t=F\circ I_t=F\circ \tau_t\circ I_0\]
				e a conta que a gente faiz \'e
				\begin{align*}
					\frac{d}{dt}F_t^*\eta&=I_0^*\frac{d}{dt}\tau_t^* (F^* \eta)\\
					&=I_0^* \tau_t^* (\mathcal{L}_{\frac{\partial }{\partial }t}F^*_{\eta}\\
					&\overset{\text{Cartan}}{=}I_0^* \tau_t^*\left( di_{ \frac{\partial }{\partial t}}F^* _\eta+i_{\frac{\partial }{\partial t}}d(F^*\eta \right) \\
					&=d\left( I_{t}^* i_{\frac{\partial }{\partial t}}F^*_{\eta}+I_t^ki_{\frac{\partial }{\partial t}}F^*(d\eta) \right) 
				\end{align*}
				e a\'i integramos para obter
				\[F_1^*\eta-F^*_0\eta=d(H\eta)+H(d\eta)\]
				Se $d\eta=0$,  $i^*\eta=0$, $\implies \eta=d(H\eta)$. Defina $\beta=H\eta$. Como  $F_t(x,0)=(x,0)\;\forall x\in Q$, assim
				\[\frac{\partial }{\partial t}F_t(x,0)=0\implies i_{\frac{\partial }{\partial t}}dF_t |_{x\in Q}=0\]
				e por fim
				\[\beta|_{x}=0.\]
			\end{claim}
		\end{proof}
\end{itemize}

Para esse teorema pode imaginar que cada vizinhança de $Q$ \'e uma variedade diferente. Mas ent\~ao a condi\c c\~ao de que as dois formas s \~ao iguais encima de $Q$ j\'a n\~ao faz sentido. Precisamos de um isomorfismo simpl\'etico entre esses espaços tangentes.

\subsection{Darboux generalizado vers\~ao 2.0}

\begin{thm}[Teorema de Darboux generalizado Vers\~ao 2.0]\leavevmode
	\[\begin{tikzcd}
		(M_0,\omega_0)\arrow[dr,"i_0",swap,hook]&&(M_1,\omega_1)\arrow[dl,"i_1",hook]\\
	&Q
	\end{tikzcd}\]
	(arrows reversed)Suponha que temos um isomorfismo de fibrados simpl\'ecticos
	\[\begin{tikzcd}
	TM_0|_{Q}\arrow[rr,"\phi"]\arrow[dr,swap]&&TM_1|_{Q}\arrow[dl]\\
	&Q
	\end{tikzcd}\]
	e tal que $d\phi|_{TQ}:TQ\to TQ$ e $\operatorname{id}_TQ$.

	Ent\~ao $\phi$ estende a um simplectomorfismo
	\[\begin{tikzcd}
	U_0\subset M_0\arrow[rr,"\varphi"]\arrow[dr,swap]&&U_1\subset M_1\arrow[dl]\\
	&Q
	\end{tikzcd}\]
	(arrows backwards) tal que
	\[d\pi|_{Q}=\phi:TM_0|_{Q}\to TM_1|_{Q}\]
	Isto \'e, a derivada do simplectomofismo (entre as vizinhanças de $M_1$ e $M_2$) que obtemos estende o isomorfismo simpl\'eticos dos fibrados tangentes.
\end{thm}

\begin{proof}[Demostra\c c\~ao]
	Podemos reduzir ao caso anterior! Basta achar um difeomorfismo $\psi:U_0\to U_1$ tal que $\psi|_{Q}=\operatorname{id}_Q$ e que $d\psi|_{Q}=\phi$. Nesse caso, $\omega_0$ e $\psi^*\omega_1$ s\~ao dois formas em $U_0$ que coincidem sobre $TM_1|_{Q}$. Vamo l\'a

	Tome dois complementos
	 \begin{align*}
		E_0,\quad TM_0|_{Q}&=TQ\oplus E_0\\
		E_1,\quad TM_1|_{Q}&=T_Q\oplus E_1
	\end{align*}
	Ent\~ao como $\phi$ preserva $T_Q$, ele tamb\'em preserva os complementos, \'e s\'o algebra linear. Isto \'e, $\phi$ se restringe a um isomorfismo
	\[\bar{\phi} :E_0\to E_1\]
	Note que
	\[ \bar{\phi}|_{Q} :TE_0\cong TQ\oplus E_0\to TE_1\cong TQ\oplus E_1\]
	Aqui estamos pegando a derivada do isomorfismo nos fibrados. O importante e que como ele \'e linear, sua derivada \'e ele mesmo (s\'o que a\'i aparecem muitas identifica\c c\~oes):
	\[d\bar{\phi}|_{Q}=\operatorname{id}\oplus \bar{\phi} =\phi\]
Agora pegamos vizinhanças tubulares de $Q$, $V_0\subset E_0$ e $V_1\subset E_1$ e usando a exponancial como antes podemos contraer
\[\begin{tikzcd}
	V_0\arrow[r,"\phi"]\arrow[d,"\phi_0{\color{magenta}= \operatorname{exp}}"]&V_1\arrow[d,"\phi_1"]\\
	U_0\subset M_0\arrow[r,"\psi",dashed]&U_1\subseteq M_1
\end{tikzcd}\]
e todo comuta:
\[\psi=\phi_1\circ \bar{\phi} \circ \phi^{-1}:U_0\overset{\cong }{\longrightarrow} U_1\]
e por fim
\[d\psi|_{Q}=\operatorname{id}\circ d \bar{\phi} \operatorname{id}=\phi\]
\end{proof}

Agora um caso particular:
\[\begin{tikzcd}
	(M,\omega)\arrow[dr,"i",swap]&&(T^*\mathcal{L}\arrow[dl,"i"]\\
&\mathcal{L}
\end{tikzcd}\]
(arrows reversed they are inclusions) onde $Q$ est\'a metida no fibrado cotangente como a se\c c\~ao zero.

\subsection{Teorema das vizinhanças Lagrangianas de Weinstein}


\begin{thm}[das vizinhanças Lagrangianas de Weinstein]\leavevmode
	(As subvariedades Lagrangianas est\~ao definidas "intrinsecamente", pois existe uma vizinhança delas que \'e simplectomorfa a ela como subvariedade no tangente dela)

	Existem vizinhanças $U_0\supseteq \mathcal{L}$ em $T^*\mathcal{L}$ e $U_1\supset eq \mathcal{L}$ em $M$ e um simplectomorfismo
	\[\varphi:U_0\to U_1\]
\end{thm}

\begin{proof}[Demostra\c c\~ao]
	S\'o precisamos de um $\phi$ como no Darboux 2, i.e.,
	\begin{align*}
		\phi: TM|_{\mathcal{L}} &\longrightarrow T(T^*\mathcal{L})|_{\mathcal{L}}
	\end{align*}
	tal que
\[\phi|_{T\mathcal{L}}:T\mathcal{L}\to T\mathcal{L}\quad =\operatorname{id}_{T\mathcal{L}}\]

\begin{lemma}
	Suponha que $\mathcal{L}\hookrightarrow (M,\omega)$ \'e Lagrangiana. Considere $TM|_{\mathcal{L}}$ um fibrado vectorial simpl\'etico. Ent\~ao
	\begin{enumerate}
		\item Existe um subfibrado lagrangiano $E\subseteq TM|_{\mathcal{L}}$ tal que $TM|_{\mathcal{L}}=T\mathcal{L}\oplus E$.

		\item Existe um isomorfismo
			\[\begin{tikzcd}
			TM|_{\mathcal{L}}\arrow[rr,"\cong "]\arrow[dr,swap]&&T\mathcal{L}\oplus (T\mathcal{L})^*\arrow[dl]\\
			&\mathcal{L}
			\end{tikzcd}\]
			onde no espaço $T\mathcal{L}\oplus (T\mathcal{L})^*$ \'e
			\[\nu((X,\alpha),(Y,\beta))=\beta(X)-\alpha(Y)\]

			Lembrando um exerc\'icio da lista 1 (de \'alebra linear) que diz que um subespaço Lagrangiano \'e nos da uma descomposi\c c\~ao do espaço usando o seu dual.

			\begin{proof}[Demostra\c c\~ao do Lema]\leavevmode 

				\begin{enumerate}[label=\textbf{Step \arabic*}]
					\item Todo espaço simpl\'etico induiz uma estrutura complexa compat\'ivel.  Se $L$ \'e lagrangiano, $JL$ tamb\'em e o espaço vetorial (acho que isso coincide com o complemento ortogonal na m\'etrica compat\'ivel). Isso vale para fibrados vetorias.

					\item Note que 
						\begin{align*}
							E &\longrightarrow (T\mathcal{L})^* \\
							u &\longmapsto \omega(\cdot ,u)
						\end{align*}
						\'e um isomorfismo. Isso \'e super elementar de algebra linear.

Tome 
\begin{align*}
	(TM|_{\mathcal{L}},\omega)  &\longrightarrow (T\mathcal{L}\oplus T\mathcal{L}^*,\nu) \\
	(x,u) &\longmapsto (X,\omega(\cdot ,u)
\end{align*}
que \'e que acontece? Ent\~ao,
\begin{align*}
	\nu(T(X,u),T(Y,v)&=\nu((X,\omega(\cdot ,u)),(Y,\omega(\cdot ,v))\\
	&=\omega(X,u)-\omega(Y,u)\\
	&=\omega((X,u),(Y,u))
\end{align*}
				\end{enumerate}
			\end{proof}
	\end{enumerate}
	Da\'i, o lema queda provado simplesmente notando que
	\[\begin{tikzcd}
	&T\mathcal{L}\oplus T\mathcal{L}\arrow[dl,"\cong ",swap]\arrow[dr,"\cong "]\\
	TM|_{\mathcal{L}}\arrow[rr,"\phi"]&&T(T^*\mathcal{L})|_{\mathcal{L}}
	\end{tikzcd}\]
	(diagonal arrows reversed).
\end{lemma}
\end{proof}



\section{Aula 11}

\subsection{Aplica\c c\~ao (de Weinstein): pontos fixos de simplectomorifsmos}

Generaliza o estudo (Poincar\'e-Birkoff) cl\'assico de pontos fixos de aplica\c c\~oes que preservan \'area:

\begin{thm}[\'Ultimo teorema de Poincar\'e]\leavevmode
	Um automorfismo de um anelo que preserva orianta\c c\~ao, \'area e rota a fronteira do anelo em dire\c c\~oes opostas tem um ponto fixo.
\end{thm}

Isso apareceo quando Poincar\'e estudava fluxos em $\mathbb{R}^{3}$.

Consideremos $(M,\omega)$ simpl\'ectica e $M \overset{f}{\longrightarrow}M$ simplectomorfismo. Nos interessa o caso em que $f$ \'e um fluxo hamiltoniano no tempo 1, ie. $f=\varphi^{t=1}_{X_{H_t}}$. Sabemos que
\[\Gamma_f=\{(x,f(x)):x\in M\} \subseteq M\times \bar{M} \]
\'e uma subvariedade lagrangiana, e tamb\'em
$\Delta =\Gamma_{\operatorname{id}_M}=\{(x,x):x\in M\} \subseteq M\times \bar{M}$ 
De forma que os pontos fixos de $f$ s\~ao os pontos de interse\c c\~ao entre $\Gamma_f$ e $\Delta$.

\begin{prop}
	Seja $M$ compacta, $H^{1}_{\operatorname{dR}}(M)=0$. Se $f$ \'e  \textit{\textbf{$C^1$-pr\'oximo}} (convergencia uniforme, \href{https://en.wikipedia.org/wiki/Compact-open_topology#Fr%C3%A9chet_differentiable_functions}{Fr\'echet differentiable}?) da $\operatorname{id}_M$, ent\~ao $f$ tem pelomenos 2 pontos fixos.
\end{prop}

\begin{proof}[Demostra\c c\~ao]
	 Note que $\Delta \cong M$ pelo teorema da vizinhança lagrangiana, como $\Delta$ \'e lagrangiana existe uma vizinhança $U\supseteq \Delta$ simplectomorfa a uma vizinhança $U'$ de  $M\hookrightarrow (T^* M,\omega_{\operatorname{can}}$.

	 \begin{itemize}
	 \item Se $f\in\operatorname{Simp}$ est\'a "perto" da $\operatorname{id}_M$, ent\~ao $\Gamma_f\subseteq U$.

	\item $f$ \'e $C^1$-pr\'oximo da $\operatorname{id}_M$, ent\~ao $\Gamma_f$ corresponde a 1-forma $\mu$ em $T^*M$ (a uma subvariedade $N_\mu$ de  $T^*M$?). (\'E uma gr\'afica de $M$ no fibrado cotangente!)

	\item $\Gamma_f$ lagrangiana  $\implies $ $d\mu=0$ (Lista 2)

	 \item $H^{1}(M)=0\implies \mu=dh$
	
	\item $M$ compacta $\implies $ $h$ tem pelo menos 2 pontos cr\'iticos.
	 \end{itemize}
\end{proof}

\begin{remark}\leavevmode 
	\begin{itemize}
	\item N\~ao podemos abrir m\~ao de $H^{1}(M)=0$. Eg. rota\c c\~ao no toro.

	\item Podemos substituir $H^{1}(M)=0$ por $f$ ser simplectomorfismo Hamiltoniano (ver McDuff-Salomon).
	\end{itemize}
\end{remark}

\begin{question}
	Remover $C^1$-proximidade da identidade? (Pelo menos no caso $f$ hamiltoniano.
\end{question}

\begin{conjecture}[Arnold]
	$M$ simpl\'etica compacta, $f$ simplectomorfismo Hamiltoniano. O n\'umero de pontos fixos de $f$ e maior o igual que o n\'umero m\'inimo de pontos cr\'iticos que uma fun\c c\~ao em $M$ deve ter:
	\begin{align*}
		\text{\# pontos fixos de $f$}&\geq \operatorname{Crit}(M)\\
		\geq  \text{LS category (Lusternik Schninelmann} 
	\end{align*}
Isso est\'a relacionado com o fato de tirar a hip\'otese de que a fun\c c\~ao est\'e pr\'oxima da identidade.
\end{conjecture}

\begin{conjecture}[Outra vers\~ao]
	Para pontos fixos n\~ao degenerados (s\~ao os pontos onde $N_\mu$ e  $M$ se intersectan transversalmente em $T^*M$).
	\begin{align*}
		\text{\# pontos cr\'iticos} &\geq \text{\# m\'inimo de pontos cr\'iticos que fun\c c\~oes de Morse devem ter.}\\
		&\underbrace{\geq }_{\text{desig. Morse} } \sum_{k}\operatorname{Betti}_{k}
	\end{align*}
\end{conjecture}

\paragraph{Projetos} 
\begin{itemize}
\item Conjetura de Arnold (Eliashbag (superf\'icies de Riemann), Hofer-Achander)
\item Homologia de Floer (\'e uma vers\~ao de Homologia de Morse em dimens\~ao infinita)
\end{itemize}

Professor Leonardo vai falar com mais detalhe desses temas.

\subsection{Outra classe de exemplos de variedades simpl\'eticas: \'orbitas coadjuntas}

S\~ao exeplos de redu\c c\~ao simpl\'etica. Isso est\'a relacionado com teor\'ia de Lie. $G \curvearrowright (M,\omega)$ simetr\'ias hamiltonianas. Da\'i vamos produzir uma nova variedade simpl\'etica.

\subsubsection{Revis\~ao de grupos e \'algebras de Lie}

Cada grupo de Lie act\'ua na sua \'algebra de Lie de maneira can\'onica. Da\'i podemos pegar a \'algebra dual. O fato importante \'e que as \'orbitas l\'a tem uma estrutura simpl\'etica.

\begin{defn}
	Um \textit{\textbf{grupo de Lie}} \'e uma variedade $C^\infty$ $G$ munida de estrutura de grupo tal que o produto e a invers\~ao s\~ao fun\c c\~oes  suaves. Os \textit{\textbf{morfismos}} s\~ao homomorfismos de grupos $C^\infty$. \textit{\textbf{Subgrupos de Lie}} s\~ao subvariedades imersas que s\~ao subgrupos.
\end{defn}

\begin{example}\leavevmode
	\begin{itemize}
	\item $\operatorname{GL}(n,\mathbb{R})=\{A\in M_{n\times n}(\mathbb{R}):\det A \neq 0\}$ 
	\item $V$ espaço vetorial, $(V,+)$  \'e grupo de Lie abeliano.
	\item  $S^1$, $S^1\times S^1\times \ldots\times S^1$ s\~ao grupos de Lie abelianos.
	\item Grupos finitos/enumer\'aveis: $\mathbb{Z},\mathbb{Z}_m,\ldots$ 
	\item 
		\begin{exercise}
			$G$ grupo de Lie conexo, ent\~ao o seu recobrimento universal $\tilde{G}$ \'e grupo de Lie.
		\end{exercise}
	\item Subgrupos de $\operatorname{GL}(n,\mathbb{R})$:
		\begin{itemize}
		\item Ortogonal $\operatorname{O}(n)=\{A\in\operatorname{GL}(n,\mathbb{R}):A A^{\mathbf{T}}=\operatorname{id}\}$.

			Mais generalmente, $(V,\left<\cdot,\cdot\right> $ espaço vetorial de produto interno, $\operatorname{O}(V) =\{T:V\to V|\left<Tv,Tv\right> =\left<u,v\right> \}$.

			Considerando
			\begin{align*}
				\psi:\operatorname{GL}(n,\mathbb{R}) &\longrightarrow \operatorname{Sim}(n) \\
				A &\longmapsto A A^{\mathbf{T}}
			\end{align*}
			temos que $\operatorname{id}$ \'e um valor regular, e assim $\operatorname{O}(n)=\psi^{-1}(\operatorname{id})$ \'e uma subvariedade (compacta \'e n\~ao conexa por $\det A=\pm 1$
	
	\item $\operatorname{SL}(n) =\{A\in\operatorname{GL}(n,\mathbb{R}):\det A=1\}$, conexo n\~ao compacto.

	\item $\operatorname{SO}(n) =\operatorname{O}(n) \cap \operatorname{SL}(n)$ compacto conexo

	\item $\operatorname{Sp}(2n) =\{A\in\operatorname{GL}(2,\mathbb{R}):A^{\mathbf{T}}J_0A=J_0\}$ com $J_0=\begin{pmatrix} 0&-I\\I&0 \end{pmatrix} $. \textit{\textbf{Grupo simpl\'etico}}.

	\item $\operatorname{GL}(n,\mathbb{C})=\{A\in M_{n\times n}(\mathbb{C}):\text{invertiveis} \} \overset{\text{aberto} }{\subseteq}M_n(\mathbb{C})=\mathbb{C}^{n^2}$.

	\item $\operatorname{U}(n) =\{A\in\operatorname{GL}(n,\mathbb{C}):A A^*=\operatorname{id}\}$, issto \'e $A^*=\overline{A}^{\mathbf{T}}$, temos $|\det A| =1$ e o mapa
		\[\det :\operatorname{U}(n)\to S^1\]
		e de fato $\operatorname{U}(1) \cong S^1$.

	\item $\operatorname{SU}(n) = \{A\in\operatorname{U}(n) :\det A=1\}$ \textit{\textbf{grupo unit\'ario especial}}
		
		\end{itemize}
	\end{itemize}	
\end{example}

\begin{remark}\leavevmode 
	\begin{itemize}
	\item 	\begin{thm}[de Cartan]\leavevmode
		(F. Warner) Subgrupo fechado de grupo de Lie \'e subgrupo de Lie! (mergulhado)
	\end{thm}

	\item Nem todo grupo de Lie \'e grupo de Lie de matrices. O espaço recobridor de $\operatorname{SL}(2,\mathbb{R})$, por exemplo.
	\end{itemize}
\end{remark}

\subsubsection{Sobre $\operatorname{SU}(2)$}

Sabemos que podemos escrever 
\[\mathbb{R}^{4}=\mathbb{H}=\{a+ib+jc+kd:i^2=j^2=k^2=-1\} \]
e como
\[S^3\hookrightarrow \mathbb{H}\]
$S^3$ herda uma estrutura de grupo de Lie, e de fato
\[S^3\overset{\cong }{\longrightarrow}\operatorname{SU}(2) = \left\{ (\frac{\alpha}{-\beta}\frac{\beta}{\alpha}:\alpha,\beta\in\mathbb{C},|\alpha|^2+|\beta|^2=1\right\} \]
Da\'i,
\[\begin{tikzcd}
	S^3\arrow[r,"\cong "]\arrow[d,"2:1",swap]&\operatorname{SU}(2)\arrow[d,"2:1"]\\
	\mathbb{R}P^{3}\arrow[r,"\cong "]&\operatorname{SO}(3)
\end{tikzcd}\]
Em geral recobrimentos duplos de $\operatorname{SO}(n)$ s\~ao grupos $\operatorname{Spin}(n)$. Para $n\geq 2$ s\~ao recobrimentos universais. S\~ao grupos de simetr\'ias de particulas que se chaman fermiones. A ideia \'e que a gente precisa dois voltas para virar a flecha que t\'a parada na particula.

Por \'ultimo vamos ver por qu\'e e que $\mathbb{R}P^{3}\cong \operatorname{SO}(3)$. Do mesmo jeito que $\mathbb{R}P^{2}$ \'e o hemisferio norte da esfera com os pontos no bordo identificados, $\mathbb{R}P^{3}$ \'e uma bola fechada em $\mathbb{R}^3$ com pontos antipodais no bordo identificados. Podemos pensar que os pontos de $\mathbb{R}P^{3}$ s\~ao rota\c c\~oes  de \'angulo $\pi\sqrt{x^2+y^2+z^2}$.


\section{Aula 12}




\end{document}
