\input{/Users/daniel/github/config/preamble.sty}

\usepackage[style=authortitle-terse,backend=bibtex]{biblatex}
\addbibresource{bibliography.bib}

\begin{document}

\begin{minipage}{\textwidth}
	\begin{minipage}{1\textwidth}
		Geometria Simpl\'etica \hfill Daniel González Casanova Azuela
		
		{\small Profs. Henrique Bursztyn and Leonardo Macarini\hfill\href{https://github.com/danimalabares/sg}{github.com/danimalabares/sg}}
	\end{minipage}
\end{minipage}\vspace{.2cm}\hrule

\vspace{10pt}
{\huge Projeto}

\tableofcontents



\section{No\c c\~oes b\'asicas de mec\^anica qu\'antica}
\begin{tabular}{ c c}
Mec\^anica Cl\'asica	 & Mec\^anica qu\^antica\\
\hline
O estado de uma part\'icula est\'a determinado pela posi\c c\~ao e a velocidade. Equivalentemente, est\'a determinada pela posi\c c\~ao $\mathbf{x}$ e o momento $\mathbf{p}=m\dot\mathbf{x}$ See \href{https://mathoverflow.net/questions/19932/what-is-a-symplectic-form-intuitively}{intuition on symplectic form}.& A fun\c c\~ao de onda cont\'em a informa\c c\~ao da part\'icula em todo tempo.\\
\end{tabular}

\subsection{Cl\'asica}

Sistema f\'isico \'e uma variedade com estrutura adicional. A variedade consiste dos estados do sistema (posi\c c\~ao, momento), e a estrutura adicional s\~ao as leis de movimento. A din\^amica do sistema est\'a determinada por uma fun\c c\~ao, o Hamiltoniano. Por medio de uma forma simpl\'etica podemos obter um campo vetorial associado à $H$ 

\begin{itemize}
\item $\omega$ n\~ao degenerada implica que sempre podemos achar esse campo vetorial

\item  $\omega$ alternante (\texttt{sg.pdf} prop. 6.11) implica que $H$ \'e constante ao longo do fluxo Hamiltoniano ($X_H$ aponta na  dire\c c\~ao de energia constante).

\item F\'ormula de Cartan implica que $\omega$  \'e constante ao longo do fluxo Hamiltoniano, ie. fluxo Hamiltoniano simpl\'etico (independente do tempo?) ie. $\mathcal{L}_{X_H}\omega=0$ se e somente se $\omega$ \'e fechada.
\end{itemize}

As equa\c c\~oes de Hamilton s\~ao s\'o outra formula\c c\~ao da segunda lei do Newton. O campo vetorial Hamiltoniano \'e uma formula\c c\~ao geom\'etrica das equa\c c\~oes de Hamilton.

\begin{prop}[18.9 \texttt{das.pdf}]
	$\{f,H\}=0$ ($f$ \'e primeira integral do fluxo de $X$) se e somente se $f$ \'e constante ao longo das curvas integrais de $X_F$.
\end{prop}

\subsection{Qu\^antica}



 Equa\c c\~ao de Schr\"odinger
 \[i\hbar \frac{\partial \psi}{\partial t}=\hat{H}\psi.\]
 Distintas escolhas de Hamiltoniano $\hat{H}$ descrevem diferentes leis da natureza. Para part\'iculas n\~ao relativistas em treis dimens\~oes com energia potencial $V(x)$, o Hamiltoniano  \'e
 \[\hat{H}=-\frac{\hbar^2}{2m}\nabla^2+V(x).\]
 \'E um operador diferencial. O Laplaciano \'e
 \[\nabla^2=\frac{\partial^2}{\partial^2 x^2}+\frac{\partial^2}{\partial^2 y^2}+\frac{\partial^2}{\partial^2z^2}.\]
 Na mec\^anica cl\'asica, o Hamiltoniano est\'a relacionado com a energia do sistema, que para n\'os \'e
 \[E=\frac{1}{2m}\mathbb{p}^2+V(x)\]
 onde $\mathbb{p}=m\dot\mathbb{x}$ o momentum da part\'icula.

 Nem toda teoria f\'isica pode ser descrita usando um Hamiltoniano. (Em termos gerais, s\'o as teorias que tem conserva\c c\~ao da energia podem ser descritas com o Hamiltonano.) Importantemente, isso mesmo acontece na mec\^anica qu\^antica.

 O experimento do buraco duplo: a fun\c c\~ao de onda se comporta como part\'icula e como onda.

 \begin{defn}
 	Um \textit{\textbf{estado qu\^antico}} \'e uma fun\c c\~ao de onda $\psi(\mathbb{x},t)$ normaliz\'avel, ie.
	\[\int d^3x|\psi|^2<\infty.\]
 \end{defn}
 Esses estados qu\^anticos moram num espaço de Hilbert (tem produto Hermitiano): se a part\'icula est\'a num espaço $M$, o espaço de Hilbert relavante \'e $L^2(M)$.

\begin{defn}
	\textit{\textbf{Observ\'avel}}: s\~ao fun\c c\~oes  de $\mathbf{x}$ e $\mathbf{p}$. Por exemplo, $\mathbf{x}$ e $\mathbf{p}$ mesmas, ou o \textit{momento angular} $\mathbf{L}=\mathbf{x}\times \mathbf{p}$ ou a \textit{energia} $E=\frac{\mathbf{p}^2}{2m}+V(\mathbf{x})$.

	Os observ\'aveis s\~ao representados por \textit{\textbf{operadores}} no espaço de Hilbert. Agem numa fun\c c\~ao de onda e d\~ao outra fun\c c\~ao.
\end{defn}

\begin{remark}
	O reemplazo das matrices nos espaços de dimens\~ao infinita s\~ao os operadores diferenciais.
\end{remark}

\begin{remark}
	O resultado de qualquer medi\c c\~ao de um operador est\'a no seu espectro (conjunto de eigenvalores).
\end{remark}

O espectro do Hamiltoniano determina os poss\'iveis n\'iveis de energia do sistema qu\^antico.

Todo observ\'avel f\'isico corresponde a um operador Hermitiano (autoadjunto).

\end{document}
