\input{/Users/daniel/github/config/preamble.sty}%available at github.com/danimalabres/config
\input{/Users/daniel/github/config/thms-por.sty}%available at github.com/danimalabres/config

\usepackage[style=authortitle-terse,backend=bibtex]{biblatex}
\addbibresource{~/github/config/bibliography.bib}

\begin{document}

\begin{minipage}{\textwidth}
	\begin{minipage}{1\textwidth}
		Geometria Simplética \hfill Daniel González Casanova Azuela
		
		{\small Profs. Henrique Bursztyn and Leonardo Macarini\hfill\href{https://github.com/danimalabares/sg}{github.com/danimalabares/sg}}
	\end{minipage}
\end{minipage}\vspace{.2cm}\hrule

\vspace{10pt}
{\large Projeto final}

{\huge Pré-quantização geométrica}

\tableofcontents

\section{Introdução}

Uma quantização é um procedimento que associa um sistema mecânico quântico a um sistema clássico. Entre outros tipos de quantizações, a quantização geométrica está focada em passar de uma variedade simplética, entendida como o cenário da mecânica clássica, para um espaço de Hilbert do lado quântico. Esse programa foi desenvolvivo por B. Kostant e J-M Soriau, de acordo com os axiomas de Dirac.

Neste trabalho vamos explorar a primeira parte do processo de quantização geométrica, chamado de pré-quantização. Começaremos com uma discusão informal onde explicamos brevemente o \textit{background} físico subjacente ao processo de quantização. Despois, vamos introduzir os objetos matemáticos necessários para construir o espaço de Hilbert que podemos associar a uma variedade simplética. Encerramos com a seção mais importante, onde definimos a correspondência entre observáveis clássicos e quânticos, e demonstramos que tal correspondência satisfaz os axiomas de pré-quantização de Dirac.

Os passos finais da quantização geométrica, entre outros tópicos interessantes como o teorema de Groenewold, não serão explicados em detalhes; apenas incluiremos referências onde podem ser consultados.

\section{Motivação}


\subsection{Mecânica clássica numa variedade simplética}

Lembre da primeira aula desse curso que a origem da geometria simplética é a formulação Hamiltoniana da mecânica clássica. Isso é porque podemos pensar que um sistema mecânico é uma variedade simplética com uma função Hamiltoniana: os pontos da variedade são interpretados como \textit{estados} (posição e momento) e a \textit{evolução} do sistema é o fluxo Hamiltoniano. A evolução de um estado em particular é a curva integral que passa por esse ponto. As curvas integrais são soluções das equações de Hamilton, que em coordenadas de Darboux são
\[\dot x(t)=\frac{\partial H}{\partial y},\qquad \dot y=-\frac{\partial H}{\partial x}.\]
Lembre que em aula definimos uma integral primeira de $H$ como uma função suave $f$ tal que $\{f,H\}=0$. As integrais primeiras são constantes ao longo do fluxo Hamiltoniano. No scenário da mecânica, dizemos que as funções suaves são \textit{observáveis}, no sentido de que são quantidades que podemos medir (como posição, momento e energia). Assim, definimos para um observável $f$ a sua  \textit{evolução} como $\{f, H\}$.

\subsection{Mecânica quântica num espaço de Hilbert}

Na mecânica quântica, o estado de uma partícula está dado por uma \textit{função de onda} $\psi(x,t)$, que é um vetor unitário em um espaço de Hilbert $(\mathcal{H},\left<\cdot,\cdot\right>)$. Assim como na mecânica clássica podemos determinar o estado de uma partícula sabendo a posição e o momento em algum tempo dado, para saber o estado de uma partícula quântica no tempo $t$ basta conhecer a função de onda em algum tempo $t_0$.

%, os eigenvalores são o nível de energia quântica do sistema a as eigenfunções normalizadas são os estados quânticos correspondentes.
A equação que determina a evolução do sistema é a equação de Schrödinger:
\[i \hslash \frac{d\psi(t)}{dt}=\hat{H}\psi,\]
onde o \textit{hamiltoniano quântico} $\hat{H}$ é um operador autoadjunto agindo em $\mathcal{H}$ e $\hslash$ é uma constante.

Um \textit{observável quântico} é um operador autoadjunto $A$ agindo em $\mathcal{H}$. O valor esperado de $A$ é
\[\left<A\right>_\psi:=\left<A\psi,\psi\right>.\]
Uma motivação linda para essa definição, usando o operador de posição, pode ser consultada em \cite{hallq} sec. 3.3. Por motivos de tempo não vamos desenvolver aqui.

O valor esperado nos permite assinar um número a um observavél quântico quando aplicado a uma função de onda, do mesmo jeito em que um observável clássico pode ser avaliado num estado clássico. Assim como a evolução de um observável clássico $f$ é $\{f,H\}$,
\begin{thing3}{Proposição 1.1}[Em \cite{wang}]\leavevmode
	A \textit{evolução} de um observável quântico é
\[\frac{d}{dt}\left<A\right>_{\psi}=\frac{1}{i\hslash}\left<[A,\hat{H}]\right>_\psi.\]
\end{thing3}
\begin{proof}\leavevmode
\begin{align*}
	\frac{d}{dt}\left<A\psi(t),\psi(t)\right>&=\left<\frac{1}{i\hslash}A\hat{H}\psi,\psi\right>+\left<A\psi,\frac{1}{i\hslash}\hat{H}\psi\right>=\frac{1}{i\hslash}\left<[A,\hat{H}]\psi,\psi\right>.
\end{align*}
\end{proof}
Isso justifica que o commutador é o análogo do colchete de Poisson.

\subsection{O axioma de Dirac}

A discussão feita até agora nos conduiz à seguinte ideia do que deveria ser uma quantização: uma correspondência
\begin{align*}
	(M,\omega)&\rightsquigarrow (\mathcal{H},\left<\cdot,\cdot\right>)\\
	H&\rightsquigarrow \hat{H}\\
	f&\rightsquigarrow A\\
	\{\cdot,\cdot\}&\rightsquigarrow [\cdot,\cdot].
\end{align*}

O axioma de Dirac (chamado assim em \cite{wang}) nos diz a propriedades que essa correspondencia deve satisfacer:

\begin{thing5}{Axioma de Dirac}\leavevmode
Uma quantização associa um operador auto-adjunto $Q(f)$ num espaço de Hilbert $\mathcal{H}$ a um observável $f \in C^\infty(M)$ satisfazendo
\begin{enumerate}
\item (Linearidade) $Q(\lambda f + \mu g)=\lambda Q(f)+\mu Q(b)$.
\item (Normalização) $Q(1)=\operatorname{Id}$.
\item (Condição quântica) $Q(\{f,g\})=\frac{1}{i\hslash}[Q(f),Q(g)]$.
\item (Minimalidade) Um conjunto completo de funções que comutam respeito ao colchete de Poisson é quantizado em um conjunto completo de operadores que comutam respeito ao colchete de Lie.
\end{enumerate}
\end{thing5}
Uma correspondência satisfazendo só os pontos (1)-(3) se chama de \textit{\textbf{pré-quantização}}.

Tem muito para dizer sobre estes postulados. Em primeiro lugar, o teorema "No Go" de Groenewold mostra que uma quantização não pode existir. O leitor pode consultar  \cite{hallq}, sec. 13.4 para um enunciado preciso e a prova dele, assim como uma discussão do que significa.

Em segundo lugar, a condição (4) está relacionada com o teorema de Stone-von Neumann, e tem uma motivação tanto física quanto matemática. Embora a construção matemática que permete passar da pré-quantização que presentaremos aseguir a uma quantização completa é pertinente num curso como o nosso, não vamos presentá-la. Referimos à Lecture 13 de \cite{wang} e as seções 23.3-7 de \cite{hallq}.

\section{Fibrados lineares complexos} 

Nesta seção definimos os objetos necessários para construir a pré-quantização associada a uma variedade simplética. O espaço de Hilbert será o espaço de seções de um fibrado linear complexa; aseguir vamos explicar as condições que deven ser satisfeitas para esse fibrado existir.

Começamos lembrando que um \textit{\textbf{fibrado linear complexo}} é um fibrado vetorial cujas fibras são espaços vetoriais complexos de dimensão 1. Um fibrado linear complexo $L$ sobre uma variedade $M$ é \textit{\textbf{Hermitiano}} se em cada fibra existe um produto interno $\left<\cdot,\cdot\right>_p$ que varia suavemente, i.e. para cada seção $s$, temos que $\left<s,s\right>$ é uma função suave em $M$.

O espaço de \textit{\textbf{$k$-formas em $M$ com coeficientes em $L$}} é
	\[\Omega^{k}(M,L):=\Gamma(M,\Lambda^{k}(T^*M)\otimes L).\]
Note que como $L$ é um fibrado complexo, $(T^*M)\otimes L$ também é complexo---mas pra frente vamos usar a estrutura complexa deste fibrado. Uma \textit{\textbf{conexão}} $\nabla$ em $L$ é um mapa linear
\[\nabla:\Gamma(M,L)\to\Omega^{1}(M,L)\]
tal que para toda $f\in C^\infty(M)$ e $s \in \Gamma(M,L)$ vale a regra de Leibniz
\[\nabla(fs)=df\otimes s+f\nabla s.\]
Pegando um campo vetorial $X \in \mathfrak{X}(M)$ podemos contraer $X$ com $\nabla$ para obter a \textit{\textbf{derivada covariante}} na direção de $X$:
\[\nabla_X:\Gamma(M,L)\to \Gamma(M,L), \qquad \nabla_Xs:=i_X\nabla s.\]
A conexão $\nabla$ pode ser extendida de maneira única a um mapa linear
\[\widehat{\nabla}:\Omega^{1}(M,L)\longrightarrow \Omega^{2}(M,L)\]
tal que para qualquer $\alpha \otimes s \in\Omega^{1}(M,L)$,
\[\widehat{\nabla}(\alpha \otimes s)=d\alpha \otimes s -\alpha\wedge\nabla s.\]
De fato, isso pode ser mostrado fácilmente no caso de um fibrado vetorial geral definindo em coordenadas locais $\widehat{\nabla}$ como satisfazendo a condição anterior em combinações lineares de um marco local (ver \cite{milnorch}, lem. C.4). Dessa definição também segue que para qualquer função suave $f$,
\[\widehat{\nabla}(f(\alpha \otimes s))=df \wedge (\theta \otimes s)+f \widehat{\nabla}(\theta \otimes s).\]
Usando essa regra de Leibniz, podemos ver que $\nabla^2:=\widehat{\nabla}\circ\nabla$ é $C^\infty(M)$-linear, i.e. para qualquer função suave $f$ e seção $s \in \Gamma(M,L)$,
\[\nabla^2(fs)=\widehat{\nabla}(df \otimes s+f\nabla s)=0-df \wedge \nabla s+df \wedge \nabla s+f \nabla^2s=f\nabla^2 s.\]
Isso significa que existe uma 2-forma $\Omega\in\Omega^{2}(M)$ tal que para toda $s \in \Gamma(M,L)$,
\[\nabla^2s=\Omega s.\]
Os detalhes disso vem de álgebra multilinear: a $C^\infty(M)$-linearidade nos diz que $\nabla^2s$ é um $(0,2)$-campo tensorial valuado em  $L$, e existe uma correspondência entre esses campos tensoriais e 2-formas em $M$ (ver \cite{tu-diff} exem. 21.9 e prop. 21.11). Essa forma se chama \textit{\textbf{curvatura}} de $\nabla$. Um resultado que não provaremos (ver  \cite{verbi}, claim 2.10) é que
\[\Omega(X,Y)=[\nabla_X,\nabla_Y]-\nabla_{[X,Y]}.\]
Usando isso podemos mostrar que $\Omega$ é fechada seguindo a prova do teo. 11.1 \cite{tu-diff}). Lembre que para qualquer fibrado vetorial, em coordenadas locais sempre podemos achar um marco de seções, i.e. uma coleção de seções linearmente independentes que geram o espaço de seções. No nosso caso, como estamos trabalhando com um fibrado linear, o marco consiste de uma seção só. Assim, em qualqer vizinhança coordenada $U$ sabemos que existe uma seção $e \in \Gamma(U,L)$ que não se anula e
\[\nabla e=\theta \otimes e\]
para alguma 1-forma $\theta \in \Omega^{1}(U,L)$.

Agora vamos calcular a curvatura localmente. Pegando $X,Y \in \mathfrak{X}(U)$,
\begin{align*}
\nabla_X\nabla_Y e&=\nabla_X(\theta(Y) e)\\
&=X\theta(Y) e+\theta(Y)\nabla_Xe\\
&=X\theta(Y) e+\theta(Y)\theta(X)e.
\end{align*}
E isso vale trocando os lugares de $X$ e $Y$. E como também $\nabla_{[X,Y]}e=\theta[X,Y]e$, obtemos
\begin{align*}
\Omega(X,Y)e&=(\nabla_X\nabla_Y-\nabla_Y\nabla_X-\nabla_{[X,Y]})e\\
&=(X\theta(Y)-Y\theta(X)-\theta[X,Y])e\\
&=d\theta(X,Y)e.
\end{align*}
Isso mostra que localmente,
\[\Omega=d\theta,\]
e isso implica que $\Omega$ é fechada. Também podemos usar essa expresão para mostrar que $\Omega$ é imaginaria. O argumento é bastante simples usando um produto hermitiano em $L$ \textit{\textbf{compatível}} com $\nabla$, ou seja, satisfazendo
\[d\left<s,s'\right>=\left<\nabla s,s'\right>+\left<s,\nabla s'\right>\]
para quaisquer $s,s'$  seções de $L$.

 Suponha que nosso "marco local" $e$ é unitário. Então
\begin{align*}
0&=d\left<e,e\right>=\left<\nabla e,e\right>+\left<e,\nabla e\right>\\
&=\left<\theta \otimes e,e\right>+\left<e, \theta \otimes e\right>\\
&=\theta + \overline{\theta}.
\end{align*}
Isso significa que $\theta$ é uma forma puramente imaginária, e de fato implica que $d\theta$ também já que a derivada exterior complexa está definida como a extensão $\mathbb{C}$-linear da derivada exterior real. Os detalhes da construção da álgebra exterior complexa podem ser consultados, por exemplo, em \cite{leec} cap. 7. Note que a nossa variedade $M$ não é a priori complexa; apenas o fibrado $T^*M \otimes L$ é complexo.

Finalmente podemos definir \textit{\textbf{primeira classe de Chern}} de $L$ como
\[c_1(L):=\left[ \frac{1}{2\pi i}\Omega \right] \in H^{2}_{\operatorname{dR}}(M,\mathbb{R}).\]
O teorema de Chern-Weil mostra que $c_1(L)$ é independente da escolha de conexão é métrica hermitiana em $L$, i.e. é um invariante topológico (ver teo. 23. 3 \cite{tu-diff}, teo. 7.12 \cite{leec}).

Usando a aplicação de de Rham de $H_{\operatorname{dR}}^2(M,\mathbb{R})$ a $H^{2}_{\operatorname{Ch}}(M,\mathbb{R})$ é possível mostrar que $c_1(L)$ é uma classe de cohomologia \textit{\textbf{integral}}, i.e. $c_1(L)\in H^{2}(M,\mathbb{Z})$. Ao contrário,
\begin{thing4}{Teorema 2.6}[Weil, em \cite{wang}]\leavevmode
	Seja $M$ uma variedade suave e $\omega$ uma forma real fechada cuja classe de cohomologia $[c]$  é integral. Então existe um único fibrado linear Hermitiano $L$ sobre $M$ com conexão unitária $\nabla$ tal que $c_1(L)=[c]$.
\end{thing4}
Esse é o resultado chave que precisamos para a seguinte seção. Contudo, parece que a prova dele não é tão simples: \cite{wang} da apenas um esbozo da demostração usando argumentos parecidos aos da implicação contrária, enquanto \cite{hallq} indica o leitor para \cite{wood}.

\section{Pré-quantização geométrica}

Dizemos que uma variedade simplética $(M,\omega)$ é pré-quantizável se
\[\left[ \frac{\omega}{2\pi} \right] \in H^{2}(M,\mathbb{Z}).\]
Um fibrado linear Hermitiano $(L,h,\nabla)$ sobre $(M,\omega)$ com $\Omega=\frac{\omega}{i\hslash}$ se chama \textit{\textbf{fibrado linear pré-quântico}}.

Dada $(M,\omega)$ uma variedade pre-quantizável e $(L,h,\nabla)$ um fibrado linear pré-quântico sobre $M$, o espaço de Hilbert onde vamos trabalhar é $\mathcal{H}:=L^2(M,L)$ com o produto interior
\[\left<s_1,s_2\right>:=\frac{1}{(2\pi \hslash)^n}\int_{M}h(s_1,s_2)\frac{\omega^n}{n!}.\]
Formalmente, $L^2(M,L)$ é o espaço de classes de equivalência de seções quadrado-integráveis de $L$ (i.e. $\|s\|=\sqrt{\left<s,s\right>}<\infty$) identificando duas seções que são iguais en quase todo ponto respeito à medida de Liouville. A construção desse espaço tem sutilezas que não vamos especificar aqui (ver \cite{hallq} sec. 7.3).

Finalmente vamos definir o operador que quântiza funções suaves. Dada $f \in C^\infty(M,\mathbb{R})$, denotamos por $m_f$ o operador que multiplica uma seção em $\mathcal{H}$ por $f$. O operador \textit{\textbf{pré-quântico}}  é
\[Q(f)=-i\hslash \nabla_{X_f}+m_f.\]

\begin{thing4}{Proposição 3.3}[Em \cite{wang}]\leavevmode
$Q(a)$ é autoadjunto em $\mathcal{H}$.	
\end{thing4}

\begin{proof}\leavevmode
Como $f$ é uma função suave com valores em $\mathbb{R}$, é suficente mostrar a proposição para o operador $i\nabla_{X_f}$.
\begin{align*}
\left<i\nabla_{X_f}s_1,s_2\right>&=\int_{M}h(i \nabla_{X_f}s_1,s_2)\frac{\omega^n}{n!}\\
&=i \int_{M}X_f(h(s_1,s_2))\frac{\omega^n}{n!}+\int_{M}h(s_1,i\nabla_{X_f}s_2))\frac{\omega^n}{n!}\\
&=i \int_{M}X_f(h(s_1,s_2))\frac{\omega^n}{n!}+\left<s_1,i\nabla_{X_f}s_2\right>\\
&=i \int_{M}\{f,h(s_1,s_2)\}\frac{\omega^n}{n!}+\left<s_1,i \nabla_{X_f}s_2\right>.
\end{align*}
Para concluir precisamos mostrar que a integral na última igualdade se anula. Supondo que $M$  não tem bordo, poderemos usar o teorema de Stokes se mostramos que
\begin{claim}\leavevmode
	Para $g,h \in C^\infty(M)$, a forma $\{g,h\}\omega^n$ é exata.
\end{claim}
\begin{proof}[Prova da afirmação]\leavevmode
O seguinte argumento vem de \href{https://math.stackexchange.com/questions/3532438/integral-of-poisson-bracket-vanishes}{StackExchange}. Primeiro note que
\begin{equation}\label{eq:1}
\{g,h\}\omega^n = -X_g(h)\omega^n.
\end{equation}
Agora vamos expressar isso como derivada de Lie. Lembre que para $Y_1,\ldots,Y_{2n}\in \mathfrak{X}(M)$,
\begin{align*}
	\left(\mathcal{L}_{X_g}h\omega^n\right)(Y_1,\ldots,Y_{2n})&=X_g\left( h \omega^n\left( Y_1,\ldots,Y_{2n} \right)  \right) -h \omega^n\left( [X_f,Y_1],Y_2,\ldots,Y_{2n} \right) \\
								  &  \qquad -\ldots-h \omega^n\left( Y_1,\ldots,Y_{2n-1},[X,Y_{2n}] \right).
\end{align*}
Aplicando a regra de Leibiniz na primeira parcela obtemos
\[X_g\left( h \omega^n\left( Y_1,\ldots,Y_{2n} \right)  \right)=(X_gh)\omega^n(Y_1,\ldots,Y_{2n})+hX_f\left( \omega^n(Y_1,\ldots,Y_{2n}) \right), \]
e desse jeito podemos expresar simplesmente
\[\mathcal{L}_{X_g}h\omega^n=(X_gh)\omega^n+\cancelto{0}{h\mathcal{L}_{X_g}\omega^n}.\]
Voltando à \cref{eq:1},
\begin{align*}
\{g,h\}\omega^n&=-\mathcal{L}_{X_g}h \omega^n=-d(i_{X_f}h\omega^n)-\cancelto{0}{i_{X_g}d(h \omega^n)}
\end{align*}
\end{proof}
Voltando a nossa prova inicial, obtemos o resultado simplesmente porque
\[\int_{M}\{f,h(s_1,s_2)\}\frac{\omega^n}{n!}=-\int_{\partial M}i_{X_f}h(s_1,s_2)\frac{\omega^n}{n!}=0.\]
\end{proof}

\begin{thing5}{Teorema 3.4}[Kostant-Souriau, em \cite{wang}]\leavevmode
A assignação $f \rightsquigarrow Q(f)$ é uma pré-quantização. Em particular, para quaisquer $f,g \in \mathcal{C}^\infty(M)$,
\begin{equation}\label{eq:prequant}\frac{1}{i\hslash}[Q(f),Q(g)]=Q(\{f,g\}).
\end{equation}
\end{thing5}

\begin{proof}\leavevmode
As propriedades (1) e (2), de linearidade e normalização, na definição de pré-quantização são obvias. Só resta mostrar  \cref{eq:prequant}.
\begin{equation}\label{eq:conta1}
\begin{aligned}
	\frac{1}{i \hslash}[Q(f),Q(g)]&=\frac{1}{i \hslash}\Big(Q(f)Q(g)-Q(g)Q(f)\Big)\\
	&=\frac{1}{i \hslash}\Big(-i \hslash \nabla_{X_f}+ m_f)(-i \hslash \nabla_{X_g}+m_g)\\& \qquad -(-i \hslash \nabla_{X_g}+m_g)(-i \hslash \nabla_{X_f}+m_f) \Big)\\
	&=\frac{1}{i \hslash} \Big( (-i \hslash)^2(\nabla_{X_f}\nabla_{X_g}-\nabla_{X_g}\nabla_{X_f})\\& \qquad -i\hslash (\nabla_{X_f}m_g+m_f \nabla_{X_g}-\nabla_{X_g}m_f-m_g\nabla_{X_f}) \Big).
	\end{aligned}	\end{equation}
A última parcela da última igualdade se simplifica notando que em qualquer seção $s$,
\[\left(\nabla_{X_f}m_g\right)(s)=\nabla_{X_f}(gs)=dg(X_f)s+g\nabla_{X_f}s;\]
daí segundo termo se cancela com $m_{g}\nabla_{X_f}$. A \cref{eq:conta1} é
\begin{align*}
&=i \hslash [\nabla_{X_f},\nabla_{X_{g}}]-\Big(dg(X_f)-df(X_g) \Big)\\
&=i \hslash [\nabla_{X_f},\nabla_{X_g}]+2\{f,g\}.
\end{align*}
Como $\Omega(X_f,X_g)=[\nabla_{X_f},\nabla_{X_g}]-\nabla_{[X_{f},X_g]}$, obtemos
\begin{align*}[\nabla_{X_f},\nabla_{X_g}]&=\Omega(X_f,X_g)+\nabla_{[X_f,X_g]}\\
	&=\frac{1}{i \hslash}\omega(X_f,X_g)+\nabla_{[X_f,X_g]}\\
	&=-\frac{1}{i \hslash}\{f,g\}+\nabla_{X_{f,g}}.
\end{align*}
\end{proof}

\iffalse
\section{Polarização}

Começamos com uma motivação. A ideia é que o espaço de Hilbert vai consistir em seções de um fibrado linear prequântico que sejam covariantemente constantes nas direções de certas funções. A escolha das funções vai determinar a polarização; por exemplo, pegando $\bar{z}$ em $\mathbb{R}^{2n}$ obtemos o espaço de Segal-Bergman.

Pegue funções suaves $\alpha_1,\ldots,\alpha_n$ para constuir uma polarização. Defina $P_z\subset T_zM$  como o espaço em que $\alpha_i$ são constantes. Pedindo também que $\{a_j,a_k\}=0$, $P_z$ terminha sendo o espaço generado pelos campos hamiltonianos $X_{\alpha_i}$.

Para a proxima: talvez definir polarização e as definições das polarizações reais e complexas. Eu diria: dar uma passadinha as reais e dai passar nas complexas. Será que a gente consegue falar das Kähler?
\fi

\section*{Referências}
\printbibliography[heading=none]
\end{document}
