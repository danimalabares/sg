\input{/Users/daniel/github/config/preamble.sty}%available at github.com/danimalabres/config
\input{/Users/daniel/github/config/thms-por.sty}%available at github.com/danimalabres/config

\usepackage[style=authortitle-terse,backend=bibtex]{biblatex}
\addbibresource{bibliography.bib}

\begin{document}

\begin{minipage}{\textwidth}
	\begin{minipage}{1\textwidth}
		Geometria Simpl\'etica \hfill Daniel González Casanova Azuela
		
		{\small Profs. Henrique Bursztyn and Leonardo Macarini\hfill\href{https://github.com/danimalabares/sg}{github.com/danimalabares/sg}}
	\end{minipage}
\end{minipage}\vspace{.2cm}\hrule

\vspace{10pt}
{\huge Projeto}

\tableofcontents

\section{Motivação: operadores de posição e momento}

Aqui sigo \cite{clas} Sec. 1.1 e \cite{qm} Sec. 1.1. Os princípios fundamentais da mecânica clássica foram estabelecidos nos séculos XVI e XVII por Galileo e Newton. No famoso texto \textit{Prinicipia Mathematica} do Newton, publicado em 1686, ele estabeleceu as tres leis de movimento e a lei da gravitação.

A segunda lei de movimento é $F=ma$. A idea  é que tendo uma coleção de partículas sujetas a uma  coleção de forças agindo em elas, essa equação nos permite descrever as velocidades das partículas no futuro.

Mais detalhadamente, o estado de uma partícula está determinado pela posição $\mathbf{x}$ e a velocidade $\mathbf{v}=\dot{\mathbf{x}}$. Conhecendo essas informações em algum tempo $t_0$, podemos usar a sgunda lei de Newton, reescrita como $\mathbf{F}=m\ddot{\mathbf{x}}$ para determinar $\mathbf{x}(t)$ e $\mathbf{v}(t)$ para cualquer tempo $t$. Note que não basta saber a posição: é necessário saber tanto a posição $\mathbf{x}(t_0)$ quanto a velocidade $\dot{\mathbf{x}}(t_0)$.

Como falamos na primeira aula desse curso, existem distintas descripções da mecânica clássica além da Newtoniana; notávelmente e mecânica Lagrangiana e a Hamiltoniana. Na seguinte seção vamos descrever rápidamente a relação entre o formalismo Hamiltoniano e a geometria simplética.

Na mecânica quântica, o estado de uma partícula está determiado por uma função de onda, que é uma função $\psi(\mathbf{x},t)$ com valores em $\mathbb{C}$. Em constraste com a mecânica clássica, para saber o estado da partícula em qualquer tempo $t$,  basta conhecer a função de onda em algum tempo $t_0$. Note que embora pareça um scenário mais simples, a sustitução de um vetor num espaço de dimensão finita por uma função é um passo não trivial.

É natural esperar que a informação da velocidade da partícula esteja de alguma maneira codificada na função de onda. Uma interpretação da função de onda é que a função de onda nos diz a \textit{probabilidade} de que a partícula esteja numa posição dada. Mas precisamente, a probabilidade de que a partícula estja em um volumen $E$ perto de  $\mathbf{x}$ é ${\color{3}\int_{E}}|\psi(\mathbf{x},t)|^2dV$. Dizemos que $|\psi(x)|^2$ é a \textit{\textbf{densidade de probabilidade}} para a posição da partícula.

É por isso que precisamos trabalhar com funções  de onda normalizadas, i.e. tais que
\[\int d^3x |\psi(\mathbf{x},t)|^2=1\]
(a partícula tem que estar em algum lugar!)

\subsection{O operador de posição}

(Essa seção é \cite{hallq}, 3.3)

Considere o caso muito simples de uma particula movendo-se na reta $\mathbb{R}$. A função de onda dessa particular é $\psi:\mathbb{R}\to \mathbb{C}$ tal que $\int_{\mathbb{R}}|\psi(x)|^2=1$. Para quem sabe um pouco de probabilidade, o valor esperado associado a essa densidade de probabilidade é
\[E(x)=\int_{\mathbb{R}}x |\psi(x)|^2dx.\]
vamos chamar isso de \textit{\textbf{operador de posição}}.

Uma important ideia na mecânica quântica é levar o valor esperado das quantidades que medimos (como posição momentum, energia, etc…) em termos de operadores num espaço de Hilbert, neste caso $L^2(\mathbb{R})$. No caso da posição, definimos o \textit{\textbf{operador de posição}} $X$ como
 \[(X\psi)(x)=x\psi(x)\]
 de modo que 
 \[E(x)=\left<\psi,X\psi\right> =\int_{\mathbb{R}}\overline{\psi}x\psi dx\]
 como é usual a definição de norma em espaços de funções complexas.

\subsection{O operador de momento}

(\cite{clas}, introdução.) No mesmo caso de uma partícula, a segunda lei de Newton pode ser reformulada como  $\mathbf{F}=\dot{\mathbf{p}}$ onde $\mathbf{p}:=m\dot{\mathbf{x}}$ é o \textit{\textbf{momento}}.

Achar um operador de momento é o que responde a pergunta de cómo uma função de onda contém a informação da velocidade de uma partícula. Embora não podemos aprofundar nisso, a explicação é que o momento está codificado nas osilações da função de onda.

Por enquanto simplesmente vamos definir \textit{\textbf{operador de momento}} como
\[P=-i\hslash\frac{d}{dx}\]
 onde por enquanto $\hslash$ é só uma constante. Esse operador permite calcular o valor esperado do momento mediante
 \[E(p)=\left<\psi,P\psi\right>.\]

\begin{remark}\leavevmode
	Note que tanto o operador momento quanto o operador de posição não estão definidos em tudo o espaço $L^2(\mathbb{R})$. (Já que a função $P\psi=x\psi$ pode não estar em $L^2(\mathbb{R})$, ou a funçao $\psi$ pode não ser diferenciável, ou a derivada dela não estar em $L^2(\mathbb{R})$.
\end{remark}

\subsection{Relação de comutação canônica}

\begin{thing4}{Proposição 3.8}[\cite{hallq}]\leavevmode
Os operadores de posição  $X$ e momento $P$ não comutam, mas satisfazem a relação
\[XP-PX=i\hslash I,\]que chamamos de \textit{\textbf{relação de comutação canônica}}.
\end{thing4}

\begin{proof}\leavevmode
	\begin{align*}
		PX\psi&=-i\hslash\frac{d}{dx}(x\psi(x))\\
		&=-i\hslash\psi(x)-i\hslash x\frac{d\psi}{dx}\\
		&=-i\hslash\psi(x)+XP\psi.
	\end{align*}
\end{proof}
Essa relação é muito importante, pois ella nos da uma intuição do que esperamos no equivalente ao colchete de Poisson no mundo quântico. Por enquanto só lembre que no caso do colchete de Poisson na variedade $\mathbb{R}^{2}=\{(x,p)\}$ sabemos que $\{x,p\} =1$.

\subsection{Os operadores de posição e momento são simétricos}

Lembre que um operador linear $A:\mathbf{H}\to\mathbf{H}$ é \textit{\textbf{limitado}} se existe uma constante $C$ tal que $\|A\psi\|\leq C\|\psi\|$ para todo $\psi\in \mathbf{H}$. Para cada operador limitado $A$ existe um único operador $A^*$, chamado o \textit{\textbf{adjunto}} de $A$, tal que
\[\left<\phi,A\psi\right>=\left<A^*\phi,\psi\right>.\]
Porém, o caso dos operadores não limitados é um pouco mais delicado. Isso vai ser importante para nos porque os operadores quânticos não são limitados.

\iffalse
\begin{thing4}{Definição 3.3}[\cite{hallq}]\leavevmode
	Seja $A$ um operador não limitado num  espaço de hilbert $\mathbf{H}$. O \textit{\textbf{adjunto}} $A^*$ de $A$ se define como segue. Um vetor  $\phi \in \mathbf{H}$ pertence a $\operatorname{Dom}(A^*)$ se o j
\end{thing4}\fi

\begin{thing4}{Definição 3.3}[\cite{hallq}]\leavevmode
	Um operador $A$ num espaço de Hilbert $\mathbf{H}$ é \textit{\textbf{simétrico}} se
	\[\left<\phi,A\psi\right> =\left<A\phi,\psi\right> \]
	para todos $\phi,\psi \in \mathbf{H}$. O operador $A$  é \textit{\textbf{autoadjunto}} se  $\operatorname{Dom}(A^*)=\operatorname{Dom}(A)$ e $A^*\phi=A\phi$ para todo $\phi \in \operatorname{Dom}(A)$.
\end{thing4}

\begin{thing7}{Proposição 3.9}[\cite{hallq}]\leavevmode
	Para funções suficientemente boas $\phi$ e $\psi$ em $L^2(\mathbb{R})$, temos que
	\[\left<\phi,X\psi\right> =\left<X\phi,\psi\right>,\qquad \left<\psi,P\psi\right>=\left<P\phi,\psi\right>.\]
\end{thing7}

\section{Os axiomas da mecânica quântica*}

Os seguintes "axiomas" não são axiomas matemáticos. São principios fundamentais para a mecânica quântica que nos ajudarám a fixar a discusão feita até agora, dexiando tudo pronto para começar a discutir o conceito de quantização no espaço euclidiano e depois em variedades simpléticas.

\begin{thing1}{Axioma 1}\leavevmode
	O estado de um sistema (quântico) está representado por um vetor unitário $\psi$ em certo espaço de Hilbert $\mathbf{H}$. Se $\psi_1$ e $\psi_2$ são dois vetores unitários em $\mathbf{H}$ com $\psi_2=c\psi_1$ para alguma constante $c \in \mathbb{C}$, então $\psi_1$ e $\psi_2$ representam o mesmo estado físico.
\end{thing1}

Vamos motivar a segunda frase mais pra frente.

\begin{thing1}{Axioma 2}\leavevmode
	A cada função real-valuada $f$ num espaço fase clássico tem associado um operador autoajunto $\hat{f}$ no $\mathbf{H}$.
\end{thing1}

\begin{remark}\leavevmode
	$\hat{f}$ típicamente não é limitado.
\end{remark}

\begin{thing1}{Axioma 3}\leavevmode
	Se um sistema quântico está num estado dado por um vetor unitário  $\psi \in \mathbf{H}$, a distribuição de probabilidade da medição de algum observável $f$ satisfaz
	\[E(f^m)=\left<\psi,(\hat{f})^m\psi\right>\]
	Em particular, o valor esperado de uma medição de $f$ está dada por
	\[\left<\psi,\hat{f}\psi\right>.\]
\end{thing1}

A segunda frase no Axioma 1 se justifica porque para qualquer operador $A$ e vetores unitários $\psi_2=c\psi_1$ com $|c|=1$,
\[\left<\psi_2,A\psi_2\right>=\left<c\psi_1,Ac\psi_1\right>=|c|^2\left<\psi_1,A\psi_1\right>=\left<\psi_1,A\psi_1\right>.\]

\begin{thing1}{Axioma 4}\leavevmode
	Relacionado com incertidumbre.
\end{thing1}

\begin{thing1}{Axioma 5}\leavevmode
	A evolução temporal de uma função de onda $\psi$ em um sistema quântico está dada pela equação de Schrödinger
	\[\frac{d\psi}{dt}=\frac{1}{i\hslash}\hat{H}\psi.\]
	Onde $\hat{H}$ é o operador que corresponde ao Hamiltoniano $H$ por meio do Axioma 2.
\end{thing1}

\begin{thing3}{Proposição 3.14}\leavevmode
	Seja $\psi(t)$ uma solução à equação de Schödinger e $A$ é um operador autoadjunto em $\mathbf{H}$. Supondo as condições necessárias no dominio de $\psi$,
	\[\frac{d}{dt}\left<A\right>_{\psi(t)}=\left<\frac{1}{i \hslash}[A,\hat{H}]\right>_{\psi(t)},\]
	onde $\left<A\right>_{\psi}:=\left<\psi,A\psi\right>$ e $[\cdot,\cdot]$ é o \textit{\textbf{comutador}} definido como  $[A,B]=AB-BA$.
\end{thing3}

Em particular, se os operadores quânticos comuataram, os valores esperados seriam 0. Essa equação é para ser comparada com a forma em que uma função $f$ muda ao longo do fluxo Hamiltoniano: $\frac{df}{dt}=\{f,H\}$. (Lembre a definição das primeiras integrais de $H$, eram funções que não variavam ao longo do fluxo Hamiltoniano, satisfazendo $\{f,H\}=0$.)

\section{O que esperamos de uma quantização}

Essa definição é de \cite{k3quant}.
\begin{defn}\leavevmode
	Uma \textit{\textbf{quantização completa}} de $M$ é um mapa
	\[\mathcal{F}:f\longmapsto \hat{f}\]
	levando observáveis clássicos $f $, i.e. funções suaves em $T^*M$ a operadores autoadjuntos $\hat{f}$ num espaço de Hilbert $\mathcal{H}$ satisfazendo:
	\begin{enumerate}
		\item $\mathcal{F}$ é linear:
			\[\widehat{f+g}=\hat{f}+\hat{g},\qquad f,g\in\mathcal{C}^\infty(T^*M)\]
			\[\widehat{\lambda f}=\lambda\hat{f},\qquad \lambda\in\mathbb{R}.\]
	
		\item $ \mathcal{F}$ é um morfismo de álgebras de Lie salvo por uma constante:
			\[\widehat{\{f,g\}}=\frac{1}{\hslash}\left[\hat{f},\hat{g}\right]\]
\item A função constante $1$ corresponde com a identidade:
	\[\hat{1}=\operatorname{Id}\]

	\item As coordenadas $\hat{q}^i$ e $\hat{p}^i$ agem irreducívelmente em $\mathcal{H}=L^2(M)$.
	\end{enumerate}
\end{defn}

\section{Quantização de Weyl e o teorema de Groenewold}

No sentido do Axioma 2, chamamos o operador $\hat{f}$ a \textit{\textbf{quantização}} de $f$. Já vimos as quantizações dos observáveis de posição, momento e energia (Hamiltoniano), então a pergunta é se é possível construir um esquema de quantização que funcione para qualquer observável de um sistema quântico. Nesta seção vamos ver rapidamente as dificultades que isso traz, levando ao scénario onde vamos construir a quantização geometrica.

Embora existem muitos otros esquemas de quantização em sistemas com um grau de liberdade, vamos presentar somente o esquema de Weyl.

\begin{defn}\leavevmode
Definimos a \textit{\textbf{quantização de Weyl (simplificada)}} como uma correspondência entre polinomios em $\mathbb{R}^{2}$ y operadores em $C^\infty_c(\mathbb{R})$ mediante a formula
\[Q(x^jp^k)=\frac{1}{(j+k)!}\sum_{\sigma\in S_{j+k}}\sigma(X,X,\ldots,X,P,P,\ldots,P),\]
onde para quaisquer operadores $A_1,A_2,\ldots,A_n$ e $\sigma \in S_n$, definimos
\[\sigma(A_1,A_2,\ldots,A_n)=A_{\sigma(1)}A_{\sigma(2)}\ldots A_{\sigma(n)}.\]
\end{defn}

Esa correspondência pode ser generalizada para polinomios em $\mathbb{R}^{2n}$ e operadores sobre $C^\infty_c(\mathbb{R}^n)$. Isso da a seguinte propriedade:

\begin{thing4}{Proposição 13.11}[\cite{hallq}]\leavevmode
Seja $f$ um polinomio em $\mathbf{x}$ e $\mathbf{p}$ de grau menor o igual que 2 e $g$ um polinômio arbitrário em $\mathbf{x}$ e $\mathbf{p}$. Então
\[\frac{1}{i \hslash}[Q(f),Q(g)]=Q(\{f,g\}),\]
onde $\{f,g\}$ é o colchete de Poisson.
\end{thing4}

Embora parece prometedor,

\begin{thing5}{Teorema "No Go" de Groenewold}[13.13 \cite{hallq}]\leavevmode
Seja $\mathcal{D}(\mathbb{R}^{n})$ o espaço de operadores diferenciais em $\mathbb{R}^n$ com coeficientes polinomiais. Não existe uma aplicação linear $Q:\mathcal{P}_{\leq 4}\to \mathcal{D}(\mathbb{R}^n)$ com as seguintes propriedades:
\begin{enumerate}
\item $Q(1)=\operatorname{Id}$.
\item $Q(x_j)=X_j$ e $Q(p_j)=P_j$.
\item Para quaisquer $f$ e $g$ em $\mathcal{P}_{\leq 3}$, 
	\[\frac{1}{i\hslash}[Q(f),Q(g)]=Q(\{f,g\}).\]
\end{enumerate}
\end{thing5}

A pergunta de se existe uma quantização não é fácil de responder. Vamos ver que um método para consertar isso é trocar o espaço de Hilbert $L^2(\mathbb{R}^n)$ por $L^2(\mathbb{R}^{2n})$. Porém, esse espaço é "muito grande" e vamos precisar de fazer ele mais pequeno para as coisas dar certas.

\section{Quantização geométrica no espaço euclideano}

Esta seção está basada no capítulo 22 de \cite{hallq}. O objetivo é presentar o programa de quantização geométrica na variedade simplética $\mathbb{R}^{2n}$ com a forma canónica $\omega=\sum dq_j\wedge dp_j$.

\subsection{Prequantização no espaço euclideano}

Vamos seguir \cite{hallq}, capítulo 22, \textit{Geometric quantization on Euclidean space}.

Os campos vetoriais Hamiltonianos, pensados como operadores diferenciais, satisfazem as relações de commutatividade desejadas: basta definir $Q(f)=i\hslash X_f$ para obter
\[\frac{1}{i\hslash}[Q(f),Q(g)]=\frac{1}{i\hslash}[i\hslash X_f,i\hslash X_g]=(i\hslash) X_{\{f,g\}}=Q(\{f,g\}).\]
Porém, esse mapa não satisfaz $Q(\mathbf{1})=\operatorname{Id}$ porque o Hamiltoniano da função 1 é zero. Pode tentar consertar isso definindo $Q(f)=i\hslash X_f+f$, mas desse jeito
\begin{align*}
	\frac{1}{i\hslash}[Q(f),Q(g)]&=\frac{1}{i\hslash}[i\hslash X_f+f,i\hslash X_g+g]\\
	&=(i\hslash)\Big(\ldots\Big) \neq Q(\{f,g\})
\end{align*}
Mas isso tem solução. Considere um \textit{\textbf{potencial simplético}}, i.e. uma forma  $\theta$ tal que $d\theta=\omega$ a defina
\begin{equation}\label{eq:prequant}
Q(f)=i\hslash \Big(X_f-\frac{i}{\hslash}\theta(X_f)\Big)+f.\end{equation}
Vai resultar que esse operador é pelo menos simétrico, e ainda,
\begin{thing5}{Proposição 22.1}[\cite{hallq}]\label{prop:22.1}\leavevmode
Para quaisquer $f,g\in C^\infty(\mathbb{R}^{2n})$,
\[\frac{1}{i\hslash}[Q(f),Q(g)]=Q(\{f,g\})\]
\end{thing5}
Vamos explicar um pouquinho o que significa a \cref{eq:prequant}. Lembre 
\begin{thing3}{Definição 10.1}[\cite{tu-diff}]\leavevmode
	Seja $E\to M$ um fibrado vetorial suave sobre uma variedade $M$. Uma \textit{\textbf{conexão}} em $E$ é um mapa
	\[\nabla:\mathfrak{X}(M)\times\Gamma(E)\longrightarrow\Gamma(E),\]
onde $\Gamma(E)$ são as seções de $E$, satisfazendo para todo $X\in \mathfrak{X}(M)$ e $s\in\Gamma(E)$ que
\begin{enumerate}[label=(\roman*)]
\item $\nabla_Xs$ é $C^\infty(M)$-linear em $X$ e $\mathbb{R}$-linear em $s$,
\item (regra de Leibniz) se $f\in C^\infty(M)$,
	\[\nabla_X(fs)=(Xf)s+f\nabla_Xs=(df(X))s+f\nabla_Xs.\]
\end{enumerate}
\end{thing3}
Formalmente, essa construção aplicada no fibrado tangente $TM$ permite definir uma \textit{\textbf{derivada covariante}}, denotada também por $\nabla_X$, que extende a noção de derivada respeito a um campo vetorial para tensores de qualquer grau na variedade. A definição para as funções suaves é simplesmente $\nabla_Xf=Xf$; e se cumple a regra de Leibniz (\cite{tu-diff}, teo. 22.8).

Agora vamos definir uma derivada covariante. Pegue um potencial simplético $\theta$ (= uma 1-forma cuja derivada exterior é a forma simplética) e defina a \textit{\textbf{derivada covariante associada a $\theta$}} como
\[\nabla_X=X-\frac{i}{\hslash}\theta(X):C^\infty(\mathbb{R}^{2n})\longrightarrow C^\infty(\mathbb{R}^{2n})\]
onde a função $\theta(X)\in C^\infty(\mathbb{R}^{2n})$ age por simples multiplicação ponto a ponto. 

Aqui é um bom momento para olhar de novo a nossa definição de $Q(f)$ (\cref{eq:prequant}). A prova da relação de commutatividade, prop. \hyperref[prop:22.1]{22.1}, é muito fácil de escrever em termos da curvatura dessa conexão. A seguinte proposição descreve essa curvatura.

\begin{thing4}{Proposição 22.3}[\cite{hallq}]\leavevmode
	Seja $\theta$ um potencial simplético e $\nabla_X$ a derivada covariante associada. Para quaisquer campos vetoriais $X,Y$ em  $\mathbb{R}^{2n}$,
	\[[\nabla_X,\nabla_Y]-\nabla_{[X,Y]}=-\frac{1}{\hslash}\omega(X,Y).\]
\end{thing4}

\begin{proof}\leavevmode
	Note que o colchete $[\cdot,\cdot]$ é o comutador de operadores, onde as funções suaves se consideram operadores que multiplicam ponto a ponto. A regra de Leibniz para derivada covariante diz que
	\[\nabla_X(fg)=gXf+fXg\]
	de modo que o operador $[\nabla_X,f]$ aplicado em uma função $g\in C^\infty(\mathbb{R}^{2n})$ da
\[[\nabla_X,f]g=\nabla_X(fg)-f \nabla_Xg=gXf+fXg-fXg=gXf\]
ou seja, como operadores temos
\[[\nabla_X,f]=Xf.\]
Agora vamos calcular o colchete.
\begin{align*}
	[\nabla_X,\nabla_Y]&=[X-\frac{i}{\hslash}\theta(X),Y-\frac{i}{\hslash}\theta(Y)]\\
			   &=[X,Y]-\frac{i}{\hslash}[X,\theta(Y)]-\frac{i}{\hslash}[\theta(Y),Y]+\frac{1}{\hslash}\cancelto{0}{[\theta(X),\theta(Y)]}\\
			   &=[X,Y]-\frac{i}{\hslash}\Big(X(\theta(Y))-Y(\theta(X))\Big).
\end{align*}
Substraindo o termo
\begin{align*}
	\nabla_{[X,Y]}=[X,Y]-\frac{i}{\hslash}\theta([X,Y])
\end{align*}
obtemos
\[[\nabla_X,\nabla_Y]-\nabla_{[X,Y]}=-\frac{i}{\hslash}\Big(X(\theta(Y))-Y(\theta(X))-\theta([X,Y])\Big),\]
qué exatamente a fórmula ``livre de coordenadas" da derivada exterior $d\theta=\omega$.
\end{proof}

Agora podemos provar a relação de commutatividade, prop. \hyperref[prop:22.1]{22.1}.

\begin{proof}[Prova da prop \cref{prop:22.3}]\leavevmode
\begin{align*}
	\frac{1}{i\hslash}[Q(f),Q(g)]&=\frac{1}{i\hslash}\left[  i\hslash \Big(X_f-\frac{i}{\hslash}\theta(X_f)\Big)+f,i\hslash \Big(X_g-\frac{i}{\hslash}\theta(X_g)\Big)+g\right]\\
&=\left[i\hslash \nabla_{X_f}+f,i\hslash\nabla_{X_g}+g \right] \\
&=i\hslash\Big([\nabla_{X_f},\nabla_{X_g}]\Big)+[\nabla_{X_f},g]-[\nabla_{X_g},f]+\cancelto{0}{[f,g]}\\
&=i\hslash\Big(\nabla_{[X_f,X_g]}+\frac{i}{\hslash}\omega(X_f,X_g)\Big)+X_f(g)-X_g(f)\\
&=i\hslash\Big(\nabla_{X_{\{f,g\}}}+\frac{i}{\hslash}\{f,g\}\Big)+\{f,g\}+\{f,g\}\\
&=i\hslash \nabla_{X_{\{f,g\}}}-\{f,g\}+\{f,g\}+\{f,g\}\\
&=Q(\{f,g\}).
\end{align*}
\end{proof}

Agora vamos dar uma olhada como ficam as prequantizações dos operadores de posição e de momento:

\begin{thing4}{Exemplo 22.4}[\cite{hallq}]\leavevmode
	Para o potencial simplético $\theta=p_jdx_j$,
	\begin{align*}
	Q(x_j)&=x_j+i\hslash\frac{\partial }{\partial p_j}\\
	Q(p_j)&=-i\hslash\frac{\partial }{\partial x_j}
	\end{align*}
\end{thing4}

Uma última proposição na seção mostra que a escolha do potencial simplético não faz muita diferencia no sentido de que as prequantizações que surgem de dois potenciais simpléticos são unitariamente equivalentes; o mapa unitário que as relaciona se chama de \textit{\textbf{gauge transformation}}.

\subsection{Quantização no espaço euclideano}

Por que essa correspondência se chama só de \textit{pre}quantização? A \hyperref[prop:22.1]{22.1} mostra uma propriedade importante para funções suaves em $\mathbb{R}^{2n}$. Qual seria o problema se trabalhasemos no espaço $L^2(\mathbb{R}^{2n}$? A resposta é que as quantizações dos operadores de posição e momento não  agem irreducivelmente no $L^2(\mathbb{R}^{2n})$. 

Talvez aqui só botar o espaço de Fock=espaço de Segel-Bergman com o que a gente já sabe. Daí passa no simplético. Pode fazer isso de Fock amanhã memo.

\section{Quantização geométrica em variedades simpléticas}

\subsection{Por que a geometria simplética é o scenario natural para a mecânica clássica}

Essa seção é inspirada \href{https://cohn.mit.edu/symplectic/}{neste documento}.

Sistema f\'isico \'e uma variedade com estrutura adicional. A variedade consiste dos estados do sistema (posi\c c\~ao, momento), e a estrutura adicional s\~ao as leis de movimento. A din\^amica do sistema est\'a determinada por uma fun\c c\~ao, o Hamiltoniano. Por medio de uma forma simpl\'etica podemos obter um campo vetorial associado a $H$ 

\begin{itemize}
\item $\omega$ n\~ao degenerada implica que sempre podemos achar esse campo vetorial

\item  $\omega$ alternante (\texttt{sg.pdf} prop. 6.11) implica que $H$ \'e constante ao longo do fluxo Hamiltoniano ($X_H$ aponta na  dire\c c\~ao de energia constante).

\item F\'ormula de Cartan implica que $\omega$  \'e constante ao longo do fluxo Hamiltoniano, ie. fluxo Hamiltoniano simpl\'etico (independente do tempo?) ie. $\mathcal{L}_{X_H}\omega=0$ se e somente se $\omega$ \'e fechada.
\end{itemize}

As equa\c c\~oes de Hamilton s\~ao s\'o outra formula\c c\~ao da segunda lei do Newton. O campo vetorial Hamiltoniano \'e uma formula\c c\~ao geom\'etrica das equa\c c\~oes de Hamilton.

\iffalse
{\color{3}Você tem os seguintes recursos falando sobre \textbf{prequantização}  (quantização em line bundles): \texttt{k3quant.pdf}, \texttt{raw.pdf}, \texttt{chernsimons.pdf}, \texttt{lecnotes.pdf},  \texttt{bere-top.pdf} e \texttt{wood.pdf}. São todos muito boms e tem muitos exemplos simples como esfera de Riemann, toro. \texttt{chernsimons.pdf} tem esse negocio de $\mathsf{SU}(2)$. Só \texttt{k3quant.pdf} fala da definição intuitiva.}

Começarei por presentar uma visão geral dos objetivos da quantizão geométrica. Mais pra frente vamos ver que essa construção não pode ser alcançada per se, e terminharemos desenvolvendo uma outra construção. Para essa noção inuitiva uso \cite{gq} e \cite{k3quant}.

Um dos pioneros da mecânica quântica, P. Dirac, desenvolveu a correspondência entre mecânica quântica e mecánica clássica. A ideia dele foi que tranformações 

Vou começar seguindo \cite{k3quant} para dar uma visão geral dos objetivos da quântização (isso também está em \href{https://en.wikipedia.org/wiki/Canonical_quantization#Classical_and_quantum_brackets}{Wiki, Canonical Quantization}). Começamos com o caso simples em que $M=\mathbb{R}^{n}$ e $\hslash$ é uma constante (a constante de Planck, que para nós é só um nome).

\begin{defn}\leavevmode
	Uma \textit{\textbf{quantização completa}} de $M$ é um mapa
	\[\mathcal{F}:f\longmapsto \hat{f}\]
	levando observáveis clássicos $f $, i.e. funções suaves em $T^*M$ a operadores autoadjuntos $\hat{f}$ num espaço de Hilbert $\mathcal{H}$ satisfazendo:
	\begin{enumerate}
		\item $\mathcal{F}$ é linear:
			\[\widehat{f+g}=\hat{f}+\hat{g},\qquad f,g\in\mathcal{C}^\infty(T^*M)\]
			\[\widehat{\lambda f}=\lambda\hat{f},\qquad \lambda\in\mathbb{R}.\]
	
		\item $ \mathcal{F}$ é um morfismo de álgebras de Lie salvo por uma constante:
			\[\widehat{\{f,g\}}=\frac{1}{\hslash}\left[\hat{f},\hat{g}\right]\]
\item A função constante $1$ corresponde com a identidade:
	\[\hat{1}=\operatorname{Id}\]

	\item As coordenadas $\hat{q}^i$ e $\hat{p}^i$ agem irreducívelmente em $\mathcal{H}=L^2(M)$.
	\end{enumerate}
\end{defn}

\subsection{Polarização}

\section{Espaço de Fock}

\section{Quantização de variedades Kähler}

\fi

\printbibliography

\end{document}
