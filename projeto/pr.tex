\input{/Users/daniel/github/config/preamble.sty}%available at github.com/danimalabres/config
\input{/Users/daniel/github/config/thms-por.sty}%available at github.com/danimalabres/config

\usepackage[style=authortitle-terse,backend=bibtex]{biblatex}
\addbibresource{bibliography.bib}

\begin{document}

\begin{minipage}{\textwidth}
	\begin{minipage}{1\textwidth}
		Geometria Simpl\'etica \hfill Daniel González Casanova Azuela
		
		{\small Profs. Henrique Bursztyn and Leonardo Macarini\hfill\href{https://github.com/danimalabares/sg}{github.com/danimalabares/sg}}
	\end{minipage}
\end{minipage}\vspace{.2cm}\hrule

\vspace{10pt}
{\huge Projeto}

\tableofcontents



\section{No\c c\~oes b\'asicas de mec\^anica qu\'antica}
\iffalse\begin{tabular}{c c}
Mec\^anica Cl\'asica	 & Mec\^anica qu\^antica\\
\hline
O estado de uma part\'icula est\'a \\determinado pela posi\c c\~ao e a velocidade.\\ Equivalentemente, est\'a determinada \\pela posi\c c\~ao $\mathbf{x}$ e o momento \\$\mathbf{p}=m\dot\mathbf{x}$ See \href{https://mathoverflow.net/questions/19932/what-is-a-symplectic-form-intuitively}{intuition on symplectic form}.& A fun\c c\~ao de onda cont\'em a informa\c c\~ao \\& da part\'icula em todo tempo.
\end{tabular}\fi

\subsection{Cl\'asica}

Sistema f\'isico \'e uma variedade com estrutura adicional. A variedade consiste dos estados do sistema (posi\c c\~ao, momento), e a estrutura adicional s\~ao as leis de movimento. A din\^amica do sistema est\'a determinada por uma fun\c c\~ao, o Hamiltoniano. Por medio de uma forma simpl\'etica podemos obter um campo vetorial associado a $H$ 

\begin{itemize}
\item $\omega$ n\~ao degenerada implica que sempre podemos achar esse campo vetorial

\item  $\omega$ alternante (\texttt{sg.pdf} prop. 6.11) implica que $H$ \'e constante ao longo do fluxo Hamiltoniano ($X_H$ aponta na  dire\c c\~ao de energia constante).

\item F\'ormula de Cartan implica que $\omega$  \'e constante ao longo do fluxo Hamiltoniano, ie. fluxo Hamiltoniano simpl\'etico (independente do tempo?) ie. $\mathcal{L}_{X_H}\omega=0$ se e somente se $\omega$ \'e fechada.
\end{itemize}

As equa\c c\~oes de Hamilton s\~ao s\'o outra formula\c c\~ao da segunda lei do Newton. O campo vetorial Hamiltoniano \'e uma formula\c c\~ao geom\'etrica das equa\c c\~oes de Hamilton.

\begin{prop}[18.9 \texttt{das.pdf}]
	$\{f,H\}=0$ ($f$ \'e primeira integral do fluxo de $X$) se e somente se $f$ \'e constante ao longo das curvas integrais de $X_F$.
\end{prop}

\subsection{Qu\^antica}



 Equa\c c\~ao de Schr\"odinger
 \[i\hbar \frac{\partial \psi}{\partial t}=\hat{H}\psi.\]
 Distintas escolhas de Hamiltoniano $\hat{H}$ descrevem diferentes leis da natureza. Para partículas não relativistas em treis dimensões com energia potencial $V(x)$, o Hamiltoniano  é
 \[\hat{H}=-\frac{\hbar^2}{2m}\nabla^2+V(x).\]
 \'E um operador diferencial. O Laplaciano \'e
 \[\nabla^2=\frac{\partial^2}{\partial^2 x^2}+\frac{\partial^2}{\partial^2 y^2}+\frac{\partial^2}{\partial^2z^2}.\]
 Na mec\^anica cl\'asica, o Hamiltoniano est\'a relacionado com a energia do sistema, que para n\'os \'e
 \[E=\frac{1}{2m}\mathbf{p}^2+V(x)\]
 onde $\mathbf{p}=m\dot{\mathbf{x}}$ o momentum da partícula.

 Nem toda teoria f\'isica pode ser descrita usando um Hamiltoniano. (Em termos gerais, s\'o as teorias que tem conserva\c c\~ao da energia podem ser descritas com o Hamiltonano.) Importantemente, isso mesmo acontece na mecânica quântica.

 O experimento do buraco duplo: a função de onda se comporta como part\'icula e como onda.

 \begin{defn}
 	Um \textit{\textbf{estado quântico}} é uma função de onda $\psi(\mathbf{x},t)$ normalizável, ie.
	\[\int d^3x|\psi|^2<\infty.\]
 \end{defn}
 Esses estados qu\^anticos moram num espaço de Hilbert (tem produto Hermitiano): se a part\'icula est\'a num espaço $M$, o espaço de Hilbert relavante \'e $L^2(M)$.

\begin{defn}
	\textit{\textbf{Observ\'avel}}: s\~ao fun\c c\~oes  de $\mathbf{x}$ e $\mathbf{p}$. Por exemplo, $\mathbf{x}$ e $\mathbf{p}$ mesmas, ou o \textit{momento angular} $\mathbf{L}=\mathbf{x}\times \mathbf{p}$ ou a \textit{energia} $E=\frac{\mathbf{p}^2}{2m}+V(\mathbf{x})$.

	Os observ\'aveis s\~ao representados por \textit{\textbf{operadores}} no espaço de Hilbert. Agem numa fun\c c\~ao de onda e d\~ao outra fun\c c\~ao.
\end{defn}

\begin{remark}
	O reemplazo das matrices nos espaços de dimens\~ao infinita s\~ao os operadores diferenciais.
\end{remark}

\begin{remark}
	O resultado de qualquer medi\c c\~ao de um operador est\'a no seu espectro (conjunto de eigenvalores).
\end{remark}

O espectro do Hamiltoniano determina os poss\'iveis n\'iveis de energia do sistema qu\^antico.

Todo observ\'avel f\'isico corresponde a um operador Hermitiano (autoadjunto).


\section{Quantização}

\href{https://math.stackexchange.com/questions/4540394/need-help-understanding-the-proof-of-diracs-famous-relation-between-commutators}{Aqui } tem uma perguna de StackExchange: "Need help understanding the proof of Dirac's famous relation between commutators and Poisson brackets". So maybe that could be kind of an exercise/proof to carry out in the talk.

No seguinte vou ler sobre tudo \href{https://en.wikipedia.org/wiki/Geometric_quantization}{wiki} e um pouco de  \texttt{quank3.pdf}

\subsection{Prequantização}

Passar de uma variedade simplética a um espaço de Hilbert assinalando operadores autoadjuntos às funções suaves na variedade, ie. oberváveis clássicos a observáveis quânticos, satisfazendo:
\begin{enumerate}
	\item Linearidade: $\widehat{f+g}=\hat{f}+\hat{g}$, $\widehat{\lambda f}=\lambda\hat{f}$.
	\item Morfismo de álgebras de Lie a menos de uma constante.
	\item Identidade vai para identidade.
	\item Os operadores $\hat{q_i}$ e $\hat{p_i}$ agem irreduzivelmente no espaço de Hilbert.
\end{enumerate}


Por exemplo, o espaço de funções integráveis respeito à forma de Liouville. Manda uma função suave $f$ em $Q(f):=-i\hbar \left( X_f+\frac{1}{i\hbar}\theta(X_f) \right) +f$.

Otro exemplo: um fibrado linear $L$ munido de uma conexão tal que a sua forma de curvatura é $ \omega /\hbar$. Aqui o espaço de Hilbert é o espaço de seções quadrado-integráveis de $L$ como o operador $Q(f)=-i\hbar \nabla_{X_f}+f$. Neste caso temos que $[Q(f),Q(g)]=i\hbar Q(\{f,g\}).$

\subsection{Polarização}

É a escolha de um subespaço Lagrangiano em cada ponto de $M$.

\begin{defn}[\href{https://ncatlab.org/nlab/show/polarization#OfASymplecticManifold}{nLab}]
	Uma \textit{\textbf{polarização}} de uma variedade simplética $(X,\omega )$ é a escolha de um subfibrado Lagrangiano involutivo $\mathcal{P}\hookrightarrow T_{\mathbb{C}}X$ do fibrado tangente complexificado de $X$.
\end{defn}

\begin{example}
	O fibrado holomorfo ou antiholomorfo caso $(X,\omega,g,I)$ seja K\"ahler.
\end{example}

A polarização no  \texttt{k3quant.pdf} é a escolha de um fibrado linear; a gente vai tomar seções holomorfas para construir o espaço de Hilbert.

\subsection{\texttt{k3quant.pdf}}

After the definition of \textit{\textbf{full quantization}} written above, we have definition of \textit{\textbf{quantizable}} manifold in terms of curvature in the subbundle. And construction of Berezin-Toeplitz operators, which includes the construction of the Hilbert space of $L^2$-measuable sections of the bundle. Then \textit{\textbf{Hardy space}}, should I read this? Is there a proof that the operatrs are good?

\subsection{Coherent states}

Let's read a little bit of \href{https://en.wikipedia.org/wiki/Coherent_states_in_mathematical_physics}{wiki}. A \textit{\textbf{family of coherent states}} is a set of vectors in some locally compact space  $X$ that satisfy the following properties in relation with some complex separable hilbert space $\mathfrak{H}$. To every element $x$ in the family of coherent states asocciate an element $|x\rangle$ and ask that
\begin{enumerate}
	\item the map $x\mapsto |x\rangle$ be weakly continuous (every functional evaluated on the vector is continuous… somehow)

	\item Some condition called \textit{\textbf{resolution of the identity}} that "ensures that an arbitrary vector  $|\psi\rangle$ be expressible as a linear integral combination of these (the coherent states?) vectors. Indeed the resolution of the identity immediately implies that $|\psi\rangle=\int_{X}\Psi(x)|x\rangle d\nu(x)$" where $\Psi(x)=\left<x|\psi\right> $ and $d\nu$ is a measure on $X$.
\end{enumerate}

{\color{4}But how is this related to the quantization in \texttt{raw.pdf}}?

{\color{1}\begin{quotation}
		"The Hilbert space on which th efunctions act as operators consists of holomorphic sections of a suitably chosen line bundle. Evaluation at a point is a continuous linear functional {\color{3}Evaluation at a point gives me a vector in the line bundle which is a functional on $\mathcal{C}^\infty(X)$ functions} and gives rise to a vector in the Hilbert space {\color{3}How because a vector in the Hilbert space is a whole section…} which is labelled by that point.
\end{quotation}}
\end{document}
