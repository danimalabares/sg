\input{/Users/daniel/github/config/preamble-por.sty}
\input{/Users/daniel/github/config/thms-por.sty}

\begin{document}

\begin{minipage}{\textwidth}
	\begin{minipage}{1\textwidth}
		Geometria Simpl\'etica \hfill Daniel González Casanova Azuela
		
		{\small Profs. Henrique Bursztyn and Leonardo Macarini\hfill\href{https://github.com/danimalabares/sg}{github.com/danimalabares/sg}}
	\end{minipage}
\end{minipage}\vspace{.2cm}\hrule

\vspace{10pt}
{\huge Lista 1}


\paragraph{Problem 1:} Let $V$ be a symplectic vector space ($\dim V=2n$), and $\Omega\in \Lambda^{2} V^{*}$be a skew-symmetric bilinear form. Show that $\Omega$ is nondegenerate iff $\Omega^{n} \neq 0$.

\begin{proof}[Solution]
	I first tried to show that $\Omega$ is degenerate iff $\Omega^{n}=0$. Suppose there is a vector $v_0$ such that $\Omega( v_0,w)=0$ for all $w\in V$ and complete to a basis. Then for any $v_1,v_2,v_3,v_4\in V$ we have
	 \begin{align*}
		 (\Omega\wedge \Omega)(v_1,v_2,v_3,v_4)&=\sum_{\sigma\in S_{4}}\operatorname{s gn}(\sigma)\Omega(v_{\sigma(1)},v_{\sigma(2)})\Omega(v_{\sigma(3)},v_{\sigma(4)}).
		 \end{align*}

		 Finally I found this proposition in Lee, Intro. Smooth Manifolds.
\end{proof}

\paragraph{Problem 2:} Let $(V,\Omega)$ be a symplectic vector space, and let $W\subseteq V$ be any linear subspace.
\begin{enumerate}[label=\alph*.]
	\item Show that $V_{W}=\frac{W}{W\cap W^{\Omega}}$ inherits a natural symplectic structure $\Omega_{W}$ uniquely determined by the condition $\pi^{*} \Omega_{W}=\Omega|_{W}$ (here $\pi:W\to W/(W\cap W^{\Omega}) $ is the quotient projection).
	
		(\textit{The space $(V_{W},\Omega_{W})$ is called the \textbf{reduced space}.})

	\item Suppose that $W$ is coisotropic, and let $L\subset V$ be lagrangian. Show that the image of $L\cap W$ via $\pi:W\to V_{W}$ is lagrangian in the reduced space.
\end{enumerate}

\begin{proof}[Solution]\leavevmode 
	\begin{enumerate}[label=\alph*.]
		\item Define
			\[\Omega_{W}([w_1],[w_2]):=\Omega(w_1,w_2)\]
			for any equivalence classes $[w_1],[w_2]\in V_{W}$. Let's check that this is well defined. Suppose $w_1'\in [w_1]$. Then $w_1-w_1'\in W\cap W^{\Omega}$ so $\Omega(w_1-w_1',w_2)=0$ since $w_2\in W$ and $w_1-w_1'$ is, in particular, in $W^{\Omega}$. So $\Omega(w_1,w_2)=\Omega(w_1',w_2)$. {\color{magenta}Why not quotient only by $W^{\Omega}$? Looks like I didn't use the $W$ part…}

			Recall that $\pi^{*} \Omega_{W}(w_1,w_2)=\Omega_{W}([w_1],[w_2])$. It is straightforward to check that $\Omega_{W}$ is the only symplectic form on $V_{W}$ satisfying $\pi^{*} \Omega_{W}=\Omega|_{W}$: if $\Omega_{W}'$ is another such form, then $\Omega'_{W}([w_1],[w_2])=\Omega|_{W}(w'_1,w'_2)=\Omega_{W}([w_1],[w_2])$ for any $w_1'\in [w_1]$ and $w_2'\in [w_2]$.

			\item Since $W$ is coisotropic, we have $V_W=W/W^\Omega$ and $L^\Omega=L$. Then
				\begin{align*}
					\pi(L)^\Omega & =\{[w]\in W/W^\Omega:\Omega(w,\ell)=0\} \\
						      &=\{[w]\in W/W^\Omega:w\in L^\Omega=L\} \\
						      &=\pi(L)
				\end{align*}
				{\color{magenta}But I didn't use the $W$ is coisotropic…}
\end{enumerate}
\end{proof}

\paragraph{Problem 3:}  We saw in class that any symplectomorphism $T:V_1\to V_2$ defines a lagrangian subspace by its graph: $\Gamma_{T}:=\{(Tu,u):u\in V_1 \}\subset V_2\oplus \overline{V}_{1}$. (Recall that if $(V,\Omega)$ is a svs,  $\overline{V}$ denotes $(V,-\Omega)$.) So we think lagrangian subspaces of $V_2\oplus \overline{V}_{1}$ a generalizations of symplectomorphisms. We now see how to generalize their composition. 

Consider symplectic vector spaces  $V_1,V_2,V_3$ and $E=V_3\oplus \overline{ V}_{2}\oplus V_2\oplus \overline{V}_1$.
\begin{enumerate}[label=\alph*.]
	\item Show that $\Delta :=\{(v_3,v_2,v_2,v_1)\in E\} $ is coisotropic in $E$ and its reduction $E_{\Delta}$ can be identified with $V_3\oplus \overline{V}_{1}$.

	\item Given lagrangian subspaces $L_1\subset V_2\oplus \overline{V}_{1}$ and $L_2\subset  V_3\oplus \overline{V}_{2}$, define the \textit{\textbf{composition}} of $L_2$ and $L_1$ by
		\[L_2\circ L_1:=\{(v_3,v_1)|\exists v_2\in V\text{ s.t. } (v_3,v_2)\in L_2,(v_2,v_1)\in L_1\}. \]
		Show that $L_2\circ L_1$ is a lagrangian subspace of $V_3\oplus \overline{V}_{1}$. (\textit{Hint: show that the composition can be identified with the reduction of $L_2\times L_1\subset E$ with respect to $\Delta$}).

	\item Let $T_1:V_1\to V_2$ and  $T_2:V_2\to V_3$ be symplectomorphisms. Show that $\Gamma_{T_2\circ T_1}=\Gamma_{T_2}\circ \Gamma_{T_1}$.
\end{enumerate}

\begin{proof}[Solution]\leavevmode
	\begin{enumerate}[label=\alph*.]
		\item Let $v:=(v_3,v_2,v_2',v_1)\in \Delta^{\Omega_E}\subset E$ where  $\Omega_E=\Omega_1\oplus -\Omega_2\oplus \Omega_2\oplus -\Omega_1$ is the symplectic form on $E$. We wish to show that $v\in \Delta$, which only means that $v_2=v_2'$. So let $\tilde{v}=(v_3,v_2,v_2,v_1)$ and $\hat{v}=(v_3,v_2',v_2',v_1)$ which are both in $\Delta$. Then we have
			\begin{align*}
				0&=\Omega_E(v,\tilde{v})\\
				& =\Omega_1(v_3,v_3)-\Omega_2(v_2,v_2)+\Omega_{2}(v_2',v_2)-\Omega_{1}(v_1,v_1)
			\end{align*}
and likeways
			\begin{align*}
				0&=\Omega_E(v,\hat{v})\\
				& =\Omega_1(v_3,v_3)-\Omega_2(v_2,v_2')+\Omega_{2}(v_2',v_2')-\Omega_{1}(v_1,v_1).
			\end{align*}
Substracting,
\[0=-\Omega_2(v_2,v_2)+\Omega_2(v_2,v_2')+\Omega_2(v_2',v_2)-\Omega_2(v_2',v_2')\]
And by linearity,
\begin{align*}
0&=-\Omega_2(-v_2+v_2,-v_2+v_2')+\Omega_2(v_2'-v_2',v_2-v_2')\\
\implies 0&=-\Omega_2(0,-v_2+v_2')+\Omega_2(0,v_2-v_2')\\
\implies 0&=\Omega_2(-v_2+v_2',0)+\Omega_2(0,v_2-v_2') \\
\implies 0&=\Omega_2(-v_2+v_2',v_2-v_2')\\
\implies 0&=\Omega_2(-(v_2-v_2'),v_2-v_2')
\end{align*}
and it follows that $v_2=v_2'$ from nondegeneracy.

Now let's try to construct an isomorphism $E_{\Delta}=V_3\oplus \overline{V}_{1}$. Consider
\begin{align*}
	\varphi: E &\longrightarrow V_3\oplus \overline{V}_{1} \\
	(v_3,v_2,v_2',v_1) &\longmapsto (v_3,v_1)
\end{align*}
which is clearly surjective and not injective, so perhaps its kernel is $ \Delta \cap \Delta^{\Omega}$. But $\ker \varphi=\{(0,v_2,v_2',0)\}$, so unfortunately no.

But perhaps we can construct some other map. Let's try
\begin{align*}
	\varphi: E &\longrightarrow V_3\oplus \overline{V}_{1} \\
	(v_3,v_2,v_2',v_1) &\longmapsto 
\end{align*}

	\item It looks like $L_2\circ L_1$ is very much like $E_\Delta$ from the last exercise. $L_2\circ L_1$ is \textit{strictly}  contained in $V_3\oplus \overline{V}_1\cong E_\Delta$.

		Ok let's have a go at the hint. Perhaps  $(L_2\times L_1)_\Delta= L_2\circ L_1$ and it is lagrangian. That'd be great. OK let's compute it. This is how to do it:
\begin{align*}
	\varphi: L_2\times L_1 &\longrightarrow V_2\circ V_1 \\
	(v_2,v_1,v_3,v_2) &\longmapsto 
\end{align*}



	
\end{enumerate}
\end{proof}

\paragraph{Problem 4:} Let $(V,J)$ be a complex vector space, let $\Omega$ be a sympletic structure on $V$. Show that $J$ and $\Omega$ are compatible iff there exists a hermitian inner product $h:V\times V\to \mathbb{C}$ such that $\Omega$ is its imaginary part. Show that any (complex) orthonormal basis of  $(V,h)$ can be extended to a symplectic basis of $(V,\Omega)$.

\begin{proof}[Solution]\leavevmode
	First suppose that $J$ and $\Omega$ are compatible, ie., $g(u,v):=\Omega(u,Jv)$ is an inner product. Define $h(u,v)=g(u,v)+i\Omega(u,v)$. Then  $h$ is the required hermitian inner product. Indeed:

	\begin{enumerate}
		\item The properties $h(u_1+u_2,v)=h(u_1,v)+h(u_1,v)$ and  $h(u,v_1+v_2)=h(u,v_1)+h(u,v_2)$ follows easily from linearity of $g$ and $\Omega$.
		
		\item $h(\lambda u,v)=\lambda h(u,v)$ follows again from linearity of $g$ and $\Omega$.

		\item The property $h(u,\lambda v)=\bar{\lambda} h(u,v)$ follows easily from 2. and 4. since 
			\begin{align*}
				h(u,\lambda v)& =\overline{h(\lambda v, u)}\\
				& =\bar{\lambda} \overline{h(v,u)}\\
				& =\bar{\lambda} h(u,v)
			\end{align*}

		\item $h(u,v)=\overline{h(v,u)}$ is clear by anti-symmetry of $\Omega$:
			\begin{align*}
				h(u,v)& =g(u,v)+i\Omega(u,v)\\
				& =g(v,u)-i\Omega(v,u)\\
				& =\overline{h(v,u)}
			\end{align*}
	\end{enumerate}
	For the converse suppose that $h$ is an hermitian inner product such that $\Omega$ is its imaginary part. Then $g(u,v):=\Omega(u,Jv)$ is an inner product:
	\begin{enumerate}
		\item Linearity of $g$ is immediate from linearity of $\Omega$ and $J$.

		\item Symmetry follows from 
			 \begin{align*}
				g(u,v)&=\Omega(u,Jv)\\
				&=\Omega(-J^{2} u,Jv)\\
				& =-\Omega(J^{2} u,Jv)\\
				& =\Omega(Jv,J^{2} u)\\
				&=\Omega(v,Ju)\\
				&=g(v,u)
			\end{align*}
		provided $\Omega(u,v)=\Omega(Ju,Jv)$. This holds since $\Omega$ is the imaginary part of $h$ identifying $J$ with multiplication by $i$:
			\begin{align*}
				\Omega(Ju,Jv)&=\operatorname{Im}(h (Ju,Jv))\\
				& =\operatorname{Im}h(iu,iv)\\
				&=\operatorname{Im}(i\bar{i}h(u,v))\\
				&=\operatorname{Im}(h(u,v))\\
				&=\Omega(u,v).
			\end{align*}

		\item For positive-definiteness let $u\neq 0$. Then
			\begin{align*}
				g(u,u)&=\Omega(u,Ju)\\
				&=\operatorname{Im}(h(u,Ju))\\
				&=\operatorname{Im}(h(u,iu))\\
				&=\operatorname{Im}(-ih(u,u))
			\end{align*}
			e sabemos que $h(u,u)>0$ … {\color{magenta}parece que $g(u,u)$  \'e negativo…}
	\end{enumerate}

\end{proof}

 \paragraph{Problem 5:} Consider the symplectic vector space $(\mathbb{R}^{2n},\Omega_0)$, where $\Omega_0(u,v)=-u^{\mathbf{T}} J_0v$. Check that its group of linear symplectomorphisms is given by $\operatorname{Sp}(2n)=\{A\in \operatorname{GL}(2n):A^{\mathbf{T}}J_0A=J_0\}.$ Show that $\operatorname{Sp}(2n)$ is a smooth submanifold of $\operatorname{GL}(2n)$ and that its tangent space at the identity $I\in \operatorname{GL}(2n)$ is given by $T_{I}\operatorname{Sp}(2n)=\{A:\mathbb{R}^{2n}\to \mathbb{R}^{2n}|A^{\mathbf{T}} J_0+J_0A=0\} $. Conclude that $\operatorname{Sp}(2n)$ has dimension $2n^{2} +n$. Verify also that $\operatorname{Sp}(2n)$ is not compact.

\begin{proof}[Solution]\leavevmode
	Suppose that $A$ is a linear symplectomorphism of $(\mathbb{R}^{2n},\Omega_0)$. Then $A^{*}\Omega_0=\Omega_0$ so 
	\[A^{*}\Omega_0(u,v) =\Omega_0(Au,Av)=-(Au)^{\mathbf{T}}J_0(Av)=-u^{\mathbf{T}}A^{\mathbf{T}}J_0(Av)\]
is equal to
		\[\Omega_0(u,v)=-u^{\mathbf{T}}J_0v\]
In terms of usual dot product of $\mathbb{R}^{2n}$, which we can denote by $\left<\cdot ,\cdot \right> $ momentarily, this means that
\begin{align*}
	\left<-u^{\mathbf{T}},A^{\mathbf{T}}J_0Av\right> &=\left<-u^{\mathbf{T}},J_0v\right>\\
	\iff\left<-u^{\mathbf{T}},A^{\mathbf{T}}J_0Av-J_0v\right> &=0
\end{align*}
for all $u\in\mathbb{R}^{2n}$, which means that $A^{\mathbf{T}}J_0Av=J_0v$ since dot product is nondegenerate. For the converse, if $A^{\mathbf{T}}J_0A=J_0$, we have
\begin{align*}
	\Omega(u,v)&=-u^{\mathbf{T}}J_0v=-u^{\mathbf{T}}A^{\mathbf{T}}J_0Av=-(Au)^{\mathbf{T}}J(Av)=\Omega(Au,Av_0)=A^*\Omega(uv).
\end{align*}


Let's try to show that $\operatorname{Sp}(2n)$ is the inverse image of a regular value of some submersion $\operatorname{GL}(2n)\to \mathbb{R}$.  I think the determinant of $J_0$ is $1$, so perhaps the map
\begin{align*}
	 D:\operatorname{GL}(2n) &\longrightarrow \mathbb{R} \\
	A &\longmapsto \det (A^{\mathbf{T}}J_0A)
\end{align*}
has inverse image of 1 equal to $\operatorname{Sp}(2n)$. It is clear that $\operatorname{Sp}(2n)\subset D^{-1}(1)$. Now suppose $A\in\operatorname{GL}(2n)$ is in $D^{-1}(1)$, but then it's not necessarily true that $A^{\mathbf{T}}J_0A=J_0$ because there are so many matrices with determinant 1 that are not $J_0$.

Maybe the determinant won't work. But what about
\begin{align*}
	D: \operatorname{GL}(2n) &\longrightarrow \operatorname{GL}(2n) \\
	A &\longmapsto A^{\mathbf{T}}J_0A
\end{align*}
maybe this map is a submersion at every point of $D^{-1}(J_0)=\operatorname{Sp}(2n)$. It really looks like a submersion at every point: it's only a composition of linear isomorphisms… of course its derivative is surjective, right? But something seems to be wrong because then it would be a sumbanifold of dimension $2n-2n=0$…

Let's have a look at the tangent space at the identity. Suppose that $V \in T_{I}\operatorname{Sp}(2n)$. I'm not sure how to continue here…

Following Misha's slides, it could be shown that $\operatorname{Sp}(2n)$ is the exponent of its Lie algebra (ie. the tangent space of the identity), and that this Lie algebra is $W:=\{A\in\operatorname{GL}(2n):A^{\mathbf{T}}J_0+J_0A=0\}$. This suggests that $\operatorname{exp}W =\operatorname{Sp}(2n)$. But then we need a theorem saying that if the exponent of a subset of endomorphisms is a Lie group then such a subspace is its Lie algebra. This can be found in Misha's notes as follows: 

Exponent is a map 
\begin{align*}
	\operatorname{exp}: \operatorname{End}V &\longrightarrow \operatorname{End}V \\
	A &\longmapsto \sum_{n=0}^{\infty}\frac{A^{n}}{n!}
\end{align*}
and its differential at $0\in\operatorname{End}V$ is the identity. But why… we have
\[d_0\operatorname{exp}=\sum_{n=1}^{\infty}\frac{A^{n}}{(n-1)!}\]
\href{https://en.wikipedia.org/wiki/Derivative_of_the_exponential_map#Statement}{I don't think} it's so straightforward to check that this is the identity so let's just suppose it is true.

But what does it tell us that this map is the identity? Well, that $\operatorname{exp}(W)$ is invertible around 0, but not only that: notice that $\operatorname{exp}(0) =I\in\operatorname{GL}(2n)$ so $d_0\operatorname{exp}: T_0\operatorname{GL}(2n)\cong \operatorname{GL}(2n)\to T_{\operatorname{I}}\operatorname{Sp}(2n)$ is an isomorphism as required, provided of course that $\operatorname{exp}(W) =\operatorname{Sp}(2n)$. So let's check that.

But that's nowhere near obvious:
\[\operatorname{exp}(W) = \left\{ \sum_{n=0}^{\infty}\frac{A^{n}}{n!}:A^{\mathbf{T}}J_0+J_0A=0 \right\}\quad \overset{?}{=}\quad \{A\in\operatorname{GL}(2n):A^{\mathbf{T}}J_0A=J_0\}=\operatorname{Sp}(2n) \]
Let's just go, let $A\in W$ and let's check wether $\operatorname{exp}(A)^{\mathbf{T}}J_0\operatorname{exp}(A) =J_0$. It \href{https://math.stackexchange.com/questions/1021900/proof-of-transpose-property-of-matrix-exponential}{looks like} transpose of exponent is exponent of transpose, but what is $\operatorname{exp}(J_0)$? \href{https://www.wolframalpha.com/input?i=exp%28%7B%7B0%2C0%2C0%2C-1%7D%2C%7B0%2C0%2C-1%2C0%7D%2C%7B0%2C1%2C0%2C0%7D%2C%7B1%2C0%2C0%2C0%7D%7D%29&lang=es}{This looks random…}

\end{proof}

 \paragraph{Problem 6:} Consider the standard compatible triple $(\Omega_0,J_0,g_0)$ on $\mathbb{R}^{2n}$. Let $\operatorname{O}(2n)$ be the linear orthogonal group of $\mathbb{R}^{2n}$ (i.e., linear transformations preserving the canonical inner product $g_0$), and let $\operatorname{Sp}(2n)$ be the symplectic linear group. Through the identification $\mathbb{R}^{2n}\cong \mathbb{C}^{n}$ (as complex vector spaces), we may see $\operatorname{GL}(n,\mathbb{C})$ (the group of linear automorphisms of $\mathbb{C}^{n}$) as a subgroup of $\operatorname{GL}(2n,\mathbb{R})$ : a complex matrix $A+iB$ is identified with the real $2n\times 2n$ matrix
 \[\begin{pmatrix}A&-B\\B&A\end{pmatrix}\]
Let now $\operatorname{U}(n)\subset\operatorname{GL}(n,\mathbb{C})$ be the group of linear transformation preserving the natural hermitian inner product of $\mathbb{C}^{n}$. Show that the intersection of any two of the groups
\[\operatorname{Sp}(2n),\operatorname{O}(2n),\operatorname{GL}(n,\mathbb{C})\subset\operatorname{GL}(2n,\mathbb{R})\]
is $\operatorname{U}(n)$.

\begin{proof}[Solution]\leavevmode

	Since the standard hermitian product of $\mathbb{C}^{n}$ is given by $h_0=g_0+i\Omega_0$, it is immediate that a transformation $A\in\operatorname{Sp}(2n)\cap \operatorname{O}(2n)$ preserves $h_0$ and conversely:
	\[A^*h=A^*(g+i\Omega)=A^*g+iA^*\Omega=g+i\Omega=h\]
	provided that the pullback is complex-linear.

	For the next item recall that
	\[\operatorname{O}(2n)=\{A\in\operatorname{GL}(2n):A^{\mathbf{T}}A=I\},\qquad \operatorname{GL}(n,\mathbb{C})=\{A\in\operatorname{GL}(2n):AJ_0=J_0A\}\]
	again identifying $J_0$ with multiplication by $i$. Observe that this implies that $A\in\operatorname{O}(2n)\cap \operatorname{GL}(n,\mathbb{C})\implies  A\in\operatorname{Sp}(2n)$ since
	\[A^{\mathbf{T}}J_0A=A^{\mathbf{T}}A J_0=J_0.\]
	Likeways we see that $A\in\operatorname{Sp}(2n)\cap \operatorname{GL}(n,\mathbb{C})\implies \operatorname{O}(2n)$ since
	\[A^{\mathbf{T}}J_0A=J_0\iff A^{\mathbf{T}}A J_0=J_0\iff A^{\mathbf{T}}A=I\]
	since $J_0$ is invertible. Going back to the initial argument for matrices in $\operatorname{Sp}(2n)\cap A\in\operatorname{O}(2n)$, we see that in both cases $A\in\operatorname{U}(n)$.

	For the converse notice that it is also true that $A\in\operatorname{Sp}(2n)\cap \operatorname{O}(2n)\implies A\in\operatorname{GL}(n,\mathbb{C})$ since
	\[J_0=A^{\mathbf{T}}J_0A=A^{-1} J_0A\iff J_0A=A J_0.\]
\end{proof}

\paragraph{Problem 7:} Let $(V,\Omega)$ be a symplectic vector space, let $W\subseteq V$. Let $J$ be a $\Omega$-compatible complex structure and $g$ the corresponding inner product. Verify that $J(W^{\Omega} )=W^{\perp_{g}}$.
\begin{enumerate}[label=\alph*.]
	\item Use this fact to show that any coisotropic subspace of $V$ has an isotropic complement. In particular, any lagrangian subspace $L\subset V$ has a lagrangian complement $L'$, $V=L\oplus L'$.
	
\item Show that there is a natural identification $L'\cong L^{*}$, that induces a symplectomorphism $V\cong L\oplus L^{*}$, where $L\oplus L^{*}$ has the natural symplectic structure \[\left( (\ell,\alpha),(\ell',\alpha') \right)\longmapsto\alpha(\ell')-\alpha'(\ell)\].
\end{enumerate}

\begin{proof}[Solution]\leavevmode
	First let's check that $J(W^{\Omega} )=W^{\perp_{g}}$. Indeed,
	\begin{align*}
		J(W^{\Omega} )&= \{Jv:v\in W^{\Omega}\}\\
&=\{Jv:\Omega(v,w)=0\;\forall w\in W\} \\
&=\{Jv:-\Omega(w,v)\;\forall w\in W\} \\
&=\{Jv:\Omega(w,-v)\;\forall w\in W\} \\
&=\{Jv:\Omega(w,J^{2}v)\;\forall w\in W\}
	\end{align*}
re-write $Jv:=\tilde{v}$ using that $J$ is bijective:
	\begin{align*}
J(W^{\Omega)}&=\{\tilde{v}\in V:\Omega(w,J\tilde{v})=0\;\forall w\in W\} \\
&=\{ \tilde{v}\in V:g(\tilde{v},w)=0\;\forall w\in W\} \\
& =W^{\perp_{g}}
	\end{align*}
	\begin{enumerate}[label=\alph*.]
		\item Let $W$ be any coisotropic subspace. We know that $V=W\oplus W^{\perp_g}$ (supposing that $V$ is finite-dimensional), so it remains to show that $W^{\perp_g}$ is isotropic.
		Since $W$ is coisotropic, we have
		\[W^{\Omega}\subseteq W\implies J(W^{\Omega})=W^{\perp_g}\subseteq JW\]
		so it would be enough to show that
		\[JW\subseteq\big(J(W^{\Omega})\big)^{\Omega}=(W^{\perp_g})^{\Omega}.\]
		Let $w\in W$ and $w'\in W^{\Omega}$, so that $Jw\in JW$ and $Jw'\in J(W^{\Omega})$. Then
		\[\Omega(Jw,Jw')=\Omega(w,w')=0,\]
		which shows that $JW\subseteq\big(J(W^{\Omega})\big)^{\Omega}$.

		\item
	\end{enumerate}


\end{proof}

\paragraph{Bonus problem: } 

\end{document}
