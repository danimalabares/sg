\input{/Users/daniel/github/config/preamble.sty}%This is available at github.com/danimalabares/config

\usepackage[style=authortitle-terse,backend=bibtex]{biblatex}
\addbibresource{bibliography.bib}

\begin{document}

\begin{minipage}{\textwidth}
	\begin{minipage}{1\textwidth}
		Geometria Simpl\'etica \hfill Daniel González Casanova Azuela
		
		{\small Profs. Henrique Bursztyn e Leonardo Macarini\hfill\href{https://github.com/danimalabares/sg}{github.com/danimalabares/sg}}
	\end{minipage}
\end{minipage}\vspace{.2cm}\hrule

\vspace{10pt}
{\huge Lista 4}

\tableofcontents

\addcontentsline{toc}{subsection}{Problem 1}
\begin{idea4}{Problema 1}\leavevmode
Let $M$ be a compact, connected, orientable $n$-dimensional manifold. Let $\Lambda_0,\Lambda_1\in\Omega^{n}(M)$ be two volume forms on $M$ such that $\int_{M}\Lambda_0=\int_{M}\Lambda_1$. Show that there is a diffeomorphism $\phi \in \operatorname{Dif}(M)$ such that $\phi^*\left( \Lambda_1 \right) =\Lambda_0$.
\end{idea4}

\begin{proof}[Solução]
Aqui sigo as definições em \cite{lee}, p. 380.  Como $M$ é orientada, em cada ponto podemos pegar um marco orientado (i.e. que em cada ponto pertence à clase de equivalencia dada pela orientação) $E_1,\ldots,E_n$ tal que as formas $\Lambda_0$ e $\Lambda_1$ são sempre positivas ou sempre negativas. Mas ainda, como $\int_{M}\Lambda_0=\int_{M}\Lambda_1>0$,
\[\Lambda_0(E_1,\ldots,E_0),\Lambda(E_1,\ldots,E_n)>0\]
para qualquer marco orientado. Daí é claro que $\Lambda_t(E_1,\ldots,E_n)>0$, de modo que $\Lambda_t$ não pode ser a forma zero em nenhum ponto de $M$, i.e. é uma forma de volumen.

	Para ver que $\left[ \Lambda_0 \right] =\left[ \Lambda_1 \right] $ lembre que $H^{n}(M)$ tem dimensão 1. Daí existe um escalar $\alpha$ tal que $\left[ \Lambda_0 \right] =\alpha \left[ \Lambda_1 \right] $. Mas, como a integral está bem definida em classes de cohomologia, $\int_{M}\left[ \Lambda_0 \right] =\int_{M}\left[ \Lambda_1 \right] \implies \alpha=1$.

	Esse argumento pode ser usado direitamente em $\Omega^{n}(M)$ em lugar de $H^{n}(M)$, concluindo que $\Lambda_0=\Lambda_1$. Mas acho que isso nem sempre é verdade.

Para concluir só devemos aplicar o Método de Moser. Já temos uma família de formas cohomologas, assim existe uma isotopía $ \varphi_t$ tal que $\varphi^*_{t}\Lambda_t=\Lambda_0$. Pegando $t=1$ obtemos o difeomorfismo buscado.
\end{proof}

\addcontentsline{toc}{subsection}{Problem 2}
\begin{idea8}{Problem 2}\leavevmode
	Give an example of two symplectic forms on $\mathbb{R}^{4}$ that induce the same orientation, but admit a convex combination that is degenerate. Is it possible to find an example like that, but admitting another of \textit{symplectic} forms from one to the other? What happens if we consider $\mathbb{R}^{2}$ instead of $\mathbb{R}^{4}$?
\end{idea8}

\begin{proof}[Solução]\leavevmode
	{\color{2} (See \href{https://math.stackexchange.com/questions/2864834/non-linear-path-between-symplectic-forms-in-mathbbr4}{StackExchange}. Here can also be found a nice general explanation of this problem where degenerate forms are seen as a hypersurface in the space of 2-forms.)}\hspace{.5em} Lembre que no problema 1 da lista 1 vimos que uma 2-forma $\omega$ é não degenerada se é só se $\omega^n\neq 0$. No nosso caso, qualquer 2-forma em $\mathbb{R}^{4}$ pode ser expressada como
	\[\omega = \alpha \, dx \wedge dy + \beta \, dx \wedge dz + \gamma \, dx \wedge dw + \delta \, dy \wedge dz + \varepsilon \, dy \wedge dw + \phi \, dz \wedge dw \, .\]
	Daí,
	\[\omega \wedge \omega =2 F \, dx \wedge dy \wedge dz \wedge dw\]
	onde $F = \alpha \phi - \beta \varepsilon + \gamma \delta$ (vou fazer essa conta num caso análogo abaixo). Segue que $\omega$ é não degenerada se e só se $F\neq 0$. (Então as formas degeneradas são a conica $F=0$)

	Nosso primeiro problema é achar $\omega_0$ e $\omega_1$ tais que as suas funções associadas como acima, $F_1$ e $F_2$, sejam não-zero, mas que exista uma combinação convexa delas $\omega_t$ cuja função $F_t$ sim seja zero. Note que se $\omega_t=(1-t)\omega_0+t\omega_1$,
\begin{align*}
	\omega_t\wedge \omega_t&=\Big( (1-t)\omega_0+t\omega_1 \Big) \wedge \Big( (1-t)\omega_0+t \omega_1 \Big) \\
	&=(1-t)^2\omega_0\wedge \omega_0+t(1-t)\Big(\omega_0\wedge \omega_1+\omega_1\wedge \omega_0\Big)+t^2\omega_1\wedge \omega_1\\
	&=(1-t)^2\omega_0\wedge \omega_0+\Big(2t(1-t)\Big)\omega_0\wedge \omega_1+t^2\omega_1\wedge \omega_1\
\end{align*}
de forma que estamos interessados em calcular $\omega_0\wedge \omega_1$:
\begin{align*}
	&\omega_0\wedge \omega_1\\
	&=\Big( \alpha_1 \, dx \wedge dy + \beta_1 \, dx \wedge dz + \gamma_1 \, dx \wedge dw + \delta_1 \, dy \wedge dz + \varepsilon_1 \, dy \wedge dw + \phi_1 \, dz \wedge dw \Big)\\
	&\wedge \Big( \alpha_2 \, dx \wedge dy + \beta_2 \, dx \wedge dz + \gamma_2 \, dx \wedge dw + \delta_2 \, dy \wedge dz + \varepsilon_2 \, dy \wedge dw + \phi_2 \, dz \wedge dw \Big) \\
	&= 2\alpha_1 \phi_2dx\wedge dy\wedge dz\wedge dw+\;2\beta_1\varepsilon_2 dx\wedge dz\wedge dy\wedge dw+\; 2\gamma_1\delta_2dx\wedge dw \wedge dy\wedge dz
\end{align*}
de forma que
\[\omega_0\wedge \omega_1=2\Big(\alpha_1\phi_2-\beta_1 \varepsilon_2+\gamma_1 \delta_2\Big)dx\wedge dy\wedge dz\wedge dw\]
Então pegue $\alpha_1=\alpha_2=\phi_1=\phi_2=\beta_1=\gamma_2=1$ e o resto zero. Obtemos
\[\omega_t\wedge \omega_t=2\left( 2(1-t)^2+2 \right)  \]

The best proof that there exists a path of symplectic forms joining $\omega_0$ to $\omega_1$ consists in showing that the hypersurface $[F=0]$ separates $\mathbb{R}^{4}$ into two connected components given by $[F>0]$ and  $F<0$ (which is proved in \href{https://math.stackexchange.com/questions/2864834/non-linear-path-between-symplectic-forms-in-mathbbr4}{StackEchange}). Then showing that there is a path joining  $\omega_0$ and $\omega_1$ ammounts to showing that they belong to the same connected component, i.e. $F_i$ is positive or negative for both $i=1,2$, which is indeed the case since we have $F_1=F_2=1$.

Para o nosso propósito basta achar notar o seguinte: os valores de $\beta$ e $\gamma$ não alteram a conta anterior, de modo que 

\end{proof}

\addcontentsline{toc}{subsection}{Problem 3}
\begin{idea5}{Problem 3}\leavevmode
Let $(V,\Omega)$ be a symplectic vector space (or vector bundle) and let $W\subseteq V$ be a coisotropic subspace (or bundle).
\begin{enumerate}[label=\alph*.]
	\item Let $E$ be a complement of $W^\Omega$ in $W$, i.e., $W=W^\Omega\oplus E$. Show that the restriction of $\Omega$ to $E$ is nondegenerate.
	\item Let $J$ be a $\Omega$-compatible complex structure, with $g$ the associated inner product. Show that $\Omega$ induces an identification of $J(W^\Omega)=W^\perp$ with $(W^\Omega)^*$. Taking $E$ as the orthogonal complement (with respect to $g$) to $W^\Omega$ in $W$ (this means that $W=W^\Omega\oplus E$), show that the identification
		\[V\cong E\oplus (W^\Omega\oplus (W^\Omega)^*),\]
		is an isomorphism of symplectiv vector spaces (bundles)---on the right-hand-side, $E$ is equipped with its induced symplectic form (see a. above) and $W^\Omega\oplus (W^\Omega)^*$  with its canonical symplectic form.
\end{enumerate}
\end{idea5}

\begin{proof}[Solução]\leavevmode 
\begin{enumerate}[label=\alph*.]
	\item 	Basta ver que $\ker \Omega|_{E}=0$. Se $e\in\ker \Omega|_{E}$, então eu gostaria de ver que $e\in W^\Omega$. Seja $w\in W$. Enão $\Omega(e,w)=\Omega(e,w_1+w_2)$ com $w_1\in W^\Omega$ e $w_2\in E$. Daí $\Omega(e,w)=\Omega(e,w_1)$. Se $e\in W$ acabamos. Se $e \notin W$…

	\item Considere o mapa
		\begin{align*}
			J(W^\Omega) &\longrightarrow (W^\Omega)^* \\
			Jw &\longmapsto i_w\Omega= \Omega(w,\cdot )
		\end{align*}
		Note que $J(W^\Omega)$ e $W^\Omega$ são espaços vetorias de dimensões iguais, e que esse mapa tem kernel trivial pela não degeneração de $\Omega$. Isso explica que é um isomorfismo.


		
\end{enumerate}
	
\end{proof}

\end{document}
