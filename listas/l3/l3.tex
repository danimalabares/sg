\input{/Users/daniel/github/config/preamble.sty}

%\usepackage[style=authortitle-terse,backend=bibtex]{biblatex}
%\addbibresource{bibliography.bib}

\begin{document}

\begin{minipage}{\textwidth}
	\begin{minipage}{1\textwidth}
		Geometria Simpl\'etica \hfill Daniel González Casanova Azuela
		
		{\small Profs. Henrique Bursztyn and Leonardo Macarini\hfill\href{https://github.com/danimalabares/sg}{github.com/danimalabares/sg}}
	\end{minipage}
\end{minipage}\vspace{.2cm}\hrule

\vspace{10pt}
{\huge Lista 3}

\tableofcontents

\addcontentsline{toc}{subsection}{Problem 1}
\paragraph{Problem 1}

\begin{proof}[Solution]\leavevmode
	\begin{enumerate}[label=\alph*.]
		\item Ideia inicial: mostrar que uma base de $T_x^*M$ \'e $df_1,\ldots,df_k,dx_{k+1},\ldots,dx_{2n}$. Da\'i, toda $2n$-forma \'e um m\'ultiplo de $df_1\wedge \ldots \wedge df_k\wedge dx_{k+1}\wedge \ldots\wedge d_{2n}$. Mas esse argumento n\~ao da porque a equa\c c\~ao deve ser v\'alida numa vizinhança de $M_c$.
			
		Segunda ideia: pegue um ponto $p\in\mathcal{U}\subset M$. Podemos completar $(df_1)_p,\ldots,(df_k)_p$ a uma base de $T^*M$. Tamb\'em podemos extender essa base para uma vizinhança $p\in V\subset \mathcal{U}$ como segue: extenda os covectores a toda $M$ usando uma fun\c c\~ao "quindim" (=bump function) que seja 1 num compacto perto de $p$. Da\'i, como os covectores s\~ao linearmente independentes em $p$, o determinante da fun\c c\~ao de coeficentes deles \'e n\~ao zero em $p$, mas como o determinante \'e cont\'inuo, existe uma vizinhança de $p$ onde \'e n\~ao zero. Defina $\tilde{\sigma}$ como o wedge product dos covectores que acabamos de const ruir.

		Essa prova n\~ao t\'a funcionando bem porque s\'o definimos $\sigma$ numa  vizinhança do ponto $p$. Falta construir uma forma definida em todo $\mathcal{U}$. 

		\item Pela f\'ormula de Cartan,
			\begin{align*}
				df_1\wedge \ldots df_k\wedge \mathcal{L}_{X_H}\sigma&=df_1\wedge \ldots df_k\wedge \left(i_{X_H}d\sigma+d(i_{X_H}\sigma)\right)\\
				%&=df_1\wedge \ldots df_k\wedge d(i_{X_{H}})\sigma\\
				&=
			\end{align*}
		
	\end{enumerate}
\end{proof}

\addcontentsline{toc}{subsection}{Problem 2}
\paragraph{Problem 2} Let $M$ be a symplectic manifold, $\Psi=(\psi^1,\ldots,\psi^k):M\to \mathbb{R}^{k}$ a smooth map, and $c$ a regular value. Consider a submanifold $N=\Psi^{-1}(c)\hookrightarrow M$.
\begin{enumerate}[label=\alph*.]
	\item Show that $N$ is coisotropic if and only if $\{\psi^i,\psi^j\}|_{N}=0$ for all $i,j=1,\ldots,k$.
	
	\item Show that $N$ is symplectic if and only if the matrix $(c^{ij})$, with $c^{ij}=\{\psi^i,\psi^j\}$, is invertible for all $x\in N$. In this case, verify that we have the following expression for the Poisson bracket $\{\cdot,\cdot\}_N$ on $N$ (known as \textit{\textbf{Dirac's bracket}}):
	\[\{f,g\}_N=\left.\left(\{\tilde{f},\tilde{g}\} =\sum_{ij}\{\tilde{f},\tilde{g}\} c_{ij}\{\psi^j,\tilde{g}\}\right)\right|_{N}\]
		where $(c_{ij})=(c^{ij })^{-1}$, $f,g\in \mathcal{C}^\infty(N)$, and $\tilde{f},\tilde{g}\in\mathcal{C}^\infty(M)$ are arbitrary extensions of $f$ and $g$.
\end{enumerate}

\begin{proof}[Solution]\leavevmode
	\begin{enumerate}[label=\alph*.]
		\item Since $M$ is symplectic we have a bundle isomorphism
\begin{align*}
	\omega^\flat: TM &\longrightarrow T^*M \\
	v &\longmapsto i_v\omega
\end{align*}
Then 
\[T N^\omega=(\omega^\flat )^{-1}(\operatorname{Ann}(T N)).\]
Since $M$ is the level set of a regular value, there are local coordinates of the form $(\psi^1,\ldots,\psi^k,x^{k+1},\ldots,x^{2n})$. Vectors tangent to $N$ are expressed only in terms of the vectors $\partial_{k+1},\ldots,\partial_{2n}$ and thus covectors that vanish on $T N$ are those which vanish on $\partial_{k+1},\ldots, \partial_{2n}$. This means that a basis for $\operatorname{Ann}(T N)$ is given by $d\psi^1,\ldots,d\psi^k$ (an explanation of this might be that the canonical basic covectors for the coordinates $\psi^i$ are the differentials $d\psi^i$, which maybe can be checked using the usual change of coordinates formula). These generators map to their hamiltonian vector fields under $(\omega^\flat)^{-1}$:
\[\left(\omega^\flat \right)^{-1}(d\psi^i)=X_{\psi^i}\]
So $T N^\omega$ is generated by the $X_{\psi^i}$. Notice that any vector $v \in TM$ is actually in $T N$ iff $\alpha(v)=0\;\forall \alpha\in\operatorname{Ann}(T N) $. Then we see that
\begin{align*}
	T N^\omega\subset TM&\iff X_{\psi^i}\in T N\quad i=1,\ldots,k\\
	&\iff d\psi^j(X_{\psi^i})|_{N}=0 \quad i,j=1,\ldots,k\\
	&\iff \omega(X_{\psi^i},X_{\psi^j})|_{N}=0\quad i,j=1,\ldots,k\\
	&\iff \{\psi^i,\psi^j\}|_{N}=0\quad i,j=1,\ldots,k
\end{align*}

\item Em qualquer sistema de coordenadas $(x^1,\ldots,x^{2n})$, os vetores hamilatonianos $X_{x^1},\ldots,X_{x^{2n}}$ formam uma base do espaço tangente. Nessa base, os coeficientes da matriz que representa $\omega$ s\~ao exatamente os colchetes de Poisson  $\{x^i,x^j\}$. A forma $\omega$ \'e n\~ao degenerada se e somente se a sua matriz \'e invert\'ivel (j\'a que nesse caso temos o isomofismo $\omega^\flat$ bem definido). Então, o que queremos \'e ver que a restri\c c\~ao $\omega|_{N}$ \'e invert\'ivel em cada ponto de $N$.

	Nas coordenadas de subvariedade $(\psi^1,\ldots,\psi^k,x^{k+1},\ldots,x^{2n})$, a matrix que representa $\omega^\flat$ \'e
	\[\begin{pmatrix} \{\psi^1,\psi^1\} &\cdots &\{\psi^k,\psi^1\} &\{x^{k+1},\psi^1\} &\cdots &\{x^{2n},\psi^1\}\\
	\vdots &&\vdots &&&\vdots\\
\{\psi^1,\psi^k\} &\cdots &\{\psi^k,\psi^k\} &\{x^{k+1},\psi^k\} &\cdots &\{x^{2n},\psi^k\}\\
	\{\psi^1,x^{k+1}\} &\cdots &\{\psi^k,x^{k+1}\} &\{x^{k+1},x^{k+1}\} &\cdots &\{x^{2n},x^{k+1}\} \\
	\vdots &&\vdots &&&\vdots\\
	\{\psi^1,x^{2n}\} &\cdots &\{\psi^k,x^{2n}\} &\{x^{k+1},x^{2n}\} &\cdots &\{x^{2n},x^{2n}\}
\end{pmatrix}\]
No entanto,
\begin{align*}
	\{x^i,\psi^j\}&=dx^i(X_{\psi^j})=0\\
\{\psi^j,x^i\} &=d\psi^j(X_{x^i})=0
\end{align*}
se os covetores $dx^i$ e $d\psi^j$ s\~ao de fato a base dual de $X_{x^i}$, $X_{\psi^j}$. Se isso for certo, podemos escrever a matriz representada acima como
\[\begin{pmatrix}\begin{array}{c|c}
	\{\psi^i,\psi^j\}&0\\ \hline
	0&\{x^i,x^j\} 
\end{array}  \end{pmatrix}, \]
cujo determinante \'e o produto dos determinantes das matrizes de bloco n\~ao zero. Infelizmente o determinante da matriz $\{x^i,x^j\}$ pode ser zero se $k=n$ e tomamos uma carta coordenada de Darboux…

Suponha por enquanto que dadas as proje\c c\~oes
\[\begin{tikzcd}
&TM|_{N}=T N\oplus T N^\omega\arrow[dl,"q_1",swap]\arrow[dr,"q_2"]\\
T N&&T N^{\omega}
\end{tikzcd}\]
\'e verdade que
\[X_f=q_1(X_{\tilde{f}}),\qquad q_2(Y)=\sum_{i,j}d\psi^i(Y)c_{ij}X_{\psi^j}.\]
Ent\~ao, (para facilitar leitura n\~ao escrevemos a resitri\c c\~ao a $N$, mas isso s\'o tem sentido em $N$ )
\begin{align*}
	\{\tilde{f},\tilde{g}\}&=\omega(X_{\tilde{f}},X_{\tilde{g}})\\
			       &=\omega(q_1(X_{\tilde{f}}),q_1(X_{\tilde{g}}))+\omega(q_1(X_{\tilde{f}}),q_2(X_{\tilde{g}}))\\
	&+\omega(q_2(X_{\tilde{f}}),q_1(X_{\tilde{g}}))+\omega(q_2(X_{\tilde{f}}),q_2(X_{\tilde{g}}))\\
	&=\omega(X_{f},X_{g})+\omega(q_2(X_{\tilde{f}}),q_2(X_{\tilde{g}}))\\
	&=\{f,g\}_{N}+\omega(q_2(X_{\tilde{f}}),q_2(X_{\tilde{g}}))
\end{align*}
	\end{enumerate}
	para calcular o \'ultimo termo notamos que
	\begin{align*}
		\begin{aligned}
		q_2(X_{\tilde{f}})&=\sum_{i,j}d\psi^i(X_{\tilde{f}})c_{ij}X_{\psi^j}\\
		&=\sum_{ij}-\{\tilde{f},\psi^i\} c_{ij}X_{\psi^j}
	\end{aligned}\qquad\quad \quad 
	\begin{aligned}
		q_2(X_{\tilde{g}})&=\sum_{k,\ell}d\psi^k(X_{\tilde{g}})c_{k \ell}X_{\psi^\ell}\\
		&=\sum_{k,\ell}\{\psi^k,\tilde{g}\} c_{k \ell}X_{\psi^\ell}
	\end{aligned}
	\end{align*}
	de modo que
	\begin{align*}
		\omega(q_2(X_{\tilde{f}}),q_2(X_{\tilde{g}}))&=\sum_{i,j,k,\ell} -\{\tilde{f},\psi^i\} c_{ij}\{\psi^k,\tilde{g}\} c_{k \ell}\omega(X_{\psi^j},X_{\psi^\ell})\\
	&=\sum_{i,j,k,\ell} -\{\tilde{f},\psi^i\} c_{ij}c^{j\ell}c_{k \ell}\{\psi^k,\tilde{g}\}\\
	&=\sum_{i,j,k,\ell} \{\tilde{f},\psi^i\} c_{ij}c^{j\ell}c_{\ell k}\{\psi^k,\tilde{g}\}\\
	&=\sum_{i,k} \{\tilde{f},\psi^i\} c_{i k}\{\psi^k,\tilde{g}\}
	\end{align*}
	usando que a $c_{\ell k}=-c_{k\ell}$ por ser uma matriz antisim\'etrica. Com isso seria demonstrado o exerc\'icio.

	O fato de que $X_f=q_1(X_{\tilde{f}})$ segue de que tanto $M$ quanto $N$ s\~ao variedades simpl\'eticas, de modo que existe um \'unico campo vetorial associado à  $df=d\tilde{f}|_{N}$ em $N$.

A equa\c c\~ao $q_2(Y)=\sum_{i,j}d\psi^i(Y)c_{ij}X_{\psi^j}$ pode ser explicada como segue. Primeiro considere o caso simples do vetor $\frac{\partial }{\partial x^i}\in TM|_{N}$ para $i\leq k$ fixa. A mudança de coordenadas $\Psi\times \operatorname{id}_{2n-k}=(\psi^1,\ldots,\psi^k,x^{k+1},\ldots,x^{2n})$ disse que
\[\frac{\partial }{\partial x^i}=\sum_{j}\frac{\partial \psi^j}{\partial x^i}V^j\]
onde $V^j$ \'e o marco de campos vetorias associado as novas coordenadas. Qual \'e esse marco? Sabemos que uma base $k$ kk


pegue um vetor tangente a $M$ ancorado sobre $N$. Podemos expressá-lo em coordenadas locais:
\[TM|_{N}\ni Y=\sum_{i=1}^kY^i\frac{\partial }{\partial x_i}+\sum_{i=k+1}^{2n}Y^i\frac{\partial }{\partial x^i}\]
Da\'i, \begin{align*}
	Y&=\sum_{i=1}^kY^i\sum_{j=1}^k\frac{\partial \psi^j}{\partial x^i}\frac{\partial }{\partial }
\end{align*}
\end{proof}

\addcontentsline{toc}{subsection}{Problem 3}
\paragraph{Problem 3}

\begin{proof}[Solution]\leavevmode
	By an analogue argument to Problem 2a we know that $\operatorname{Ann}(D)$ is a coisotropic submanifold iff $\{f,g\} =0$ for all $f,g\in I_{\operatorname{Ann}(D)}$. Now a vector field $X$ corresponds naturally to an element of the double dual $\xi \in T^*(T^*(M))$ given by $\xi\eta(X)=\eta(X)$ for  $\eta\in T^*M$. Notice that if $X \in D$ then $\xi \in \operatorname{Ann}(\operatorname{Ann}(D) ) $.
\end{proof}

\addcontentsline{toc}{subsection}{Problem 4}
\paragraph{Problem 4} Consider a smooth map $\phi:Q_1\to Q_2$, and let
\[R_{\phi}:=\left\{ \left( (x,\xi),(y,\eta) \right) :y=\phi(x),\;\xi=((T\phi)^*\eta \right\} \subset T^*Q_1\times T^*Q_2.\]
Verify that $R_\phi$ is a lagrangian submanifold of $T^* Q_1\times \overline{T^*Q_2}$. Whenever $\phi$ is a diffeo, what is the relation between $R_\phi$ and the cotangent lift $\hat{\phi}$?

Denote by $\Gamma_\phi\subset Q_1\times Q_2$ the graph of $\phi$. What is the relation between $N^*T_\phi$ (the conormal bundle of $\Gamma_\phi$) and  $R_\phi$?

\begin{proof}[Solution]\leavevmode
	Parece que o pullback $(T\phi)^*$ coincide com o pullback usual  $\phi^*$, pois o primeiro \'e s\'o composi\c c\~ao de fun\c c\~oes enquanto o segundo \'e composi\c c\~ao com a diferencial de $\phi$.

	Note que $R_\phi$ \'e o produto cartesiano dos grafos de $\phi$ e do seu pullback $\phi^*$ :
	\begin{align*}
		\Gamma_\phi&=\{(x,\phi(x)):x\in Q_1\} \\
		\Gamma_{\phi^*}&=\{\left( \phi^*\eta,\eta \right) :\eta\in T^*Q_2\}
	\end{align*}
	esses dois s\~ao variedades suaves de dimens\~oes
	\begin{align*}
		\dim \Gamma_\phi&=\dim Q_1\\
		\dim \Gamma_{\phi^*}&=(\dim Q_2)^2
	\end{align*}

\end{proof}

\end{document}
