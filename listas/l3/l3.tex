\input{/Users/daniel/github/config/preamble.sty}

%\usepackage[style=authortitle-terse,backend=bibtex]{biblatex}
%\addbibresource{bibliography.bib}

\begin{document}

\begin{minipage}{\textwidth}
	\begin{minipage}{1\textwidth}
		Geometria Simpl\'etica \hfill Daniel González Casanova Azuela
		
		{\small Profs. Henrique Bursztyn and Leonardo Macarini\hfill\href{https://github.com/danimalabares/sg}{github.com/danimalabares/sg}}
	\end{minipage}
\end{minipage}\vspace{.2cm}\hrule

\vspace{10pt}
{\huge Lista 3}

\tableofcontents

\addcontentsline{toc}{subsection}{Problem 1}
\paragraph{Problem 1}

\begin{proof}[Solution]\leavevmode
	\begin{enumerate}[label=\alph*.]
		\item {\color{magenta}Note} that a basis of $T_x^*M$ is $df_1,\ldots,df_k,x_{k+1},\ldots,x_{2n}$. Then any $2n$-form is a multiple of  $df_1\wedge \ldots \wedge df_k\wedge dx_{k+1}\wedge \ldots\wedge d_{2n}$. 
	\end{enumerate}
\end{proof}

\addcontentsline{toc}{subsection}{Problem 2}
\paragraph{Problem 2} Let $M$ be a symplectic manifold, $\Psi=(\psi^1,\ldots,\psi^k):M\to \mathbb{R}^{k}$ a smooth map, and $c$ a regular value. Consider a submanifold $N=\Psi^{-1}(c)\hookrightarrow M$.
\begin{enumerate}[label=\alph*.]
	\item Show that $N$ is coisotropic if and only if $\{\psi^i,\psi^j\}|_{N}=0$ for all $i,j=1,\ldots,k$.
	
	\item Show that $N$ is symplect if and only if the matrix $(c^{ij})$, with $c^{ij}=\{\psi^i,\psi^j\}$, is invertible for all $x\in N$. In this case, verify that we have the following expression for the Poisson bracket $\{\cdot,\cdot\}_N$ on $N$ (known as \textit{\textbf{Dirac's bracket}}):
	\[\{f,g\}_N=\left.\left(\{\tilde{f},\tilde{g}\} =\sum_{ij}\{\tilde{f},\tilde{g}\} c_{ij}\{\psi^j,\tilde{g}\}\right)\right|_{N}\]
\end{enumerate}

\begin{proof}[Solution]\leavevmode
	\begin{enumerate}[label=\alph*.]
		\item Since $M$ is symplectic we have a bundle isomorphism
\begin{align*}
	\omega^\flat: TM &\longrightarrow T^*M \\
	v &\longmapsto i_v\omega
\end{align*}
Then 
\[T N^\omega=(\omega^\flat )^{-1}(\operatorname{Ann}(T N)).\]
Since $M$ is the level set of a regular value, there are local coordinates of the form $(\psi^1,\ldots,\psi^k,x^{k+1},\ldots,x^{2n})$. Vectors tangent to $N$ are expressed only in terms of the vectors $\partial_{k+1},\ldots,\partial_{2n}$ and thus covectors that vanish on $T N$ are those which vanish on $\partial_{k+1},\ldots, \partial_{2n}$. This means that a basis for $\operatorname{Ann}(T N)$ is given by $d\psi^1,\ldots,d\psi^k$ (indeed, the canonical basic covectors for the coordinates $\psi^i$ are the differentials $d\psi^i$---this can be checked using a change of coordinates matrix). These generators map to their hamiltonian vector fields under $(\omega^\flat)^{-1}$:
\[\left(\omega^\flat \right)^{-1}(d\psi^i)=X_{\psi^i}\]
So $T N^\omega$ is generated by the $X_{\psi^i}$. Notice that any vector $v \in TM$ is actually in $T N$ iff $\alpha(v)=0\;\forall \alpha\in\operatorname{Ann}(T N) $. Then we see that
\begin{align*}
	T N^\omega\subset TM&\iff X_{\psi^i}\in T N\quad i=1,\ldots,k\\
	&\iff d\psi^j(X_{\psi^i})|_{N}=0 \quad i,j=1,\ldots,k\\
	&\iff \omega(X_{\psi^i},X_{\psi^j})|_{N}=0\quad i,j=1,\ldots,k\\
	&\iff \{\psi^i,\psi^j\}|_{N}=0\quad i,j=1,\ldots,k
\end{align*}

\item The matrix $(c^{ij})$ determines the bundle isomorphism $\omega^\sharp:$
	\end{enumerate}
\end{proof}

\addcontentsline{toc}{subsection}{Problem 3}
\paragraph{Problem 3}

\begin{proof}[Solution]\leavevmode
	By an analogue argument to Problem 2a we know that $\operatorname{Ann}(D)$ is a coisotropic submanifold iff $\{f,g\} =0$ for all $f,g\in I_{\operatorname{Ann}(D)}$. Now a vector field $X$ corresponds naturally to an element of the double dual $\xi \in T^*(T^*(M))$ given by $\xi\eta(X)=\eta(X)$ for  $\eta\in T^*M$. Notice that if $X \in D$ then $\xi \in \operatorname{Ann}(\operatorname{Ann}(D) ) $.
\end{proof}

\end{document}
