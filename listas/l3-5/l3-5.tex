\input{/Users/daniel/github/config/preamble.sty}%This is available at github.com/danimalabares/config

%\usepackage[style=authortitle-terse,backend=bibtex]{biblatex}
%\addbibresource{bibliography.bib}

\begin{document}

\begin{minipage}{\textwidth}
	\begin{minipage}{1\textwidth}
		Geometria Simpl\'etica \hfill Daniel González Casanova Azuela
		
		{\small Profs. Henrique Bursztyn e Leonardo Macarini\hfill\href{https://github.com/danimalabares/sg}{github.com/danimalabares/sg}}
	\end{minipage}
\end{minipage}\vspace{.2cm}\hrule

\vspace{10pt}
{\huge Lista 3}

\addcontentsline{toc}{subsection}{Problem 5}
\paragraph{Problem 5} Let $M$ be a manifold and $\omega\in\Omega^{k}(M)$. Suppose that $\pi:M\to B$ is a surjective submersion with connected fibers. We say that $\omega$ is \textit{\textbf{basic}} (with respect to $\pi$) if there exists a form $\overline{\omega} \in \Omega^{k}(B)$ such that $\pi^*\overline{\omega} =\omega$.
\begin{enumerate}[label=\alph*.]
	\item Show that $\omega$ is basic iff $i_X\omega=0$ and $\mathcal{L}_{X}\omega=0$ for all vector fields $X$ tangent to the fibers of $\pi$. In particular, if $\omega$ is closed, show that it is basic if $\ker(T\pi)\subseteq\ker \omega$ (pointwise in $M$ ).

		\item Suppose that $\omega$ is a closed 2-form on $M$ and $\ker(T\pi)=\ker \omega$. Show that $\omega=\pi^*\overline{\omega}$ and $\overline{\omega}\in\Omega^{2}(B)$ is symplectic.

		\item 
\end{enumerate}

\begin{proof}[Solution]\leavevmode
\begin{enumerate}[label=\alph*.]
	\item Primeiro note que se $X$ é tangente às fibras de $\pi$, a projeção de $X$ é zero já que o espaço tangente a um ponto é vazio. Daí a implicação $\implies $ é imediata.

	Para $\impliedby$ vamos provar primeiro localmente


	(Ver \href{https://math.stackexchange.com/questions/69658/basic-differential-forms}{StackExchange}) Para  $\impliedby$ defina uma forma $\overline{\omega}\in\Omega^{k}(B)$ usando que a derivada de $\pi$ é surjetiva em todo ponto de $M$ da seguinte forma:
	\[\overline{\omega}(\pi_*X_1,\ldots,\pi_*X_k):=\omega(X_1,\ldots,X_1)\]
Devemos mostrar que $\overline{\omega}$ está bem definida.

\begin{itemize}
\item {\color{persimmon}$\pi_*X_1=\pi_*X_1'\implies \omega(X_1,X_2,\ldots,X_k)=\omega(X_1',X_2,\ldots,X_k)$.}

	Isso segue de que 
\end{itemize}

\end{enumerate}	
\end{proof}

\end{document}
