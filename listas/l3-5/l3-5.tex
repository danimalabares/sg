\input{/Users/daniel/github/config/preamble.sty}%This is available at github.com/danimalabares/config

%\usepackage[style=authortitle-terse,backend=bibtex]{biblatex}
%\addbibresource{bibliography.bib}

\begin{document}

\begin{minipage}{\textwidth}
	\begin{minipage}{1\textwidth}
		Geometria Simpl\'etica \hfill Daniel González Casanova Azuela
		
		{\small Profs. Henrique Bursztyn e Leonardo Macarini\hfill\href{https://github.com/danimalabares/sg}{github.com/danimalabares/sg}}
	\end{minipage}
\end{minipage}\vspace{.2cm}\hrule

\vspace{10pt}
{\huge Lista 3}

\addcontentsline{toc}{subsection}{Problem 5}
\paragraph{Problem 5} Let $M$ be a manifold and $\omega\in\Omega^{k}(M)$. Suppose that $\pi:M\to B$ is a surjective submersion with connected fibers. 
We say that $\omega$ is \textit{\textbf{basic}} (with respect to $\pi$) if there exists a form $\overline{\omega} \in \Omega^{k}(B)$ such that $\pi^*\overline{\omega} =\omega$.
\begin{enumerate}[label=\alph*.]
	\item Show that $\omega$ is basic iff $i_X\omega=0$ and $\mathcal{L}_{X}\omega=0$ for all vector fields $X$ tangent to the fibers of $\pi$. In particular, if $\omega$ is closed, show that it is basic if $\ker(T\pi)\subseteq\ker \omega$ (pointwise in $M$ ).

		\item Suppose that $\omega$ is a closed 2-form on $M$ and $\ker(T\pi)=\ker \omega$. Show that $\omega=\pi^*\overline{\omega}$ and $\overline{\omega}\in\Omega^{2}(B)$ is symplectic.

		\item {\color{6}(Application to reduction.)}\hspace{0.5em}Let $(M,\omega)$ be a symplectic manifold and $\iota:N\hookrightarrow M$ a submanifold such that $D=TN\cap T N^\omega\subset T N$ has constant rank (e.g. $N$ could be coisotropic). We saw in class that $D$ is an integrable distribution (by Frobenius); suppose that the leafspace $B:=N/\sim$ is smooth so that the natural projection $\pi:N\longrightarrow B$ is a submersion. Show that $B$ inherits a unique symplectic form $\omega_{\operatorname{red}}$ with the propery that $\pi^*\omega_{\operatorname{red}}=\iota ^*\omega$
\end{enumerate}

\begin{proof}[Solution]\leavevmode
\begin{enumerate}[label=\alph*.]
	\item Primeiro note que se $X$ é tangente às fibras de $\pi$, o pushforward dele baixo $\pi$ é zero já que o espaço tangente a um ponto é trivial (podemos ver $X$ como um campo em cada fibra, que é uma subvariedade, e a projeção manda ele no vetor zero na base). Daí a implicação $\implies $ é imediata.

	Para $\impliedby$ vamos provar primeiro localmente

	(Ver \href{https://math.stackexchange.com/questions/69658/basic-differential-forms}{StackExchange}) Para $\impliedby$ o mais natural é definir uma forma em $B$ como
	\[\overline{\omega}(\pi_*X_1,\ldots,\pi_*X_k):=\omega(X_1,\ldots,X_1)\]
já que assim $\pi^*\overline{\omega}=\omega$. Mais não é imediato para mim que isso faz sentido, pois devo comprovar todo campo vetorial em $B$ pode ser visto como o pushforward de um campo vetorial em 

Devemos mostrar que $\overline{\omega}$ está bem definida.

Agora suponha que $\omega$ é fechada e que $\ker \pi_*\subseteq \ker \omega$. Pegue $X$ tangente às fibras de $\pi$; vimos acima que $\pi_*X=0$, então $X\in\ker \omega$, i.e. $i_X\omega=0$ e também $0=\mathcal{L}_{X}\omega =di_X\omega+i_Xd\omega$.

\begin{itemize}
\item {\color{4}$\pi_*X_1=\pi_*X_1'\implies \omega(X_1,X_2,\ldots,X_k)=\omega(X_1',X_2,\ldots,X_k)$.}

	Isso segue de que 
\end{itemize}

\item Usando o item anterior, basta mostrar que $i_X\omega =0=\mathcal{L}_{X}\omega$ para todo $X$ tangente às fibras de $\pi$. Mas, se $X$ é tangente às fibras de $\pi$, ele tá no $\ker \pi_*=\ker \omega$. Daí, $i_X\omega=0$ e também $0=\mathcal{L}_{X}\omega=di_X\omega+i_Xd\omega$. Para ver que $\overline{\omega}$ é simplética lembre que $\ker \overline{\omega}=\{v\in TM: i_{v}\overline{\omega}=0\}$, logo se $v\in\ker \overline{\omega}$ sabemos que existe $u\in TM$ tal que $\pi_*u=v$, e daí $i_u\omega=i_u\pi^*\overline{\omega}=\overline{\omega}(v,\cdot )=i_v\overline{\omega}=0$. Isso mostra que $u\in\ker \omega=\ker \pi_*\omega\implies \pi_*u=v=0$.

	\item De acordo com o inciso b., basta ver que $\ker \pi_*=\ker \omega|_{N}$ (já que $\omega|_{N}$ é uma forma fechada, pois é o pullback de $\omega$ baixo a inclusão).Como $TN\cap TN^\omega$ é uma distribuição intergável, por cada ponto de $N$ pasa uma folha de uma folheação. Pegue $V$ tangente às folhas da distribuição, de modo que  $\pi_*V=0$ já que as folhas são pontos em $B$. Mas ainda, por definição dessa distribuição, que $V$ seja tangente às folhas significa que $V\in TN\cap TN^\omega$. Mas já sabemos que $TN\cap TN^{\omega}$ é o kernel de $\omega|_{N}$.

		{\color{5}…Mas talvez existe outro campo vetorial $V\in\ker\pi_*$ que não é tangente às folhas da distribuição. Nesse caso $V$  não pode estar em $TN^\omega$, …}

		Pegue $V\in\ker \omega |_{N}=TN\cap TN^\omega$, então $V$ é tangente às fibras da distribuição e portanto está em $\ker \pi_*$.

		Com isso, usando o inciso b., sabemos que existe uma forma  $\overline{\omega}:=\omega_{\operatorname{red}}$ tal que $\pi^*\overline{\omega}=\omega|_{N}=\iota^*\omega$.

\end{enumerate}	
\end{proof}

\end{document}
