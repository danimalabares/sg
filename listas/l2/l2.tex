\input{/Users/daniel/github/config/preamble-por.sty}
\input{/Users/daniel/github/config/thms-por.sty}

\usepackage[style=authortitle-terse,backend=bibtex]{biblatex}
\addbibresource{bibliography.bib}

\begin{document}

\begin{minipage}{\textwidth}
	\begin{minipage}{1\textwidth}
		Geometria Simpl\'etica \hfill Daniel González Casanova Azuela
		
		{\small Profs. Henrique Bursztyn and Leonardo Macarini\hfill\href{https://github.com/danimalabares/sg}{github.com/danimalabares/sg}}
	\end{minipage}
\end{minipage}\vspace{.2cm}\hrule

\vspace{10pt}
{\huge Lista 2}

\paragraph{Problem 1} Verify (and justify) wether or not the following manifolds admit a symplectic structure: $S^{1}\times S^3$, $\mathbb{R}\times S^3$ $\mathbb{R}^{3}\times S^{3}$, $\mathbb{T}^3\times S^3$.

\begin{proof}[Solution]\leavevmode
	\begin{enumerate}[label=\alph*.]
		\item 

		\item Considere o mapa
			\begin{align*}
				(0,+\infty)\times \mathbb{S}^3 &\longrightarrow \mathbb{R}^4 \\
				(t,v) &\longmapsto tv
			\end{align*}
			com inversa $x\mapsto \left( |x|,\frac{x}{|x|} \right)$. Ele \'e um difeomorfismo, assim o pullback da forma simpl\'etica can\'onica em $\mathbb{R}^{4}$ induz uma estrutura simpl\'etica em $\mathbb{R}\times S^3\overset{\operatorname{dif}}{\cong} (0,+\infty)\times S^3$.

		\item Considerando que $S^3$ \'e a esfera unit\'aria nos quat\'ernios, os campos vetoriais $z\mapsto iz$, $z\mapsto jz$ e $z\mapsto kz$ s\~ao uma base global, de forma que $S^3$ \'e paraleliz\'avel. Sendo uma 3-variedade, temos que $TS^3\cong \mathbb{R}^{3}\times S^3$. Em aula vimos que todo fibrado tangente \'e uma variedade simpl\'etica usando o pullback da forma can\'onica no fibrado cotangente.

		\item 
	\end{enumerate}
\end{proof}

\paragraph{Problem 2} Show that the tautological 1-form $\Omega^{1}(T^*Q)$ is \textit{uniequely characterized} by the following property: for any 1-form $\mu\in\Omega^{1}(Q)$,
\[\mu^*\alpha=\mu\]
where on the left-hand side we view  $\mu$ as a map $\mu:Q\to T^*Q$.

\begin{proof}[Solution]\leavevmode
	(After several tries, consulted \href{https://math.stackexchange.com/questions/4141107/a-characterization-of-the-tautological-form-liouville-1-form-on-the-cotangent}{StackExchange}) Simply notice that if $\beta$ is another 1-form on  $T^*Q$ such that $\mu^*\beta=\mu$ for all $\mu\in\Omega^{1}(Q)$, then  $\mu^*(\alpha-\beta)=0$. Further, if any  $\theta\in\Omega^{1}(T^*Q)$ satisfies $\mu^*\theta=0$ for all $\mu\in\Omega^{1}(Q)$, it must be identiacally zero. This follows since for any $w\in T(T^*Q)$ we can find a form $\mu$ and a vector $v$ such that $\mu_*v=w$.
\end{proof}


\begin{proof}[Solution]\leavevmode
	Temos que se $v\in T_x(Q)$ \'e um vetor tangente a $Q$ no ponto $x\in Q$,
	\begin{align*}
		(\mu^*\alpha)_xv&=\alpha_{\mu_x}(\mu_*v)
	\end{align*}
	onde $\mu_*v\in T T^*Q$, pois $\mu_*:TQ\to  T T^*Q$. Lembre que $\alpha_\eta=\pi^*_{\eta}\eta$ para qualquer $\eta\in\Omega^{1}(Q)$. Assim, podemos escrever
	\begin{align*}
		\alpha_{\mu_x}(\mu_*v)&=\pi_{\mu_x}^*(\mu_{*}v)\\
		& =\mu_x(\pi_{*}\mu_*v)\\
		&=\mu_x(v)
	\end{align*}
	j\'a que $\mu$ comuta com a proje\c c\~ao.
\end{proof}

\paragraph{Problem 3} 

\begin{proof}[Solution]\leavevmode
	\begin{enumerate}[label=\alph*.]
		\item Seguindo a nota\c c\~ao do problema anterior, note que
			\begin{align*}
			(F^*\alpha)v&=-\alpha(F_*v)=-\pi^*_{\xi_{F_*v}}\xi_{F_*v}(F_*v)=\xi_{F_*v}(\pi_*F_*v)
			\end{align*}
	\end{enumerate}
\end{proof}

\paragraph{Problem 6} Let $\omega\in\Omega^2(M)$ be a nondegenerate 2-form. For $f\in C^\infty(M)$, let $X_f\in\mathfrak{X}(M)$ be defined by $i_{X_{f}}\omega=df$. Consider the bracket $\{f,g\}:=\omega(X_g,X_f)$. Verify that $d\omega=0$ if and only if $\{\cdot ,\cdot \}$ satisfies the Jacobi identity.

\begin{proof}[Solution]\leavevmode
	(Taken from \href{https://math.stackexchange.com/questions/1692891/how-to-show-that-jacobi-identity-for-is-equivalent-to-omega-being-clo}{StackExhange}). Using that $X_{\{g,h\}} =-[X_g,X_h]$ (Prop 22.19, \cite{lee} ), we may write the Jacobi identity as
	\begin{align*}
		0&=\{f,\{g,h\}\}+\{h,\{f,g\}\} +\{g,\{h,f\}\} \\
		 &=\omega(X_f,[X_h,X_g])+\omega(X_h,[X_g,X_f])+\omega(X_g,[X_f,X_f])
	\end{align*}
	Next we use the coordinate-free expression of the exterior derivative of a 3-form to write
	\begin{align*}
		d\omega(X_f,X_g,X_h)=&X_f\omega(X_g,X_h)-X_g\omega(X_f,X_h)+X_h\omega(X_f,X_g)\\
				     &-\omega([X_f,X_g],X_h)+\omega([X_f,X_h],X_g)-\omega([X_g,X_h],X_f)
	\end{align*}
	Finally we use that $X_f\omega(X_g,X_h)=\omega(X_f,[X_g,X_h])$ ({\color{magenta}why?)} to see that, in the last equation, the first and second row on the right hand side are actually the same, and each of them equals the expression of the Jacobi identity.
\end{proof}

\paragraph{Problem 7} Consider symplectic manifolds $(M_i,\omega_i)$, with Poisson bracket $\{\cdot ,\cdot \}_{i}$, $i=1,2$, and let  $\phi:M_1\to M_2$ be a smooth map.
\begin{enumerate}[label=\alph*.]
	\item Prove that if $\phi$is a diffeomorphism, then it is a Poisson map ($\{\phi^*f,\phi^*g\}_1=\phi^* (\{f,g\}_2)$ for all $f,g\in C^\infty(M_2)$) if and only if $\phi^*\omega_2=\omega_1$.

	\item Find examples of $M_1$, $M_2$ and $\phi:M_1\to M_2$ such that (1) $\phi$ is not a Poisson map but soes not satisfy $\phi^*\omega_2=\phi_1$, (2) $\phi$ satisfies $\phi^*\omega_2=\omega_1$ but it is not a Poisson map.
\end{enumerate}

\begin{proof}[Solution]\leavevmode
	\begin{enumerate}[label=\alph*.]
		\item Suppose that $\omega_1=\phi^*\omega_2$. Then
			\begin{align*}
				\{\phi^*f,\phi^*g\} & =\omega_1(X_{\phi^*f},X_{\phi^*g})\\
						    &=\phi^* \omega_2(X_{\phi^*f},X_{\phi^*g})\\
						    & =\omega_2(\phi_*X_{\phi^*f},\phi_{*} X_{\phi^*g})\\
						    & =\omega_2(X_f,X_g)\\
						    & =\{f,g\}
			\end{align*}
	For the converse I would li

	\item It is clear that $\phi_1:(q_1,p_1,q_2,p_2)\mapsto (q_1,p_1)$ does not satisfy $\phi_1^*\omega_{\mathbb{R}^{2}}=\omega_{\mathbb{R}^{4}}$ since 
		\[\omega_{\mathbb{R}^{4}}((0,1,0,1),(0,1,0,-1))=2\]
		but $(\phi_1)_{*}(0,1,0,1)=(0,1)=(\phi_1)_*(0,1,0,-1)$ giving 
		\[\phi_1^*\omega_{\mathbb{R}^{2}}((0,1,0,1),(0,1,0,1))=\omega_{\mathbb{R}^{2}}((0,1),(0,1))=0.\]
		However we can see that $\phi_1$ is a Poisson map since for any $f,g\in\mathcal{C}^\infty(\mathbb{R}^{2})$, at $x\in\mathbb{R}^{4}$ we have
		\[\phi_1^*\{f,g\}_{\mathbb{R}^{2}}(x)=\left( \sum_{i=1}^2\frac{\partial f}{\partial x_i}\frac{\partial g}{y_i}-\frac{\partial f}{\partial y_i}\frac{\partial g}{\partial x_i} \right)_{\phi_1(x)} \]
and
\[\{\phi_1^*f,\phi_1^*g\}_{\mathbb{R}^{4}} =\left( \sum_{i=1}^2\frac{\partial f\circ \phi_1}{\partial x_i}\frac{\partial g\circ \phi_1}{y_i}-\frac{\partial f\circ \phi_1}{\partial y_i}\frac{\partial g\circ \phi_1}{\partial x_i} \right)_x\]
and they are equal by the chain rule and the fact that the derivative of $\phi_1$ does not alter the first two coordinates of a vector while vanishing the last two.


		It is even more clear that $\phi_2:(q_1,p_2)\mapsto (q_1,p_1,0,0)$ satisfies $\phi_2\omega_{\mathbb{R}^{4}}=\omega_{\mathbb{R}^{2}}$. However, in this case we have
		\[\phi_2^*\{f,g\}_{\mathbb{R}^{4}}(x)=\left( \sum_{i=1}^4\frac{\partial f}{\partial x_i}\frac{\partial g}{y_i}-\frac{\partial f}{\partial y_i}\frac{\partial g}{\partial x_i} \right)_{\phi_2(x)} \]
		while
\[\{\phi_2^*f,\phi_2^*g\}_{\mathbb{R}^{2}} =\left( \sum_{i=1}^2\frac{\partial f\circ \phi_2}{\partial x_i}\frac{\partial g\circ \phi_2}{y_i}-\frac{\partial f\circ \phi_2}{\partial y_i}\frac{\partial g\circ \phi_2}{\partial x_i} \right)_x\]
	\end{enumerate}
	meaning that the information of the last two coordinates is lost in the second expression, so they cannot be equal.
	\end{proof}

\paragraph{Problem 8} \leavevmode 

\begin{enumerate}[label=\alph*.]
	\item Consider $S^{2}=\{x\in\mathbb{R}^{3}:\|x\|=1\}$ equipped with the are form $\omega_x(u,v)=\left<x,u\times v\right> $ (where $x\in S^{2}$, $u,v\in T_xS^{2}$ and $\times $ is the vector product). Use cylindrical coordinates to prove Darboux's theorem directly in this example.

	\item More generally: show that on a 2-dimensional manifold, any non-vanishing 1-form can be locally written as $fdg$, where $f$ and $g$ are smooth functions. Use this fact to give a proof of Darboux's theorem in 2 dimensions.
\end{enumerate}

\begin{proof}[Solution]\leavevmode
	\begin{enumerate}[label=\alph*.]
		\item Inspired in Da Silva, we wish to show that there are local coordinates $(\theta,z)$ of  $S^{2}$ such that $\omega_x(u,v)=d\theta\wedge dz$. Since 
	\end{enumerate}
\end{proof}

\end{document}
