\input{/Users/daniel/github/config/preamble-por.sty}
\input{/Users/daniel/github/config/thms-por.sty}

\usepackage[style=authortitle-terse,backend=bibtex]{biblatex}
\addbibresource{bibliography.bib}

\begin{document}

\begin{minipage}{\textwidth}
	\begin{minipage}{1\textwidth}
		Geometria Simpl\'etica \hfill Daniel González Casanova Azuela
		
		{\small Profs. Henrique Bursztyn and Leonardo Macarini\hfill\href{https://github.com/danimalabares/sg}{github.com/danimalabares/sg}}
	\end{minipage}
\end{minipage}\vspace{.2cm}\hrule

\vspace{10pt}
{\huge Lista 1}


\paragraph{Problem 1:} Let $V$ be a symplectic vector space ($\dim V=2n$), and $\Omega\in \Lambda^{2} V^{*}$be a skew-symmetric bilinear form. Show that $\Omega$ is nondegenerate iff $\Omega^{n} \neq 0$.

\begin{proof}[Solution]
	First suppose that $\Omega$ is nondegenerate. Then there are two vectors  $v,w\in V$ such that $\Omega(v,w)\neq 0$. Recall that for any $2n$ vectors we have:
	\[\Omega^n(x_1,x_2,\ldots,x_{2n-1},x_{2n})=\sum_{\sigma\in S_{2n}}\operatorname{sgn}(\sigma)\Omega(x_{\sigma(1)},x_{\sigma(2)})\ldots\Omega(x_{\sigma(2n-1)},x_{\sigma(2n)})\]
However,
	\[\Omega^n(v,w,\ldots,v,w)\]
has a much simpler expression since any term where $\Omega$ has a repeated entry vanishes. Further, since interchanging the entries of $\Omega$ changes it sign according to the order of the permutation, no terms in the sum cancel out and the result is nonzero. More explicitly, it is a nonzero integer multiple of some power of $\Omega(v,w)$.

	For the converse let's suppose that $\Omega$ is degenerate and show that $\Omega^{n}=0$. Suppose there is a vector $v_1$ such that $\Omega( v_0,w)=0$ for all $w\in V$ and complete to a basis $\{v_1,\ldots,v_n\}$. At this point I got stuck and found in \cite{lee}, Prop. 22.8 that interior multiplication is an antiderivation, yielding
	\[i_v(\Omega^n)=n(i_v\Omega)\wedge \Omega^{n-1}=0.\]
	(Here $i_x\alpha(y):=\alpha(x,y)$ for any vectors $x,y\in V$ and any 2-form $\alpha$.) The proof of this antiderivation property is more involved. It immediately yields that $\Omega^n(v_1,\ldots,v_n)=0$, making $\Omega^n=0$.

\end{proof}

\paragraph{Problem 2:} Let $(V,\Omega)$ be a symplectic vector space, and let $W\subseteq V$ be any linear subspace.
\begin{enumerate}[label=\alph*.]
	\item Show that $V_{W}=\frac{W}{W\cap W^{\Omega}}$ inherits a natural symplectic structure $\Omega_{W}$ uniquely determined by the condition $\pi^{*} \Omega_{W}=\Omega|_{W}$ (here $\pi:W\to W/(W\cap W^{\Omega}) $ is the quotient projection).
	
		(\textit{The space $(V_{W},\Omega_{W})$ is called the \textbf{reduced space}.})

	\item Suppose that $W$ is coisotropic, and let $L\subset V$ be lagrangian. Show that the image of $L\cap W$ via $\pi:W\to V_{W}$ is lagrangian in the reduced space.
\end{enumerate}

\begin{proof}[Solution]\leavevmode 
	\begin{enumerate}[label=\alph*.]
		\item Define
			\[\Omega_{W}([w_1],[w_2]):=\Omega(w_1,w_2)\]
			for any equivalence classes $[w_1],[w_2]\in V_{W}$. Let's check that this is well defined. Suppose $w_1'\in [w_1]$. Then $w_1-w_1'\in W\cap W^{\Omega}$ so $\Omega(w_1-w_1',w_2)=0$ since $w_2\in W$ and $w_1-w_1'$ is, in particular, in $W^{\Omega}$. So $\Omega(w_1,w_2)=\Omega(w_1',w_2)$.

			Recall that $\pi^{*} \Omega_{W}(w_1,w_2)=\Omega_{W}([w_1],[w_2])$. It is straightforward to check that $\Omega_{W}$ is the only symplectic form on $V_{W}$ satisfying $\pi^{*} \Omega_{W}=\Omega|_{W}$: if $\Omega_{W}'$ is another such form, then $\Omega'_{W}([w_1],[w_2])=\Omega|_{W}(w'_1,w'_2)=\Omega_{W}([w_1],[w_2])$ for any $w_1'\in [w_1]$ and $w_2'\in [w_2]$.

\item Since $W$ is coisotropic, we have $V_W=W/W^\Omega$ and $L^\Omega=L$. First notice that $\pi(L\cap W)\subseteq\pi(L\cap W)^{\Omega_W}$ since $L$ is lagrangian:
				\begin{align*}
					[w]\in\pi(L\cap W)&\implies \Omega_W([w],[\ell])=\Omega(w',\ell')=0				\end{align*}
for any representants of each equivalence class.

				I couldn't really prove the other contention…
				\begin{align*}
					\pi(L\cap W)^{\Omega_W} & =\{[w]\in W/W^\Omega:\Omega_W([w],[\ell])=0\;\forall [\ell]\in \pi(L\cap W)\} \\
					& =\{[w]\in W/W^\Omega:\Omega(w',\ell'
					)=0\;\forall \ell'\in[\ell]\in\pi(L\cap W), w'\in [w]\} \\
						& =\{[w]\in W/W^\Omega:\Omega(w',\ell'
						)=0\; \forall \ell'-\ell,w-w'\in W^\Omega\text{ \& } \ell\in\pi(L\cap W)\}\\
						&\subseteq\{[w]\in W/W^\Omega:w\in(L+(L\cap W))^\Omega\}
				\end{align*}
				Moreover,
				\begin{align*}
					(L\cap W)^\Omega & =L^\Omega+ W^\Omega =L+ W^\Omega\subseteq L + W
				\end{align*}
				since $L$ is lagrangian and $W$ coisotropic. Not sure if this was the correct way…
\end{enumerate}
\end{proof}

\paragraph{Problem 3:}  We saw in class that any symplectomorphism $T:V_1\to V_2$ defines a lagrangian subspace by its graph: $\Gamma_{T}:=\{(Tu,u):u\in V_1 \}\subset V_2\oplus \overline{V}_{1}$. (Recall that if $(V,\Omega)$ is a svs,  $\overline{V}$ denotes $(V,-\Omega)$.) So we think lagrangian subspaces of $V_2\oplus \overline{V}_{1}$ a generalizations of symplectomorphisms. We now see how to generalize their composition. 

Consider symplectic vector spaces  $V_1,V_2,V_3$ and $E=V_3\oplus \overline{ V}_{2}\oplus V_2\oplus \overline{V}_1$.
\begin{enumerate}[label=\alph*.]
	\item Show that $\Delta :=\{(v_3,v_2,v_2,v_1)\in E\} $ is coisotropic in $E$ and its reduction $E_{\Delta}$ can be identified with $V_3\oplus \overline{V}_{1}$.

	\item Given lagrangian subspaces $L_1\subset V_2\oplus \overline{V}_{1}$ and $L_2\subset  V_3\oplus \overline{V}_{2}$, define the \textit{\textbf{composition}} of $L_2$ and $L_1$ by
		\[L_2\circ L_1:=\{(v_3,v_1)|\exists v_2\in V\text{ s.t. } (v_3,v_2)\in L_2,(v_2,v_1)\in L_1\}. \]
		Show that $L_2\circ L_1$ is a lagrangian subspace of $V_3\oplus \overline{V}_{1}$. (\textit{Hint: show that the composition can be identified with the reduction of $L_2\times L_1\subset E$ with respect to $\Delta$}).

	\item Let $T_1:V_1\to V_2$ and  $T_2:V_2\to V_3$ be symplectomorphisms. Show that $\Gamma_{T_2\circ T_1}=\Gamma_{T_2}\circ \Gamma_{T_1}$.
\end{enumerate}

\begin{proof}[Solution]\leavevmode
	\begin{enumerate}[label=\alph*.]
		\item First let's compute $\Delta^\Omega$. Let $v=(v_3,v_2,v_2',v_1)\in E$ and $(w_3,w_2,w_2,w_1)\in\Delta^\Omega$. This means that
		\begin{align*}\Omega_3(v_3,w_3)-\Omega_2(v_2,w_2)+\Omega(v_2',w_2)-\Omega(v_1,w_1)&=0\\
		\iff\Omega_3(v_3,w_3)+\Omega_2(v_2'-v_2,w_2)-\Omega(v_1,w_1)&=0
		\end{align*}
		Letting $w_2=w_1$ and varying $w_3$ we see that $v_3=0$ by nondegeneracy of $\Omega$. Likeways, $v_1=0$ and $v_2'=v_2$. This shows that $\Delta^\Omega$ is coisotropic.

Now let's try to construct an isomorphism $E_{\Delta}=V_3\oplus \overline{V}_{1}$. Consider
\begin{align*}
	\varphi: \Delta &\longrightarrow V_3\oplus \overline{V}_{1} \\
	(v_3,v_2,v_2,v_1) &\longmapsto (v_3,v_1)
\end{align*}
which is clearly surjective and not injective, and its kernel is $ \Delta \cap \Delta^{\Omega}=\Delta^\Omega$. But $\ker \varphi=\{(0,v_2,v_2',0)\}$.

	\item As in the last exercise, we see that $(L_2\times L_1)_\Delta= L_2\circ L_1$ via the map
\begin{align*}
	\varphi: \Delta\cap(L_2\times L_1) &\longrightarrow V_2\circ V_1 \\
	(v_3,v_2,v_2,v_1) &\longmapsto (v_3,v_1).
\end{align*}
Further, since both $L_1$ and $L_2$ are lagrangian, so is their product. The conclusion follows from Problem 2b.

	\item  Perhaps I'm missing something, but
		\begin{align*}
			\Gamma_{T_2}\circ \Gamma_{T_1}&=\{(T_2v,u):T_1u=v\} =\{(T_2(T_1u),u):u\in V\} =\Gamma_{T_2\circ T_1}.
		\end{align*}
\end{enumerate}
\end{proof}

\paragraph{Problem 4:} Let $(V,J)$ be a complex vector space, let $\Omega$ be a sympletic structure on $V$. Show that $J$ and $\Omega$ are compatible iff there exists a hermitian inner product $h:V\times V\to \mathbb{C}$ such that $\Omega$ is its imaginary part. Show that any (complex) orthonormal basis of  $(V,h)$ can be extended to a symplectic basis of $(V,\Omega)$.

\begin{proof}[Solution]\leavevmode
	First suppose that $J$ and $\Omega$ are compatible, ie., $g(u,v):=\Omega(u,Jv)$ is an inner product. Define $h(u,v)=g(u,v)+i\Omega(u,v)$. Then  $h$ is the required hermitian inner product. Indeed:

	\begin{enumerate}
		\item The properties $h(u_1+u_2,v)=h(u_1,v)+h(u_1,v)$ and  $h(u,v_1+v_2)=h(u,v_1)+h(u,v_2)$ follows easily from linearity of $g$ and $\Omega$.
		
		\item Homegenity on the first argument, $h(\lambda u,v)=\lambda h(u,v)$, is also immediate.

		\item The property $h(u,\lambda v)=\bar{\lambda} h(u,v)$ follows easily from 2. and 4. since 
			\begin{align*}
				h(u,\lambda v)& =\overline{h(\lambda v, u)}\\
				& =\bar{\lambda} \overline{h(v,u)}\\
				& =\bar{\lambda} h(u,v)
			\end{align*}

		\item $h(u,v)=\overline{h(v,u)}$ is clear by anti-symmetry of $\Omega$:
			\begin{align*}
				h(u,v)& =g(u,v)+i\Omega(u,v)\\
				& =g(v,u)-i\Omega(v,u)\\
				& =\overline{h(v,u)}
			\end{align*}
	\end{enumerate}
	For the converse suppose that $h$ is an hermitian inner product such that $\Omega$ is its imaginary part. Then $g(u,v):=\Omega(u,Jv)$ is an inner product:
	\begin{enumerate}
		\item Linearity of $g$ is immediate from linearity of $\Omega$ and $J$.

		\item Symmetry follows from 
			 \begin{align*}
				g(u,v)&=\Omega(u,Jv)\\
				&=\Omega(-J^{2} u,Jv)\\
				& =-\Omega(J^{2} u,Jv)\\
				& =\Omega(Jv,J^{2} u)\\
				&=\Omega(v,Ju)\\
				&=g(v,u)
			\end{align*}
		provided $\Omega(u,v)=\Omega(Ju,Jv)$. This holds since $\Omega$ is the imaginary part of $h$ identifying $J$ with multiplication by $i$:
			\begin{align*}
				\Omega(Ju,Jv)&=\operatorname{Im}(h (Ju,Jv))\\
				& =\operatorname{Im}h(iu,iv)\\
				&=\operatorname{Im}(i\bar{i}h(u,v))\\
				&=\operatorname{Im}(h(u,v))\\
				&=\Omega(u,v).
			\end{align*}

		\item For positive-definiteness let $u\neq 0$. Then
			\begin{align*}
				g(u,u)&=\Omega(u,Ju)\\
				&=\operatorname{Im}(h(u,Ju))\\
				&=\operatorname{Im}(h(u,iu))\\
				&=\operatorname{Im}(ih(u,u))>0
			\end{align*}
			since $h(u,u)>0$.
	\end{enumerate}
	Now suppose that $\{v_1,\ldots,v_n\}$ is an orthonormal basis of  $(V,h)$. We show that $\{v_1,\ldots,v_n,Jv_1,\ldots,Jv_n\}$ is a symplectic basis of $(V,\Omega)$:
	\begin{align*}
		\Omega(v_i,v_j)&=\begin{cases}
			\operatorname{Im}(h(v_i,v_i))=\operatorname{Im}(1)=0,\qquad \qquad & i=j\\
			\operatorname{Im}(h(v_i,v_j))=0,\qquad &i\neq j
		\end{cases}\\
		\Omega(v_i,Jv_j)& =\operatorname{Im}(h(v_i,Jv_j))=\operatorname{Im}(ih(v_i,v_j))=\operatorname{Im}(i\delta_{ij})=\delta_{ij}\\
		\Omega(Jv_i,Jv_j)&=\begin{cases}
			\operatorname{Im}(h(Jv_i,Jv_i))=\operatorname{Im}(-1)=0,\qquad \qquad & i=j\\
			\operatorname{Im}(h(Jv_i,Jv_j))=\operatorname{Im}(-h(v_i,v_j))=0,\qquad &i\neq j.
		\end{cases}
	\end{align*}
\end{proof}

 \paragraph{Problem 5:} Consider the symplectic vector space $(\mathbb{R}^{2n},\Omega_0)$, where $\Omega_0(u,v)=-u^{\mathbf{T}} J_0v$. Check that its group of linear symplectomorphisms is given by $\operatorname{Sp}(2n)=\{A\in \operatorname{GL}(2n):A^{\mathbf{T}}J_0A=J_0\}.$ Show that $\operatorname{Sp}(2n)$ is a smooth submanifold of $\operatorname{GL}(2n)$ and that its tangent space at the identity $I\in \operatorname{GL}(2n)$ is given by $T_{I}\operatorname{Sp}(2n)=\{A:\mathbb{R}^{2n}\to \mathbb{R}^{2n}|A^{\mathbf{T}} J_0+J_0A=0\} $. Conclude that $\operatorname{Sp}(2n)$ has dimension $2n^{2} +n$. Verify also that $\operatorname{Sp}(2n)$ is not compact.

\begin{proof}[Solution]\leavevmode
	Suppose that $A$ is a linear symplectomorphism of $(\mathbb{R}^{2n},\Omega_0)$. Then $A^{*}\Omega_0=\Omega_0$ so 
	\[A^{*}\Omega_0(u,v) =\Omega_0(Au,Av)=-(Au)^{\mathbf{T}}J_0(Av)=-u^{\mathbf{T}}A^{\mathbf{T}}J_0(Av)\]
is equal to
		\[\Omega_0(u,v)=-u^{\mathbf{T}}J_0v\]
In terms of usual dot product of $\mathbb{R}^{2n}$, which we can denote by $\left<\cdot ,\cdot \right> $ momentarily, this means that
\begin{align*}
	\left<-u^{\mathbf{T}},A^{\mathbf{T}}J_0Av\right> &=\left<-u^{\mathbf{T}},J_0v\right>\\
	\iff\left<-u^{\mathbf{T}},A^{\mathbf{T}}J_0Av-J_0v\right> &=0
\end{align*}
for all $u\in\mathbb{R}^{2n}$, which means that $A^{\mathbf{T}}J_0Av=J_0v$ since dot product is nondegenerate. For the converse, if $A^{\mathbf{T}}J_0A=J_0$, we have
\begin{align*}
	\Omega(u,v)&=-u^{\mathbf{T}}J_0v=-u^{\mathbf{T}}A^{\mathbf{T}}J_0Av=-(Au)^{\mathbf{T}}J(Av)=\Omega(Au,Av_0)=A^*\Omega(uv).
\end{align*}

To show that $\operatorname{Sp}(2n)$ is a smooth manifold consider the map
\begin{align*}
	D: \operatorname{GL}(2n) &\longrightarrow \operatorname{GL}(2n) \\
	A &\longmapsto A^{\mathbf{T}}J_0A
\end{align*}
This map is a submersion at every point of $D^{-1}(J_0)=\operatorname{Sp}(2n)$ because its derivative is only a composition of linear isomorphisms. But something seems to be wrong because then it would be a sumbanifold of dimension $(2n)^2-(2n)^2=0$ (according to \cite{loring}, Thm 9.9, Regular level set theorem, where the dimension of a regular level set is the difference of the dimension of the domain manifold minus the dimension of the codomain manifold).

After consulting \href{https://math.stackexchange.com/questions/4299154/the-symplectic-group-as-a-submanifold-of-mathbbr2n-times-2n}{StackExchange} I have found that we may define the codomain of $D$ as the vector space of skew-symmetric matrices $\{A\in\operatorname{Mat}(2n):A^{\mathbf{T}}=-A\}$, which is of dimension $\frac{1}{2}(2n)(2n-1)$. This gives $\dim \operatorname{Sp}(2n)=(2n)^2-\frac{1}{2}n(n-1)=4n^2-\frac{4n^2}{2}+\frac{2n}{2}=2n^2+n$. {\color{magenta}But I still can't see what's wrong with the other map…}

Now let's find the tangent space at the identity. First recall the tangent space at the identity of a Lie group is isomorphic to its Lie algebra (\cite{lee}, Thm 8.37). Then we use that the exponential map is an isomorphism from the Lie algebra to the Lie group. This is because the differential of the exponential map at the identity is the identity (\cite{lee}, Prop. 20.8) and $\operatorname{exp}(0) =I\in\operatorname{GL}(2n)$, so $d_0\operatorname{exp}: T_0\operatorname{GL}(2n)\cong \operatorname{GL}(2n)\to T_{\operatorname{I}}\operatorname{Sp}(2n)$ is an isomorphism.

Now let's check that $\operatorname{exp}(W) =\operatorname{Sp}(2n)$, that is,
\[\operatorname{exp}(W) = \left\{ \operatorname{exp}(A):A^{\mathbf{T}}J_0+J_0A=0 \right\}\quad \overset{?}{=}\quad \{A\in\operatorname{GL}(2n):A^{\mathbf{T}}J_0A=J_0\}=\operatorname{Sp}(2n) \}\]
Borrowing a proof from \href{https://math.stackexchange.com/questions/445088/finding-the-lie-algebra-of-the-symplectic-lie-group}{StackExchange}, we have
\begin{align*}
	&A^{\mathbf{T}}J_0+J_0A =0\\
	\implies & A^{\mathbf{T}} J=-JA\\
	\implies & A^{\mathbf{T}} =J(-A)J^{-1}\\
	\implies &\operatorname{exp}(A)^{\mathbf{T}}=\operatorname{exp}(J(-A) J^{-1})\\
	\implies &\operatorname{exp}(A)^{\mathbf{T}} =J\operatorname{exp}(A)^{-1}J^{-1}\\
	\implies &\operatorname{exp}(A)^{\mathbf{T}} J\operatorname{exp}(A) =J
\end{align*}
Using that $\operatorname{exp}(-A) =\operatorname{exp}(A)^{-1}$ (\cite{lee}, Prop 20.8), that $\operatorname{exp}(A^{\mathbf{T}}) =\operatorname{exp}(A)^{\mathbf{T}}$ and that $\operatorname{exp}(J_0) =J_0$ (remain to check).

I still wonder why $\operatorname{Sp}(2n)$ is not compact…

\end{proof}

 \paragraph{Problem 6:} Consider the standard compatible triple $(\Omega_0,J_0,g_0)$ on $\mathbb{R}^{2n}$. Let $\operatorname{O}(2n)$ be the linear orthogonal group of $\mathbb{R}^{2n}$ (i.e., linear transformations preserving the canonical inner product $g_0$), and let $\operatorname{Sp}(2n)$ be the symplectic linear group. Through the identification $\mathbb{R}^{2n}\cong \mathbb{C}^{n}$ (as complex vector spaces), we may see $\operatorname{GL}(n,\mathbb{C})$ (the group of linear automorphisms of $\mathbb{C}^{n}$) as a subgroup of $\operatorname{GL}(2n,\mathbb{R})$ : a complex matrix $A+iB$ is identified with the real $2n\times 2n$ matrix
 \[\begin{pmatrix}A&-B\\B&A\end{pmatrix}\]
Let now $\operatorname{U}(n)\subset\operatorname{GL}(n,\mathbb{C})$ be the group of linear transformation preserving the natural hermitian inner product of $\mathbb{C}^{n}$. Show that the intersection of any two of the groups
\[\operatorname{Sp}(2n),\operatorname{O}(2n),\operatorname{GL}(n,\mathbb{C})\subset\operatorname{GL}(2n,\mathbb{R})\]
is $\operatorname{U}(n)$.

\begin{proof}[Solution]\leavevmode

	Since the standard hermitian product of $\mathbb{C}^{n}$ is given by $h_0=g_0+i\Omega_0$, it is immediate that a transformation $A\in\operatorname{Sp}(2n)\cap \operatorname{O}(2n)$ preserves $h_0$ and conversely:
	\[A^*h=A^*(g+i\Omega)=A^*g+iA^*\Omega=g+i\Omega=h\]
	provided that the pullback is complex-linear.

	For the next item recall that
	\[\operatorname{O}(2n)=\{A\in\operatorname{GL}(2n):A^{\mathbf{T}}A=I\},\qquad \operatorname{GL}(n,\mathbb{C})=\{A\in\operatorname{GL}(2n):AJ_0=J_0A\}\]
	again identifying $J_0$ with multiplication by $i$. Observe that this implies that $A\in\operatorname{O}(2n)\cap \operatorname{GL}(n,\mathbb{C})\implies  A\in\operatorname{Sp}(2n)$ since
	\[A^{\mathbf{T}}J_0A=A^{\mathbf{T}}A J_0=J_0.\]
	Likeways we see that $A\in\operatorname{Sp}(2n)\cap \operatorname{GL}(n,\mathbb{C})\implies \operatorname{O}(2n)$ since
	\[A^{\mathbf{T}}J_0A=J_0\iff A^{\mathbf{T}}A J_0=J_0\iff A^{\mathbf{T}}A=I\]
	since $J_0$ is invertible. Going back to the initial argument for matrices in $\operatorname{Sp}(2n)\cap A\in\operatorname{O}(2n)$, we see that in both cases $A\in\operatorname{U}(n)$.

	For the converse notice that it is also true that $A\in\operatorname{Sp}(2n)\cap \operatorname{O}(2n)\implies A\in\operatorname{GL}(n,\mathbb{C})$ since
	\[J_0=A^{\mathbf{T}}J_0A=A^{-1} J_0A\iff J_0A=A J_0.\]
\end{proof}

\paragraph{Problem 7:} Let $(V,\Omega)$ be a symplectic vector space, let $W\subseteq V$. Let $J$ be a $\Omega$-compatible complex structure and $g$ the corresponding inner product. Verify that $J(W^{\Omega} )=W^{\perp_{g}}$.
\begin{enumerate}[label=\alph*.]
	\item Use this fact to show that any coisotropic subspace of $V$ has an isotropic complement. In particular, any lagrangian subspace $L\subset V$ has a lagrangian complement $L'$, $V=L\oplus L'$.
	
\item Show that there is a natural identification $L'\cong L^{*}$, that induces a symplectomorphism $V\cong L\oplus L^{*}$, where $L\oplus L^{*}$ has the natural symplectic structure \[\left( (\ell,\alpha),(\ell',\alpha') \right)\longmapsto\alpha(\ell')-\alpha'(\ell)\].
\end{enumerate}

\begin{proof}[Solution]\leavevmode
	First let's check that $J(W^{\Omega} )=W^{\perp_{g}}$. Indeed,
	\begin{align*}
		J(W^{\Omega} )&= \{Jv:v\in W^{\Omega}\}\\
&=\{Jv:\Omega(v,w)=0\;\forall w\in W\} \\
&=\{Jv:-\Omega(w,v)\;\forall w\in W\} \\
&=\{Jv:\Omega(w,-v)\;\forall w\in W\} \\
&=\{Jv:\Omega(w,J^{2}v)\;\forall w\in W\}
	\end{align*}
re-write $Jv:=\tilde{v}$ using that $J$ is bijective:
	\begin{align*}
J(W^{\Omega)}&=\{\tilde{v}\in V:\Omega(w,J\tilde{v})=0\;\forall w\in W\} \\
&=\{ \tilde{v}\in V:g(\tilde{v},w)=0\;\forall w\in W\} \\
& =W^{\perp_{g}}
	\end{align*}
	\begin{enumerate}[label=\alph*.]
		\item Let $W$ be any coisotropic subspace. We know that $V=W\oplus W^{\perp_g}$ (supposing that $V$ is finite-dimensional), so it remains to show that $W^{\perp_g}$ is isotropic.
		Since $W$ is coisotropic, we have
		\[W^{\Omega}\subseteq W\implies J(W^{\Omega})=W^{\perp_g}\subseteq JW\]
		so it would be enough to show that
		\[JW\subseteq\big(J(W^{\Omega})\big)^{\Omega}=(W^{\perp_g})^{\Omega}.\]
		Let $w\in W$ and $w'\in W^{\Omega}$, so that $Jw\in JW$ and $Jw'\in J(W^{\Omega})$. Then
		\[\Omega(Jw,Jw')=\Omega(w,w')=0,\]
		which shows that $JW\subseteq\big(J(W^{\Omega})\big)^{\Omega}$.

	\item Out of time!
	\end{enumerate}


\end{proof}

\paragraph{Bonus problem: } [content…]

\printbibliography

\end{document}
