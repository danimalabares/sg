\input{/Users/daniel/github/config/preamble.sty}%This is available at github.com/danimalabares/config
\input{/Users/daniel/github/config/thms-eng.sty}%This is available at github.com/danimalabares/config

%\usepackage[style=authortitle-terse,backend=bibtex]{biblatex}
%\addbibresource{bibliography.bib}

\begin{document}

\begin{minipage}{\textwidth}
	\begin{minipage}{1\textwidth}
		Geometria Simpl\'etica \hfill Daniel González Casanova Azuela
		
		{\small Profs. Henrique Bursztyn e Leonardo Macarini\hfill\href{https://github.com/danimalabares/sg}{github.com/danimalabares/sg}}
	\end{minipage}
\end{minipage}\vspace{.2cm}\hrule

\vspace{10pt}
{\huge Lista 3}

\addcontentsline{toc}{subsection}{Problem 2}
\paragraph{Problem 2} Let $M$ be a symplectic manifold, $\Psi=(\psi^1,\ldots,\psi^k):M\to \mathbb{R}^{k}$ a smooth map, and $c$ a regular value. Consider a submanifold $N=\Psi^{-1}(c)\hookrightarrow M$.
\begin{enumerate}[label=\alph*.]
	
	\item[b.] Show that $N$ is symplectic if and only if the matrix $(c^{ij})$, with $c^{ij}=\{\psi^i,\psi^j\}$, is invertible for all $x\in N$.\end{enumerate}

\begin{proof}[Solution]\leavevmode
	(From \href{https://math.stackexchange.com/questions/4974063/level-set-submanifold-is-symplectic-iff-poisson-bracket-matrix-is-nonsingular?noredirect=1#comment10675364_4974063}{Jordan Payette in Stack Exchange}.) \hspace{0.5em} 
	First I recall an explanation I provided earlier for item a. of this question:

\begin{quotation}
		Since $M$ is the level set of a regular value, there are local coordinates of the form $(\psi^1,\ldots,\psi^k,x^{k+1},\ldots,x^{2n})$. Vectors tangent to $N$ are expressed only in terms of the vectors $\partial_{k+1},\ldots,\partial_{2n}$ and thus covectors that vanish on $T N$ are those which vanish on $\partial_{k+1},\ldots, \partial_{2n}$. This means that a basis for $\operatorname{Ann}(T N)$ is given by $d\psi^1,\ldots,d\psi^k$. These generators map to their hamiltonian vector fields under $(\omega^\flat)^{-1}$:
\[\left(\omega^\flat \right)^{-1}(d\psi^i)=X_{\psi^i}\]
So $T N^\omega= (\omega^\flat)^{-1}\operatorname{Ann}(TN)$ is generated by the $X_{\psi^i}$.
\end{quotation}

Hence $\operatorname{Ann}(TN)_m$ is symplectic when $\{\psi^i,\psi^j\}=\omega^\sharp$ is nonsingular.

Now remember that a subspace of a symplectic space is symplectic if and only if its symplectic orthogonal is symplectic (this is because  a subspace  $W\subset(V,\Omega)$ is symplectic iff $V=W\oplus W^\Omega=W^\Omega\oplus (W^\Omega)^\Omega$). Then solving the exercise ammounts to showing that $\operatorname{Ann}(TN)$ is symplectic.

\begin{upshot}\leavevmode
	The whole point is that $TN$ is symplectic iff $TN^\omega$ is symplectic. And $TN^\omega$ is symplectic when $\omega ^\sharp$ is symplectic and this is given by $\{\psi^i,\psi^j\}$.
\end{upshot}


\begin{idea5}{First attempt by Jordan Payette}[Wrong]\leavevmode
	
	Consider the following form on $M$ determined by the symplectic form $\omega$:
	\begin{align*}
		\pi: T^*M\times T^*M  &\longrightarrow \mathbb{R} \\
		(\alpha,\beta) &\longmapsto \omega_m(\omega ^\sharp\alpha,\omega ^\sharp\beta)
	\end{align*}
It is immediate that $\pi$ is nondegenerate when $\omega$ is. We will show that the induced map $i^*\pi$ on $N$, where $i:N\hookrightarrow M$ is the inclusion, is also nondegenerate. This furnishes $N$

\begin{enumerate}[label=\textbf{Step \arabic*}]
\item \begin{claim}\leavevmode
	A form $\eta:TX\times TX\to \mathbb{R}$ is nondegenerate iff
	\begin{align*}
		\pi_\eta: T^*X\times T^*X &\longrightarrow \mathbb{R} \\
		(\alpha,\beta) &\longmapsto \eta(\eta ^\sharp \alpha,\eta^\sharp\beta)
	\end{align*}
	is nondegenerate.
\end{claim}
\item Then we only need to show that
\begin{align*}
	\pi_{i^*\omega}: T^*N\times T^*N &\longrightarrow \mathbb{R} \\
	(i^*\alpha,i^*\beta) &\longmapsto \omega(\omega ^\sharp,\omega ^\sharp\beta)
\end{align*}
is nondegenerate.

\item Then we use that that $\mathsf{N}N\oplus T^*N=TM$ and that a form on a vector space decomposed as a direct sum is nondegenerate iff it is nondegenerate on any of the components. That is, it suffices to show that $\pi_{i^*\omega}$ is nondegenerate on $\mathsf{N}N\times \mathsf{N}N$.

\item There's another part concerning the Possion bracket.
\end{enumerate}
\end{idea5}
\end{proof}

\end{document}
